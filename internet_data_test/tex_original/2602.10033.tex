\documentclass[12pt]{article}

\usepackage{graphicx}
\usepackage{latexsym,amssymb}
\usepackage{amsthm}
\usepackage{indentfirst}
\usepackage{amsmath}
\usepackage{color}
\usepackage{xcolor}
\usepackage{fourier}
\usepackage[colorlinks=true,backref=page]{hyperref}

\usepackage{float}

\newcommand\SY[1]{{\noindent\color{red}  #1}}
\newcommand\sy[1]{\marginpar{\tiny \SY{#1}}}

\textwidth=16. true cm
\textheight=24. true cm
\voffset=-2. true cm
\hoffset = -1.5 true cm

\newtheorem{theoremalph}{Theorem}
\renewcommand\thetheoremalph{\Alph{theoremalph}}
\newtheorem*{Main Theorem}{Main Theorem}
\newtheorem{Coro}[theoremalph]{Corollary}
\newtheorem{Theorem}{Theorem}[section]
\newtheorem*{Theorem A}{Theorem A}
\newtheorem*{Theorem A'}{Theorem A'}
\newtheorem*{Theorem B'}{Theorem B'}
\newtheorem*{Thm}{Theorem}
\newtheorem*{Lem}{Lemma}
\newtheorem{Conj}{Conjecture}
\newtheorem*{Conjecture}{Conjecture}
\newtheorem{Definition}[Theorem]{Definition}
\newtheorem{Proposition}[Theorem]{Proposition}
\newtheorem{Lemma}[Theorem]{Lemma}
\newtheorem{Example}[Theorem]{Example}
\newtheorem{Criterion}[Theorem]{Criterion}
\newtheorem{Sublemma}[Theorem]{Sublemma}
\newtheorem{Question}{Question}
\newtheorem*{Principle}{Principle}
\newtheorem*{Problem}{Problem}
\newtheorem*{Notation}{Notation}
\newtheorem{Remark}[Theorem]{Remark}
\newtheorem*{Remarks}{Remarks}
\newtheorem*{Property}{Property}
\newtheorem{Remark-numbered}[Theorem]{Remark}
\newtheorem{Corollary}[Theorem]{Corollary}
\newtheorem*{Cor}{Corollary}
\newtheorem{Case}{Case}
\newtheorem{Subcase}{Subcase}[Case]
\newtheorem*{Claim}{Claim}
\newtheorem{Claim-numbered}[Theorem]{Claim}
\newtheorem*{Acknowledgements}{Acknowledgements}



\setcounter{tocdepth}{1}
\setcounter{secnumdepth}{3}
\setcounter{tocdepth}{2}

\begin{document}

	\title{Entropy formula for surface diffeomorphisms}

	\author{Yuntao Zang\footnote{
		Yuntao Zang is supported by National Key R\&D Program of China (2022YFA1005802) and NSFC (12201445, 12471186).
}}


	\maketitle
	\begin{abstract}

		\medskip
	Let $f$ be a $C^r$ ($r>1$) diffeomorphism on a compact surface $M$ with $h_{\rm top}(f)\geq\frac{\lambda^{+}(f)}{r}$ where $\lambda^{+}(f):=\lim_{n\to+\infty}\frac{1}{n}\max_{x\in M}\log \left\|Df^{n}_{x}\right\|$. We establish an equivalent formula for the topological entropy: $$h_{\rm top}(f)=\lim_{n\to+\infty}\frac{1}{n}\log\int_{M}\left\|Df^{n}_{x}\right\|\,dx.$$  Our approach builds on the key ideas developed in the works of Buzzi-Crovisier-Sarig (\emph{Invent. Math.}, 2022) and
	Burguet (\emph{Ann. Henri Poincar\'e}, 2024) concerning the continuity of the Lyapunov exponents.
	\end{abstract}

	\tableofcontents
	\section{Introduction}
	Let $f$ be a diffeomorphism on a compact surface $M$. Write $$\lambda^{+}(f):=\lim_{n\to+\infty}\frac{1}{n}\max_{x\in M}\log \left\|Df^{n}_{x}\right\|.$$ The sequence $\left(\max_{x\in M}\log \left\|Df^n_x\right\|\right)_{n\ge 1}$ is sub-additive, and hence the above limit exists. For surface diffeomorphisms with large entropy, we establish an equivalent formula for the topological entropy in terms of the \emph{volume growth of the tangent cocycle} :
	\begin{theoremalph}\label{main result}
		For any $C^r$ ($r>1$) diffeomorphism $f$ on a compact surface $M$ with $h_{\rm top}(f)\geq\frac{\lambda^{+}(f)}{r}$, we have $$h_{\rm top}(f)=\lim_{n\to+\infty}\frac{1}{n}\log\int_{M}\left\|Df^{n}_{x}\right\|\,dx.$$

	\end{theoremalph}
Let us first make some remarks about Theorem \ref{main result}.
	\begin{itemize}
		\item Note that the integral is taken with respect to the Lebesgue measure. The Lyapunov regular set (that is, the set of points with well-defined Lyapunov exponents) usually has zero Lebesgue measure and therefore does not contribute to the integral above. Even for ergodic volume-preserving Anosov systems-where Lebesgue almost every point has a positive Lyapunov exponent equal to the entropy of the volume measure (Lebesgue measure)-the integral appearing in Theorem~\ref{main result} does not converge to the entropy of the volume measure. Instead, it converges to the entropy of the measure of maximal entropy. In this sense, for sufficiently large time $n$, it is the \emph{exceptional} points that dominate the integral. The order of the operations $\log$ and $\int$ is thus crucial (see Corollary~\ref{order of log and integral}).


		\item In the special case $r=\infty$, no additional term is added to the topological entropy, and our result thus implies the main results of Kozlovski \cite{Koz98} for surface diffeomorphisms. Also note that Theorem~\ref{main result} is formulated directly in terms of the derivative $Df^{n}_{x}$, rather than the induced map $(Df^{n}_{x})^{\wedge}$ between exterior algebras of the tangent spaces which are considered by Kozlovski \cite{Koz98}. Although the equivalence of these two quantities for evaluating the topological entropy is not apparent from their respective forms, it is nonetheless natural to work directly with $Df^{n}_{x}$ in the two-dimensional setting, after a more careful analysis. The use of the derivative $Df^{n}_{x}$ in Theorem~\ref{main result} might provide a more effective approach for numerically estimating the topological entropy.
		\item The threshold $\frac{\lambda^{+}(f)}{r}$ for the topological entropy is typically not expected to be attained in Yomdin-type results. For instance:
		\begin{itemize}
			\item For $C^r$ ($r>1$) surface diffeomorphisms with $h_{\rm top}(f)>\frac{\lambda^{+}(f)}{r}$, Buzzi-Crovisier-Sarig \cite{BCS22'} (see also subsequent extensions by Buzzi-Luo-Yang \cite{BLY25}, Burguet-Luo-Yang \cite{BLY25'} and Zang \cite{Zan25})  showed that there are at most finitely many ergodic measures of maximal entropy.
			\item For $C^r$ ($r>1$) surface diffeomorphisms with $h_{\rm top}(f)>\frac{\lambda^{+}(f)}{r}$, Burguet \cite{Bur24} showed if $\mu_{n}$ is a sequence of ergodic measures with $h(f,\mu_{n})\to h_{\rm top}(f)$ and $\mu_{n}\to \mu$ for some invariant measure $\mu$, then $\mu$ is a measure of maximal entropy and the positive Lyapunov exponent of $\mu_{n}$ converges to that of $\mu$.
			\item For $C^r$ ($r>1$) surface diffeomorphisms, Burguet \cite{Bur24'} also showed that if Lebesgue almost every point has upper Lyapunov exponent strictly larger than $\frac{\lambda^{+}(f)}{r}$, then the system admits an SRB measure.
		\end{itemize}
		All these results rely on the assumption that certain invariants-such as the topological entropy or Lyapunov exponents-are \emph{strictly} larger than the threshold $\frac{\lambda^{+}(f)}{r}$. The critical value $\frac{\lambda^{+}(f)}{r}$ itself is delicate and remains poorly understood. In contrast, Theorem~\ref{main result} does not suffer from this limitation, as it concerns a formula for the topological entropy itself.
		\item \begin{itemize}
			\item The case of systems with low regularity ($r=1$) is currently not understood by us.
			\item The condition $h_{\rm top}(f)\geq\frac{\lambda^{+}(f)}{r}$ appears to be critical. There are examples for interval maps (see Kozlovski \cite{Koz98}) with
			$h_{\rm top}(f)<\frac{\lambda^{+}(f)}{r}$, in which the topological entropy vanishes while the volume growth of the tangent cocycle appearing in Theorem~\ref{main result} is positive. Although this does not immediately yield counterexamples in our setting of surface diffeomorphisms, we do not expect a fundamental gap between the situations of interval maps and surface diffeomorphisms. We also note that Kozlovski \cite[Page 2]{Koz98} mentioned, without providing a precise reference, that Misiurewicz constructed counterexamples for surface diffeomorphisms where the topological entropy is strictly smaller than the volume growth.
			\item Recently, S. Ben Ovadia and Burguet \cite{BeB25} studied the Viana conjecture in arbitrary dimensions using a refined high-dimensional version of Yomdin theory. This work may shed light on possible extensions of Theorem~\ref{main result} to higher dimensions.
		\end{itemize}


	\end{itemize}

	\medskip
	The study of the volume growth of the tangent cocycle:
	$$\frac{1}{n}\log\int\left\|(Df^{n}_{x})^{\wedge}\right\|\,d x$$ appears to go back to Sacksteder-Shub\cite{SaS78}. Here $(Df^{n}_{x})^{\wedge}$ denotes the linear map induced by $Df^{n}_{x}$ on the exterior algebras of the tangent spaces $T_{x}M$ and $T_{f^{n}x}M$, and $\left\|\cdot\right\|$ is the operator norm induced by the Riemannian metric.

Based on Pesin theory, an inequality was obtained by Przytycki \cite{Prz80} for $C^r$ ($r>1$) diffeomorphisms and by Newhouse \cite{New88} for $C^r$ ($r>1$) maps on a compact manifold $M$ of arbitrary dimension:
\begin{equation}\label{Przytycki upper bound of entropy}
	\liminf_{n\to+\infty}\footnote{Przytycki established the inequality in terms of a $\limsup$ and remarked at the end of the paper that it remains valid with $\liminf$.}\frac{1}{n}\log\int\left\|(Df^{n}_{x})^{\wedge}\right\|\,d x\geq h_{\rm top}(f).
\end{equation}
Subsequently, Kozlovski \cite{Koz98}(see also Klapper-L. S. Young \cite{KlY95}) showed that, for $C^\infty$ diffeomorphisms, by Yomdin theory \cite{Yom87}, a lower bound for the topological entropy is provided, i.e.,
\begin{equation}\label{Kozlovski lower bound of entropy}
	\limsup_{n\to+\infty}\frac{1}{n}\log\int\left\|(Df^{n}_{x})^{\wedge}\right\|\,d x\leq h_{\rm top}(f).
\end{equation}
 Combining (\ref{Przytycki upper bound of entropy}) and (\ref{Kozlovski lower bound of entropy}), we obtain an equality for the topological entropy. One might wonder what the $C^{r}$ version of the above formula is for finite $r$. Indeed, in the $C^{r}$ setting, one may expect an inequality that combines the topological entropy with the Yomdin term:  \begin{equation}\label{Kozlovski lower bound of entropy with Yomdin term}
 	\limsup_{n\to+\infty}\frac{1}{n}\log\int\left\|(Df^{n}_{x})^{\wedge}\right\|\,d x\leq\footnote{Although this formula is not stated explicitly in Kozlovski \cite{Koz98} or elsewhere (as far as we know), it can be deduced from the arguments in \cite{Koz98} as an application of Yomdin theory in the $C^{r}$ case.} h_{\rm top}(f)+\frac{\lambda^{+}(f)}{r}.
 \end{equation} In the $C^{\infty}$ case, the Yomdin term $\frac{\lambda^{+}(f)}{r}$ vanishes, which yields the inequality (\ref{Kozlovski lower bound of entropy}). In contrast, for finite $r$, the inequality (\ref{Kozlovski lower bound of entropy with Yomdin term}) cannot be reduced to inequality (\ref{Kozlovski lower bound of entropy}), as the Yomdin term does not vanish in general.  Rather than the \emph{additive} bound of the topological entropy and the Yomdin term appearing in (\ref{Kozlovski lower bound of entropy with Yomdin term}), we obtain a \emph{maximum} bound that yields a substantial improvement (see Theorem~\ref{main theorem of decomposition of volume growth of derivatives}):

 \medskip	\begin{equation}\label{Our lower bound of entropy convex with Yomdin term}
 \limsup_{n\to+\infty}\frac{1}{n}\log\int_{M}\left\|Df^{n}_{x}\right\|\,dx\leq \max \left\{h_{\rm top}(f),\,\, \frac{\lambda^{+}(f)}{r}\right\}.
 \end{equation}\medskip

 If $h_{\rm top}(f)\geq\frac{\lambda^{+}(f)}{r}$, inequality (\ref{Our lower bound of entropy convex with Yomdin term}) immediately implies inequality (\ref{Kozlovski lower bound of entropy}). Together with inequality (\ref{Przytycki upper bound of entropy}) which holds for any $C^r$ ($r>1$) diffeomorphism, this yields Theorem~\ref{main result}.

There are also related results in the partially hyperbolic setting (see the works of Saghin \cite{Sag14}, Yang-Zang \cite{YaZ23} and Guo-Liao-Sun-Yang \cite{GLS18}), as well as in the setting of random dynamical systems (see Ma \cite{MaX20}).


\medskip
\medskip

	The order of the operations $\log$ and $\int$ is crucial in the following sense.
	\begin{Coro}\label{order of log and integral}
		For any $C^r$ ($r>1$) diffeomorphism $f$ on a compact surface $M$ with $h_{\rm top}(f)>\frac{\lambda^{+}(f)}{r}$ \footnote{Note that here we have to assume a strict inequality. Otherwise, Burguet's result \cite{Bur24'} does not guarantee the existence of an SRB measure.}, if $$\limsup_{n\to+\infty}\frac{1}{n}\int_{M} \log\left\|Df^{n}_{x}\right\|\,dx=\limsup_{n\to+\infty}\frac{1}{n}\log\int_{M} \left\|Df^{n}_{x}\right\|\,dx, $$ then there exists finitely many ergodic SRB measures. Moreover, each of them is a measure of maximal entropy.
	\end{Coro}
	\begin{proof}
		By Reverse Fatou Lemma (see Lemma \ref{Reverse Fatou Lemma}) and Theorem \ref{main result},

			$$\begin{aligned}
		\int_{M} \limsup_{n\to+\infty}\frac{1}{n}\log\left\|Df^{n}_{x}\right\|\,dx
			&\geq \limsup_{n\to+\infty}\frac{1}{n}\int_{M} \log\left\|Df^{n}_{x}\right\|\,dx\\
			&=\limsup_{n\to+\infty}\frac{1}{n}\log\int_{M} \left\|Df^{n}_{x}\right\|\,dx\\
			&=h_{\rm top}(f)\\
			&>\frac{\lambda^{+}(f)}{r}.
		\end{aligned}$$
 We claim that for Lebesgue almost every $x$, \begin{equation}\label{leb almost every points has exponent equal to topo entropy}
		\limsup_{n\to+\infty}\frac{1}{n}\log\left\|Df^{n}_{x}\right\|= h_{\rm top}(f).
		\end{equation} Because otherwise, since the integral of the left side is no less than $h_{\rm top}(f)$, there must exist some set $A$ with positive Lebesgue measure such that $$\limsup_{n\to+\infty}\frac{1}{n}\log\left\|Df^{n}_{x}\right\| > h_{\rm top}(f)>\frac{\lambda^{+}(f)}{r},\quad\forall x\in A.$$ By Burguet's result \cite{Bur24'}, there is an ergodic SRB measure $\mu$ whose positive Lyapunov exponent is strictly larger than $h_{\rm top}(f)$. This is a contradiction since the positive exponent of an ergodic SRB measure coincides with its entropy.

		Again, by Burguet's result \cite{Bur24'}, under condition (\ref{leb almost every points has exponent equal to topo entropy}), there are countably many ergodic SRB measures $\{\mu_{i}\}_{i\in \mathbb{N}}$ such that Lebesgue almost every $x$ lies in the basin of some $\mu_{i}$ and each $\mu_{i}$ carries maximal entropy.  Indeed, in our setting there are only finitely many such ergodic SRB measures, as a direct consequence of the main results of Buzzi-Crovisier-Sarig \cite{BCS22'}, which assert that there are at most finitely many ergodic measures of maximal entropy.

	\end{proof}
	We also provide a formula expressing the topological entropy in terms of the \emph{volume growth of sub-manifolds} which extends the classical results in Yomdin theory \cite{Yom87} in $C^{\infty}$ setting.

	The \emph{volume} (i.e., arc length) of a curve $\sigma: [0,1]\to M$ is denoted by ${\rm Vol}\left(\sigma\right)$.

\begin{theoremalph}
		\label{entropy coincides with volume growth}
		For any $C^r$ ($r>1$) diffeomorphism $f$ on a compact surface $M$ with $h_{\rm top}(f)\geq\frac{\lambda^{+}(f)}{r}$, we have

		$$\begin{aligned}
		h_{\rm top}(f)
			&=\lim_{n\to+\infty}\frac{1}{n}\log \sup_{\sigma}{\rm Vol}\left(f^{n}\left(\sigma\right)\right)\\
			&=\sup_{\sigma}\limsup_{n\to+\infty}\frac{1}{n}\log{\rm Vol}\left(f^{n}\left(\sigma\right)\right)\\
			&=\sup_{\sigma}\liminf_{n\to+\infty}\frac{1}{n}\log{\rm Vol}\left(f^{n}\left(\sigma\right)\right)
		\end{aligned}$$	where the supremum is taken over all $C^{r}$ embedded curves $\sigma: [0,1]\to M$ with $\left\|\sigma\right\|_{C^{r}}\leq 1$ and $\left\|d_{t}\sigma\right\| \geq 1/2, \,t\in[0,1]$ \footnote{These conditions ensure that the curves are neither degenerate nor highly oscillatory and are imposed only for technical convenience. One may allow more general families.}.
\end{theoremalph}



We make some comments.
\begin{itemize}
	\item In general, beyond our $C^{r}$ setting together with the condition $h_{\rm top}(f)\geq \frac{\lambda^{+}(f)}{r}$, it is quite subtle to establish an equivalence between the two kinds of volume growth of sub-manifolds in Theorem \ref{entropy coincides with volume growth} : $$\limsup_{n\to+\infty}\frac{1}{n}\log \sup_{\sigma}{\rm Vol}\left(f^{n}\left(\sigma\right)\right)\stackrel{?}{ = }\sup_{\sigma}\limsup_{n\to+\infty}\frac{1}{n}\log{\rm Vol}\left(f^{n}\left(\sigma\right)\right).$$ This question was implicitly raised in an unpublished manuscript of Buzzi \cite[Remark 1.2]{Buz96}.
	\item If the topological entropy is strictly larger than $\frac{\lambda^{+}(f)}{r}$, then there exists an invariant measure $\mu$ of maximal entropy (\cite{Bur24,BCS25}). Moreover, by a result of Cogswell \cite{Cog00} (see also Zang \cite{Zan22}), the supremum in Theorem~\ref{entropy coincides with volume growth} is attained by the local unstable manifold $W^{u}_{\rm loc}(x)$ for $\mu$-almost every point $x$.
	\item It is worth noting that Theorem~\ref{entropy coincides with volume growth} actually implies Theorem~\ref{main result} via a Fubini-type argument (see the discussions in Section~\ref{proof of main theorems}).
	However, we choose to prove Theorem~\ref{main result} using a more fundamental convex bound, namely Theorem~\ref{main theorem of decomposition of volume growth of derivatives}. This approach offers a deeper insight into the mechanism governing the volume growth of tangent cocycle. There is no essential difference between these two approaches, since both ultimately follow from Theorem~\ref{main theorem of decomposition of volume growth of submanifolds}.

\end{itemize}

We believe that the ideas of Theorem~\ref{main result} and Theorem~\ref{entropy coincides with volume growth} might extend to other topological invariants (see Llibre--Saghin \cite{LlS09} for a survey of various invariants measuring dynamical complexity), such as the growth rate of hyperbolic periodic points. This has been established by Burguet~\cite{Bur20} in the $C^{\infty}$ setting and it was conjectured by Burguet in the $C^{r}$ case:

\begin{Conjecture}[\cite{Bur20}]
	For any $C^r$ ($r>1$) diffeomorphism $f$ on a compact surface $M$ with $h_{\rm top}(f)>\frac{\lambda^{+}(f)}{r}$. For any $0<\delta< h_{\rm top}(f)$, $$h_{\rm top}(f)=\limsup_{n\to+\infty}\frac{1}{n}\log \#\mathcal{P}^{n}_{\delta}$$ where $\mathcal{P}^{n}_{\delta}$ is the set of $n$-periodic points with Lyapunov
	exponents $\delta$-away from zero.
\end{Conjecture}

	\section{Convex bounds for volume growth}
	In this section, we present the main results used to prove Theorem~\ref{main result} and Theorem~\ref{entropy coincides with volume growth}.



\subsection{Mechanism behind convex bounds for volume growth}

	\begin{Theorem}\label{main theorem of decomposition of volume growth of submanifolds}
		Let $f$ be a $C^r$ ($r>1$) diffeomorphism on a compact surface $M$. There is some $\alpha\in[0,1]$ such that $$\limsup_{n\to+\infty}\frac{1}{n}\log \sup_{\sigma}{\rm Vol}(f^n\sigma)\leq \alpha\cdot h_{\rm top}(f) + (1-\alpha)\cdot\frac{\lambda^{+}(f)}{r}.$$ Equivalently, $$\limsup_{n\to+\infty}\frac{1}{n}\log \sup_{\sigma}{\rm Vol}(f^n\sigma)\leq\max \left\{h_{\rm top}(f),\,\, \frac{\lambda^{+}(f)}{r}\right\}.$$ Here the supremum is taken over all $C^{r}$ embedded curves $\sigma: [0,1]\to M$ with $\left\|\sigma\right\|_{C^{r}}\leq 1$ and $\left\|d_{t}\sigma\right\| \geq 1/2, \,t\in[0,1]$.
	\end{Theorem}
	\begin{Theorem}\label{main theorem of decomposition of volume growth of derivatives}
		Let $f$ be a $C^r$ ($r>1$) diffeomorphism on a compact surface $M$. There is some $\alpha\in[0,1]$ such that $$\limsup_{n\to+\infty}\frac{1}{n}\log\int_{M}\left\|Df^{n}_{x}\right\|\,dx\leq \alpha\cdot h_{\rm top}(f) + (1-\alpha)\cdot\frac{\lambda^{+}(f)}{r}.$$ Equivalently, $$\limsup_{n\to+\infty}\frac{1}{n}\log\int_{M}\left\|Df^{n}_{x}\right\|\,dx\leq\max \left\{h_{\rm top}(f),\,\, \frac{\lambda^{+}(f)}{r}\right\}.$$
	\end{Theorem}

The reason for presenting Theorems \ref{main theorem of decomposition of volume growth of submanifolds} and \ref{main theorem of decomposition of volume growth of derivatives} in a convex-combination form is that our proof actually yields a more refined upper bound, in which the coefficient $\alpha\in[0,1]$ reflects how the volume growth is distributed between the entropic contribution and the Lyapunov-type contribution (the Yomdin term). In particular, the argument implicitly determines an admissible choice of $\alpha$ and thus contains strictly more information than the crude bound given by the maximum. However, making this choice explicit would require introducing a substantial amount of additional technical machinery. For this reason, we do not pursue an explicit description of $\alpha$ here in the statements. As a consequence, the statements of the theorems may be equivalently weakened to the simpler form involving $\max\{h_{\rm top}(f),\lambda^{+}(f)/r\}$.


	These convex-type upper bounds originate from the idea that dynamical invariants such as volume growth measure complexity from two different perspectives: a global one and a local one. To illustrate this idea, let us consider the volume growth of a curve $\sigma$ and write $${\rm Vol}(f,\sigma):=\limsup_{n\to+\infty}\frac{1}{n}\log {\rm Vol}(f^n\sigma).$$ At time $n$ and at scale $\varepsilon$, the volume of $f^{n}(\sigma)$ can be estimated by decomposing it into two factors:

	$${\rm Vol}(f^n\sigma)\leq \overbrace{\left(\text{number of $(n,\varepsilon)$ dynamical balls to cover $\sigma$}\right)}^{\text{Global complexity}}\times \overbrace{\max_{x}{\rm Vol}\left(f^n\left(\sigma\cap \left(B\left(x,n,\varepsilon\right)\right)\right)\right)}^{\text{Local complexity}}$$

	The global complexity is controlled by the topological entropy, while the classical Yomdin theory \cite{Yom87} shows that the local complexity grows at most at the rate $\lambda^{+}(f)/r$. Combining these two estimates yields the classical Yomdin bound:
	\begin{equation}\label{classical Yomdin bound}
		{\rm Vol}(f,\sigma)\leq h_{\rm top}(f) + \frac{\lambda^{+}(f)}{r}.
	\end{equation}

	However, this estimate is in general quite loose, since it implicitly assumes that the global and the local complexities are simultaneously present at all times.

	The key insight behind the convex bound in Theorem~\ref{main theorem of decomposition of volume growth of submanifolds} is that these two sources of complexity need not act at the same time. Instead, the time interval $[0,n)$ can be decomposed into two major types of periods:
	\begin{itemize}
		\item Geometric times (with proportion $\alpha$): during which the curve stretches across the manifold in a globally expanding and essentially straight manner;
		\item Neutral times (with proportion $1-\alpha$): during which the curve remains confined inside a single dynamical ball and develops complexity only through local folding.
	\end{itemize}

	\begin{figure}[htbp]
		\centering
		\includegraphics[width=1\textwidth]{GeometricNeutral}
		\caption{Growth at geometric and neutral times}
		\label{Pic expansion at Geometric and Neutral times}
	\end{figure}


	During geometric times, one only needs to count the number of dynamical balls that the curve visits, which contributes an entropy-type growth, while no significant local complexity is generated inside each ball.
	In contrast, during neutral times, the curve stays inside a single dynamical ball, so the global complexity vanishes, and the growth is entirely governed by local effects, captured by the Yomdin term $\lambda^{+}(f)/r$. See Figure \ref{Pic expansion at Geometric and Neutral times}.

	This separation of time scales explains the convex structure of the resulting bound and clarifies the mechanism responsible for the improvement over the classical Yomdin estimate. As a consequence, the classical Yomdin bound \eqref{classical Yomdin bound} can be refined to the following inequality:  $${\rm Vol}(f,\sigma)\leq \alpha\cdot h_{\rm top}(f) + (1-\alpha)\cdot\frac{\lambda^{+}(f)}{r}.$$


The separation of time into geometric and neutral components was first introduced by Buzzi-Crovisier-Sarig \cite{BCS22} and was further developed and systematized by Burguet \cite{Bur24}. This idea has since been deepened and applied in several related contexts; see, for instance, \cite{BLY25', BLY25, LuY25}.

We emphasize that all these works are formulated in the presence of a reference invariant measure, and the quantities under consideration are measure-theoretic invariants, such as metric entropy, Lyapunov exponents, or the uniform length of unstable manifolds w.r.t. some ergodic measures. In contrast, our approach focuses on purely topological quantities, namely the volume growth of sub-manifolds and of the tangent cocycle. The absence of a reference measure leads to several substantial technical difficulties. We mention here two of the main ones.

\begin{itemize}
	\item \emph{The presence of periodic sources \footnote{A periodic point $p$ is called a \emph{periodic source} if both Lyapunov exponents of $p$ are positive.}.}
	When a reference hyperbolic measure $\mu$ is given, the analysis may be restricted to $\mu$-typical points, which are not influenced by periodic sources. In our setting, however, when considering, for example, the volume growth of a curve $\sigma$, a large portion of points in $\sigma$ may spend a long time near periodic sources. During such intervals, which still correspond to geometric times, the curve needs time to escape from the "trapping region" of these sources. As a consequence, one must carefully control the amount of topological entropy produced during these geometric times, avoiding an unnecessary decomposition of the curve into exponentially many pieces before it has moved away from periodic sources. To address this issue, we introduce a finer decomposition of time and define the notion of \emph{trapping times}, which allows us to prevent an excessive creation of entropy (see Proposition \ref{grow rate for major component in finite step n}).

	\item \emph{Lack of uniformity in the absence of a reference measure.}
	Without a reference measure, it becomes more delicate to justify that no local complexity is generated during geometric times in a uniform way. We overcome this difficulty by using, in Proposition \ref{volume growth for time n}, an upper semi-continuity property for sub-additive sequences, which provides the necessary control.
\end{itemize}

	\subsection{Proofs of Theorem \ref{main result} and Theorem \ref{entropy coincides with volume growth}}\label{proof of main theorems}

	We now directly apply the two convex bounds established above to prove Theorem \ref{main result} and Theorem \ref{entropy coincides with volume growth}, in order to provide the reader with a clear overview of the global framework.

	\begin{proof}[Proof of Theorem \ref{main result}]

	The inequality
	$$ h_{\rm top}(f)\leq\liminf_{n\to+\infty}\frac{1}{n}\log\int_{M}\left\|Df^{n}_{x}\right\|\,d x$$ is, in a certain sense, already well known (see Przytycki \cite{Prz80}, Kozlovski \cite{Koz98}). Its proof relies only on the $C^r$ ($r>1$) regularity and follows as a consequence of Pesin theory. The only distinction from the classical formulations is that we work directly with the derivative $Df^{n}_{x}$, rather than with the induced map $(Df^{n}_{x})^{\wedge}$ acting on the exterior algebras of the tangent spaces. This difference is natural and expected in our setting, since we restrict ourselves to dimension two. For the reader's convenience and completeness, we reproduce this part of the argument in Theorem~\ref{volume of tangent cocycles bounds entropy}.

	The reverse inequality is provided by Theorem \ref{main theorem of decomposition of volume growth of derivatives}: $$\limsup_{n\to+\infty}\frac{1}{n}\log\int_{M}\left\|Df^{n}_{x}\right\|\,dx\leq \alpha\cdot h_{\rm top}(f) + (1-\alpha)\cdot\frac{\lambda^{+}(f)}{r}\leq h_{\rm top}(f).$$

	The proof of Theorem \ref{main result} is complete.
	\end{proof}



	\begin{proof}[Proof of Theorem \ref{entropy coincides with volume growth}]

		In the following arguments, the notation $\sup_{\sigma}$ refers to the supremum taken over all
		$C^{r}$ embedded curves $\sigma:[0,1]\to M$ with
		$\left\|\sigma\right\|_{C^{r}}\le 1$ and $\left\|d_{t}\sigma\right\| \geq 1/2, \,t\in[0,1]$.

		By the work of Cogswell \cite{Cog00} (see also Zang \cite{Zan22}), based purely on Pesin theory, for any hyperbolic ergodic measure $\mu$ and for $\mu$-almost every point $x$, $$h(f,\mu)\leq \liminf_{n\to+\infty}\frac{1}{n}\log {\rm Vol}\left(f^{n}\left(W^{u}_{\rm loc}(x)\right)\right)\leq\sup_{\sigma}\liminf_{n\to+\infty}\frac{1}{n}\log {\rm Vol}\left(f^{n}\left(\sigma\right)\right)$$ where $W^{u}_{\rm loc}(x)$ denotes the local unstable manifold at $x$.

		Hence, by the variational principle,

		$$h_{\rm top}(f)\leq\sup_{\sigma}\liminf_{n\to+\infty}\frac{1}{n}\log {\rm Vol}\left(f^{n}\left(\sigma\right)\right)\leq\liminf_{n\to+\infty}\frac{1}{n}\log\sup_{\sigma}{\rm Vol}\left(f^n(\sigma)\right).$$

		The reverse inequality is provided by Theorem \ref{main theorem of decomposition of volume growth of submanifolds},

		 $$\begin{aligned}
		\sup_{\sigma}\limsup_{n\to+\infty}\frac{1}{n}\log {\rm Vol}\left(f^{n}\left(\sigma\right)\right)
			&\leq \limsup_{n\to+\infty}\frac{1}{n}\log\sup_{\sigma}{\rm Vol}\left(f^n(\sigma)\right)\\
			&\leq \alpha\cdot h_{\rm top}(f) + (1-\alpha)\cdot\frac{\lambda^{+}(f)}{r}\\
			&\leq h_{\rm top}(f).
		\end{aligned}$$


		The proof of Theorem \ref{entropy coincides with volume growth} is complete.

	\end{proof}

	Let us remark that Theorem~\ref{main result} can also be derived from Theorem~\ref{entropy coincides with volume growth}.
	 Indeed, from the proof of Theorem~\ref{main theorem of decomposition of volume growth of derivatives} given in Section~\ref{Upper bound of the volume growth rate}, one obtains
	 \begin{equation}\label{volume growth of tangent space is less than that of sub-manifolds}
	 	\limsup_{n\to+\infty}\frac{1}{n}\log\int_{M}\left\|Df^{n}_{x}\right\|\,dx\leq\limsup_{n\to+\infty}\frac{1}{n}\log \sup_{\sigma}{\rm Vol}(f^n\sigma).
	 \end{equation} More precisely, by a Fubini-type argument (see Section~\ref{Upper bound of the volume growth rate}), there exists a constant $K>0$ such that for any $n\ge 1$, $$\int_{M}\left\|Df^{n}_{x}\right\|\,dx\leq K \cdot \sup_{\sigma}{\rm Vol}(f^n\sigma).$$ As a consequence, Theorem~\ref{entropy coincides with volume growth} combined with \eqref{volume growth of tangent space is less than that of sub-manifolds} yields $$	\limsup_{n\to+\infty}\frac{1}{n}\log\int_{M}\left\|Df^{n}_{x}\right\|\,dx\leq h_{\rm top}(f).$$

	 The reverse inequality ($\geq$) follows from pure Pesin theory (see Theorem~\ref{volume of tangent cocycles bounds entropy}).

	\medskip
	\medskip

	It remains to prove Theorem \ref{main theorem of decomposition of volume growth of submanifolds} (Section \ref{Volume growth of the sub-manifolds}) and Theorem \ref{main theorem of decomposition of volume growth of derivatives} (Section \ref{Volume growth of the tangent cocycle}).


\section{Preliminaries}
In this section, we collect several notions and basic facts that will be used throughout the paper.
We briefly recall Lyapunov exponents and their associated invariant splittings, and then introduce the induced dynamics on the projective tangent bundle, which provides a convenient framework for describing the measure-theoretic invariants.
\subsection{Lyapunov exponents}
The topological entropy of $f$ is denoted by $h_{\rm top}(f)$ and the metric entropy of an invariant measure $\mu$ is denoted by $h(f, \mu)$. By Oseledets Multiplicative Ergodic Theorem, there is an invariant set $\mathcal{R}$ (called the \emph{Lyapunov regular set}) with total measure (i.e., $\mu(\mathcal{R})=1$ for any invariant measure $\mu$) such that for any $x\in\mathcal{R}$, there are a splitting (called the  \emph{Oseledets splitting}) $T_{x}M=E^{1}\oplus E^{2}\oplus\cdots\oplus E^{l}$ and finitely many numbers (called the \emph{Lyapunov exponents}) $\lambda_{1}(x)>\lambda_{2}(x)>\cdots>\lambda_{l}(x)$ such that for any nonzero vector $v\in E^{j}$, we have $$\lim_{n\to \pm\infty}\frac{1}{n}\log\left\|Df^{n}_{x}(v)\right\|=\lambda_{j}.$$

In our special setting (surface diffeomorphism), given an invariant measure $\mu$, there are at most two well defined Lyapunov exponents for $\mu$-almost every $x$. The larger Lyapunov exponent $\lambda_{1}(x)$ is denoted by $\lambda^{+}(f,x)$ and it can be also defined as $$\lambda^{+}(f,x)=\lim_{n\to+\infty}\frac{1}{n}\log\left\|Df^{n}_{x}\right\|.$$ Similarly, the smaller Lyapunov exponent is denoted by $\lambda^{-}(f,x)$. We rewrite the Oseledets splitting at $x$ as $$T_{x}M=E^{+}_{x}\oplus E_{x}^{-}.$$ An invariant measure $\mu$ is called \emph{hyperbolic} if for $\mu$-a.e. $x$, one Lyapunov exponents of $x$ is positive and the other is negative. For an ergodic measure $\mu$ with $h(f,\mu)>0$, $\mu$ is hyperbolic by Ruelle's inequality.  We define $$\lambda^{+}(f, \mu):=\int\lambda^{+}(f,x) d\,\mu(x),\quad \lambda^{-}(f, \mu):=\int\lambda^{-}(f,x) d\,\mu(x).$$ If $\mu$ is ergodic, then $\lambda^{\pm}(f, \mu)=\lambda^{\pm}(f,x)$ for $\mu$-a.e. $x$.


\subsection{Tangent dynamics}\label{tangent dynamics}
With the Riemannian structure inherited from the tangent bundle $TM$, we denote the projective tangent bundle of $M$ by  $$\hat{M}:=\left\{(x, [v]):~x\in M, \,[v] \text{ is the linear subspace of $T_{x}M$ generated by the unit vector $v$}\right\} .$$ Let $\pi:\hat{M}\to M$ be the natural projection. Let $\hat{f}: \hat{M}\to \hat{M}$ be the induced map of $f$ (also called the \emph{canonical lift}) defined by $$\hat{f}(x, [v]):=\left(f(x), [Df_{x}(v)]\right).$$ If $f$ is of class $C^{r}$, then $\hat{f}$ is of class $C^{r-1}$.

Let $\rho:\hat{M}\to\mathbb{R}$ be the continuous function defined by $$\rho(x, [v]):=\log\left\|Df_{x}\left(v\right)\right\|.$$

Given an $\hat{f}$-invariant measure $\hat{\mu}$, the \emph{projection} of $\hat{\mu}$ is defined by $\mu:=\hat{\mu}\circ\pi^{-1}$ and $\hat{\mu}$ is called the \emph{lift} of $\mu$. $\hat{\mu}$ is called an \emph{unstable lift} of $\mu$ if for $\hat{\mu}$-almost every point $\hat{x}$, $$\lim_{n\to +\infty}\frac{1}{n}\sum_{i=0}^{n-1}\rho\left(\hat{f}^{i}\left(\hat{x}\right)\right)>0.$$

We define $$\lambda(\hat{f},\hat{\mu}):=\int_{\hat{M}}\rho d\,\hat{\mu}.$$ We list some basic properties.

\begin{itemize}
	\item If $\hat{\mu}_{n}\xrightarrow{\text{weak $\ast$}}\hat{\mu}$, then $\lambda(\hat{f},\hat{\mu}_{n})\to \lambda(\hat{f},\hat{\mu})$. This is a consequence of the continuity of the function $\rho$.
	\item Let $\mu$ be an invariant measure with $\lambda^{+}(f,x)>0\geq\lambda^{-}(f,x)$ for $\mu$-almost every $x$. Then $\mu$ has a unique unstable lift $\hat{\mu}^{+}$ which can be defined by: $$\hat{\mu}^{+}:=\int_{\hat{M}}\delta_{(x, E^{+}_{x})} d\,\mu(x).$$ Moreover (refer to \cite[Proposition 3.8]{BCS25}, $$\lambda^{+}(f,\mu)=\lambda(\hat{f},\hat{\mu})\quad\text{if and only if}\quad\hat{\mu}=\hat{\mu}^{+}.$$

	For more detailed discussions on the lifted dynamics, we refer to \cite[Section 3.4]{BCS22} and the references therein.


\end{itemize}

\subsection{Notations and conventions}
\begin{itemize}
	\item Whenever we say that $\sigma:[0,1]\to M$ is a $C^{r}$ curve, we always assume that it is an embedded curve.
	\item We identify $\sigma$ with its image whenever there is no ambiguity; in particular, we freely write $x\in\sigma$.
	\item Once a curve $\sigma$ is fixed, for any $x\in\sigma$, we denote by $\hat{x}=(x,[v])\in\hat{M}$  where $v$ is the unit tangent vector of $\sigma$ at $x$.
	\item Given $\hat{x}:=(x,[v])\in \hat{M}$ where $v$ is a unit vector in $T_{x}M$, we shall slightly abuse notation and write $$Df(\hat{x}):=Df_{x}(v)\in T_{f(x)}M.$$
	\item Define $$\left\|Df\right\|:=\max_{x\in M}\left\|Df_{x}\right\|, \quad \left\|Df^{-1}\right\|:=\max_{x\in M}\left\|Df^{-1}_{x}\right\|.$$ Note that $\left\|Df\right\|, \left\|Df^{-1}\right\| \geq 1$.
	\item Throughout the paper, the notation $\sup_{\sigma}$ refers to the supremum taken over all
	$C^{r}$ embedded curves $\sigma:[0,1]\to M$ with
	$\left\|\sigma\right\|_{C^{r}}\le 1$ and $\left\|d_{t}\sigma\right\| \geq 1/2, \,t\in[0,1]$.
\end{itemize}


	\section{Volume growth of the sub-manifolds}\label{Volume growth of the sub-manifolds}
	The purpose of this section is to establish Theorem~\ref{main theorem of decomposition of volume growth of submanifolds} via two key technical results: Proposition~\ref{grow rate for major component in finite step n} and Proposition~\ref{volume growth for time n}.
	\begin{itemize}
		\item Proposition~\ref{grow rate for major component in finite step n} separates the volume growth into the entropic contribution and the Lyapunov-type contribution (the Yomdin term) via Burguet's reparametrizations, a variant of Yomdin's scheme.
		\item Proposition~\ref{volume growth for time n} refines Proposition~\ref{grow rate for major component in finite step n} by controlling the error term arising from entropic contributions, more precisely, the local volume growth at geometric times in Proposition~\ref{grow rate for major component in finite step n}. It shows that this error term is negligible in the asymptotic regime.
	\end{itemize}


	\subsection{Reparametrization}

	Let $\sigma: [0,1]\to M$ be a $C^{r}$ curve. For an integer $s\ge 1$, the quantity $\left\|d^{s}\sigma\right\|$ denotes the supremum norm of the $s$-th derivative of $\sigma$, namely
	$$
	\left\|d^{s}\sigma\right\|
	:=
	\sup_{t\in[0,1]} \left\|d^{s}_{t}\sigma\right\|,
	$$
	where the norm is taken with respect to the Riemannian metric on $M$. We define the $C^{r}$ norm by $$
	\left\|\sigma\right\|_{C^{r}}
	:=
	\max_{1\le s\le r}\,
	\left\|d^{s}\sigma\right\|.$$ We say that $\sigma$ is \emph{bounded} if $$\max_{2\le s\le r}\,\left\|d^{s}\sigma\right\|\leq\frac{1}{6}\cdot\left\|d\sigma\right\|.$$ Note, consequently, we have
\begin{equation}\label{Property of boundedness}
	\left\|d\sigma\right\|\leq 2\cdot\left\|d_t\sigma\right\|,\,\forall t\in[0,1].
\end{equation} Moreover, the curve $\sigma$ is said to be \emph{$\varepsilon$-bounded} if $$\left\|d\sigma\right\|\leq \varepsilon.$$ Roughly speaking, a bounded curve is one that is nearly straight, whereas an $\varepsilon$-bounded curve is nearly straight and, in addition, has small length.

We often call an affine map from $[0,1]$ to itself a \emph{reparametrization}.

The following proposition is a reparametrization result in the spirit of Yomdin theory, developed by Burguet \cite[Lemma 12]{Bur24}. It shows that an $\varepsilon$-bounded curve can be decomposed, via affine reparametrizations, into finitely many pieces on which the image under $f$ remains bounded. Moreover, the number of such pieces and the sizes of the reparametrizations are quantitatively controlled by the expansion rates along the curve.

The ceiling function $\left\lceil x\right\rceil$ assigns to each real number $x$ the smallest integer $n$ such that $n\ge x$.

	\begin{Proposition}\cite[Lemma 12]{Bur24}\label{Burguet-Yomdin reparametrization}
		Let $f$ be a $C^r$ ($r>1$) diffeomorphism on a compact manifold $M$. There is a small constant $\varepsilon_{f}>0$ such that for any $\varepsilon_{f}$-bounded $C^r$ curve $\sigma :[0,1]\to M$ and any two integers $a,b\in\mathbb{Z}$, there is a family $\Theta$ of affine maps from $[0,1]$ to itself with the following properties:
		\begin{enumerate}
			\item $\left\{ x \in\sigma :~ \left\lceil\log\left\|Df_{x}\right\|\right\rceil=a, \,\left\lceil\log\left\|Df\left(\hat{x}\right)\right\|\right\rceil=b \right \}\subset \cup_{\theta\in \Theta} \sigma\circ\theta$,
			\item $f\circ \sigma\circ\theta$ is bounded for any $\theta\in\Theta$,
			\item $|\theta'|\leq e^{-\frac{a-b+3r}{r-1}}$ for any $\theta\in\Theta$,
			\item $\log\#\Theta\leq  A_{r} + \frac{a-b}{r-1}$ where $A_{r}$ is a universal constant depending only on $r$.
		\end{enumerate}

	\end{Proposition}
	\begin{Remark}
	 Compared with \cite[Lemma 12]{Bur24}, we slightly modify the third property by adding the term " $3r$ " in the exponent. This modification is acceptable, since the contraction factor is only required to depend on $r$, and any adjustment by a uniform constant depending solely on $r$ does not affect the validity of the result. The role of the additional term " $3r$ " is purely technical (see the remarks following Proposition~\ref{grow rate for major component in finite step n} for details).
	\end{Remark}


The following result, due to Burguet \cite[Lemma 8]{Bur24'}, can be regarded as an $n$-step and local version of the reparametrization result stated in Proposition~\ref{Burguet-Yomdin reparametrization}.

For $x\in M$, $n\in\mathbb{N}$ and $\varepsilon>0$, recall the usual \emph{$(n,\varepsilon)$-dynamical ball} at $x$ is defined by $$B(x,n,\varepsilon):=\{y\in M:\ d\left(f^{k}(y),f^{k}(x)\right)<\varepsilon \ \text{ for all } 0\le k<n\}.$$
		\begin{Proposition}\cite[Lemma 8]{Bur24'}\label{Corollary 2 Burguet-Yomdin reparametrization}
		Let $f$ be a $C^r$ ($r>1$) diffeomorphism on a compact manifold $M$. For any integer $p\geq 1$, there are a small constant $\varepsilon_{f^{p}}>0$ and a large constant $B_{p}$ such that for any  $\varepsilon_{f^{p}}$-bounded $C^r$ curve $\sigma :[0,1]\to M$, any $x\in\sigma$ and any $n\in\mathbb{N}$, there is a family $\Theta$ of affine maps from $[0,1]$ to itself with the following properties:
		\begin{enumerate}
			\item $B(x,n,\varepsilon_{f^{p}})\cap\sigma\subset \cup_{\theta\in \Theta} \sigma\circ\theta$,
			\item $f^{k}\circ \sigma\circ\theta$ is $\varepsilon_{f^{p}}$-bounded for any $\theta\in\Theta$ and $0\leq k\leq n$,

			\item $$\log\#\Theta\leq B_{p}+n\cdot A_{r}+\frac{1}{r-1}\cdot\frac{1}{p}\sum_{k=0}^{n-1}\left(\left\lceil\log\left\|Df^{p}_{f^{k}(x)}\right\|\right\rceil-\left\lceil\log\left\|Df^{p}\left(\hat{f}^{k}\left(\hat{x}\right)\right)\right\|\right\rceil\right)$$ where $A_{r}$ is a universal constant depending only on $r$.
		\end{enumerate}

	\end{Proposition}
\begin{Remark}
Compared with \cite[Lemma 8]{Bur24'}, we insert ceiling operations in the third property. This modification is acceptable, since the resulting discrepancy is at most linear in $n$ and can therefore be absorbed into the constant $A_r$. Also, to simplify notation, we assume that the constant $A_{r}$ appearing here is the same as the one in Proposition~\ref{Burguet-Yomdin reparametrization}.
\end{Remark}


	\subsection{Decomposition of the volume growth}
	Recall that $\rho:\hat{M}\to\mathbb{R}$ is the continuous function $$\rho(x, [v]):=\log\left\|Df_{x}\left(v\right)\right\|$$ where $v$ is a unit vector in $T_{x}M$. Define another continuous function $\rho':\hat{M}\to\mathbb{R}$ (introduced by Burguet \cite{Bur24}) by $$\rho'(x,[v]):=\rho(x, [v])-\frac{1}{r}\log\left\|Df_{x}\right\|.$$

	\begin{Definition}
		We call $n\geq 1$   a \emph{geometric time} of $\hat{x}=(x,[v])$  if for any $0\leq k\leq n-1$,
	\begin{equation}\label{definition of geometric time}
		\sum_{i=k}^{n-1}\rho'\left(\hat{f}^{k}(\hat{x})\right)\geq n-k.
	\end{equation}
	We let $E(\hat{x})$ denote the set of all geometric times of $\hat{x}$.
	\end{Definition}
	Roughly speaking, the existence of infinitely many geometric times for a point $\hat{x}$ suggests that its projection $x=\pi(\hat{x})$ has a positive upper Lyapunov exponent, at least of order one. At first glance, this may appear to be an overly strong requirement in the definition of geometric times. However, the precise size of the expansion rate used in this definition is in fact irrelevant: since our arguments ultimately concern the dynamics of large iterates of $\hat{f}$ (see the proof of Theorem~\ref{main theorem of decomposition of volume growth of submanifolds}), any fixed normalization of the expansion threshold leads to an equivalent notion for our purposes.


	\begin{Lemma}\label{property of geometric time}
		Let $\hat x\in\hat M$. Suppose that $m\ge1$ is not a geometric time of $\hat x$, and let
		$n<m$ be the largest geometric time smaller than $m$ (set $n=0$ if there is no geometric time before $m$). Then
		$$\sum_{i=n}^{m-1}\rho'\left(\hat f^{i}(\hat{x})\right)<m-n.$$
	\end{Lemma}

	\begin{proof}
		Since $m$ is not a geometric time, there exists $k\in\{0,\dots,m-1\}$ such that
		$$\sum_{i=k}^{m-1}\rho'\left(\hat f^{i}(\hat{x})\right)<m-k.$$
		Choose $k$ minimal with this property.
		\begin{itemize}
			\item If $k<n$, since $n$ is a geometric time,
			$$\sum_{i=k}^{n-1}\rho'\left(\hat f^{i}(\hat{x})\right)\ge n-k.$$ By the minimality of $k$, $$\sum_{i=n}^{m-1}\rho'\left(\hat f^{i}(\hat{x})\right)\ge m-n.$$ Hence
			$$\sum_{i=k}^{m-1}\rho'\left(\hat f^{i}(\hat{x})\right)\ge (n-k)+(m-n)=m-k,$$ a contradiction.


			\item  If $k>n$, since $k$ is not a geometric time, there exists
			$\ell\in\{0,\dots,k-1\}$ such that
			$$\sum_{i=\ell}^{k-1}\rho'\left(\hat f^{i}(\hat{x})\right)<k-\ell.$$
			Adding this to the inequality at $k$ yields with $\ell<k$
			$$\sum_{i=\ell}^{m-1}\rho'\left(\hat f^{i}(\hat{x})\right)<m-\ell,$$ contradicting the minimality of $k$.
		\end{itemize}
Therefore $k=n$, and the conclusion follows.
	\end{proof}


	Given a subset $E\subset \mathbb{N}$ and a number $L\in\mathbb{N}$, we define
	$$E^{L}:=\bigcup [i,j]$$ where the union runs over all $i,j\in E$ with $0\leq j-i\leq L$. Any element of $E^{L}(\hat{x}):=\left(E(\hat{x})\right)^{L}$ lies at distance at most $L$ from a geometric time. Since this discrepancy is uniformly bounded by $L$, it does not affect the asymptotic estimates as $n\to+\infty$. We can view $E^{L}(\hat{x})$ as the set of \emph{$L$-expanded geometric times} of $\hat{x}$.


Let $\varepsilon_{f}\gg \varepsilon_{f^{p}}$ be the small constants in Proposition \ref{Burguet-Yomdin reparametrization} w.r.t. $f, f^{p}$.

A periodic point $p$ of $f$ is called a \emph{periodic source} if both Lyapunov exponents of $p$ are positive. We list all periodic sources by $z_1, z_{2}, z_{3},\cdots$ (since there are at most countably many of them).


Given a $C^{r}$ curve $\sigma:[0,1]\to M$ and a point $x\in\sigma$, recall that we denote by $\hat{x}=(x,[v])\in\hat{M}$  where $v$ is the unit tangent vector of $\sigma$ at $x$. Given any $m,n\in \mathbb{N}$ with $m\leq n$, any $E\subset [0,n)$ and a family of integers $\beta:=\left\{\left(\beta'_{i},\, \beta''_{i}\right)\right\}_{i\in [0,n)}$, we define

$$\begin{aligned}
	O^{p,m,L}_{n, E,\beta}:=\Big\{x\in\sigma:~&  \left(E(\hat{x})\cap [0,n)\right)^{L}=E,\,\,\exists 1\leq s\leq n, \text{ s.t. } x\in B(z_{s}, m, \frac{\varepsilon_{f}}{100\left\|Df\right\|}) \text{ and }\\ &  \frac{1}{p}\left\lceil\log\left\|Df^{p}_{f^{i}(x)}\right\|\right\rceil= \beta'_{i}, \,\frac{1}{p}\left\lceil\log\left\|Df^{p}\left(\hat{f}^{i}\left(\hat{x}\right)\right)\right\|\right\rceil=\beta''_{i},\, i\in [m,n)\cap E,\\ & \left\lceil\log\left\|Df_{f^{i}(x)}\right\|\right\rceil=\beta'_{i}, \,\left\lceil\log\left\|Df\left(\hat{f}^{i}\left(\hat{x}\right)\right)\right\|\right\rceil=\beta''_{i},\, i\in [0,n)\setminus\left([m,n)\cap E\right)\Big \}.
\end{aligned}$$
If $m=0$, we define
$$\begin{aligned}
	O^{p,0,L}_{n, E,\beta}:=\Big\{x\in\sigma:~& \left(E(\hat{x})\cap [0,n)\right)^{L}=E,\,\,\forall 1\leq s\leq n, \text{ s.t. } x\notin B(z_{s},  \frac{\varepsilon_{f}}{100\left\|Df\right\|}) \text{ and }\\&  \frac{1}{p}\left\lceil\log\left\|Df^{p}_{f^{i}(x)}\right\|\right\rceil= \beta'_{i}, \,\frac{1}{p}\left\lceil\log\left\|Df^{p}\left(\hat{f}^{i}\left(\hat{x}\right)\right)\right\|\right\rceil=\beta''_{i},\, i\in [m,n)\cap E,\\ & \left\lceil\log\left\|Df_{f^{i}(x)}\right\|\right\rceil=\beta'_{i}, \,\left\lceil\log\left\|Df\left(\hat{f}^{i}\left(\hat{x}\right)\right)\right\|\right\rceil=\beta''_{i},\, i\in [0,n)\setminus\left([m,n)\cap E\right)\Big \}.
\end{aligned}$$


	\begin{Proposition}\label{grow rate for major component in finite step n}
		Let $f$ be a $C^r$ ($r>1$) diffeomorphism on a compact surface $M$. Consider the following setting:
		\begin{itemize}
			\item any $L,m,n, p\in\mathbb{N}$ with $m\leq n$,
			\item any $\varepsilon_{f}$-bounded $C^{r}$ curve $\sigma:[0,1]\to M$,

			\item any  $O^{p,m,L}_{n, E,\beta}$ with $\beta:=\{(\beta'_{i}, \beta''_{i})\}_{i\in [0,n)}$.
		\end{itemize}
		 Then we have
		 $$\begin{aligned}
		 	\log{\rm Vol}\left(f^{n}\left(O^{p,m,L}_{n, E,\beta}\right)\right)
		 	&\leq \#\left([m,n)\cap E\right)\cdot h_{\rm top}(f)+\left(m + \#\left([m,n)\setminus E\right)\right)\cdot\frac{\log\left\|Df\right\|}{r}\\
		 	&+\sum_{k\in [m,n)\cap E}\frac{\beta'_{k}-\beta''_{k}}{r-1}+ C_{n,L}.
		 \end{aligned}$$ where the error term $C_{n,L}$ satisfies $$\limsup_{L \to +\infty}\limsup_{n \to +\infty}\frac{C_{n,L}}{n}=C_{r}$$ for some universal constant $C_{r}$ depending only on $r$.

	\end{Proposition}
The proof of Proposition~\ref{grow rate for major component in finite step n} is rather involved. We therefore first explain the underlying idea.

The set $O^{p,m,L}_{n,E,\beta}$ represents a portion of the curve $\sigma$ on which all points share the same collection of (L-expanded) geometric times $E$ (we call the complementary times the \emph{neutral times}) and the same derivative data along their orbits. For simplicity, we assume that $\sigma$ intersects only one periodic source, namely a repeller $z$.

We decompose the time interval $[0,n)$ into three parts \footnote{In the proof of Proposition~\ref{grow rate for major component in finite step n}, an additional part, referred to as \emph{tail times}, is introduced for technical reasons.}
: trapping times, (partial) geometric times, and (partial) neutral times, $$[0,m),\quad [m,n)\cap E,\quad [m,n)\setminus E.$$

	\begin{itemize}
		\item Trapping times $[0,m)$. During this period, the set $O^{p,m,L}_{n,E,\beta}$ remains confined in a small neighborhood of the repeller $z$. Consequently, its growth is purely local \footnote{Positivity of the local volume growth is not excluded even near a repeller; it may occur when oscillatory effects dominate the geometric repelling of the region. See Example~\ref{example repeller positive volume growth}.}.
		 The corresponding growth rate is bounded by the Yomdin term $\frac{\lambda^{+}(f)}{r}$.
		\item Geometric times $[m,n)\cap E$.  At these times, $O^{p,m,L}_{n,E,\beta}$ exhibits both a \emph{global} growth, bounded by $\#\left([m,n)\cap E\right)\cdot h_{\rm top}(f)$, and a \emph{local} growth, bounded by

		$$\sum_{k\in [m,n)\cap E}\frac{\beta'_{k}-\beta''_{k}}{r-1}.$$

		In Proposition~\ref{grow rate for major component in finite step n}, we retain the latter term temporarily and later show, in Proposition~\ref{volume growth for time n}, that its contribution is negligible.
		\item Neutral times $[m,n)\setminus E$. At neutral times, the growth of $O^{p,m,L}_{n,E,\beta}$ is again purely local, with growth rate bounded by the Yomdin term $\frac{\lambda^{+}(f)}{r}$.
	\end{itemize}

It is worth emphasizing that, at neutral times, the curve does not necessarily fail to grow. Rather, by definition of geometric times, the growth rate is modest—specifically, it is bounded by a constant strictly less than $1$ (see the definition of geometric times (\ref{definition of geometric time})). The precise threshold is irrelevant for our purposes; what matters is that this bound is uniform and does not depend on the dynamics of $f$. In practice, we absorb this contribution by the additional "$3r$" term appearing in Proposition~\ref{Burguet-Yomdin reparametrization}.


\begin{proof}[Proof of Proposition \ref{grow rate for major component in finite step n}]
	We divide the proof into three steps.
		\medskip
		\medskip

\noindent\textbf{\large Step 1: An inductive construction of reparametrizations.}
	\medskip

 We first define some numbers:
\begin{itemize}
	\item Let $n_E$ be the maximal geometric time in $[0,n)$, that is, the largest element of $E\cap[0,n)$. If no such geometric time exists, we set $n_{E}=0$.

	\item Let $\mathcal{A}$ be the subset of $[0,n_{E}]$ obtained by removing all the integers $k$ such that $$k, \, k+1\in [m,n_{E}]\cap E.$$
	We enumerate the elements of $\mathcal{A}$ as $$0=a_{1}<a_{2}<\cdots<a_{l_{0}}=n_{E}.$$
	\item We consider several parts  \footnote{$n_E$ is regarded as both a geometric time and a tail time, purely for technical convenience.} in $[0,n)$. Write $\mathcal{A}_{i}:=\{a_{1},\cdots,a_{i}\}\subset\mathcal{A}$.
	\begin{itemize}
		\item {\bf Trapping times}: $\Gamma^{1}_{i}:=\mathcal{A}_{i}\cap [0,m)$,
		\item {\bf Geometric times} \footnote{We emphasize that the set $\Gamma^{2}_{i}$ is not meant to represent all geometric times. For technical reasons, it only records the right endpoints of the geometric intervals in $[m,n_{E}]\cap E$.}: $\Gamma^{2}_{i}:=\left(\mathcal{A}_{i}\cap E\right)\setminus [0,m)$,
		\item {\bf Neutral times}: $\Gamma^{3}_{i}=\mathcal{A}_{i}\setminus\left(\Gamma^{1}_{i}\cup\Gamma^{2}_{i}\right)$,
		\item {\bf Tail times}: $[n_{E},n)$.
	\end{itemize}
\end{itemize}

Recall that $1\gg\varepsilon_{f}\gg\varepsilon_{f^{p}}>0$ are the small constants in Proposition \ref{Burguet-Yomdin reparametrization} w.r.t. $f$ and  $f^{p}$. Let $B_{p}$ be the constant in Proposition \ref{Corollary 2 Burguet-Yomdin reparametrization}. We may further assume that $\varepsilon_{f^{p}}$ is sufficiently small and $B_{p}$ is sufficiently large such that
\begin{itemize}
	\item for any $k\in\mathbb{N}$, the minimal cardinality of a $(k,\varepsilon)$-spanning set of $M$ is bounded above: $$\log r(f,k,\varepsilon_{f^{p}})  \footnote{ Recall (see \cite{Wal82} for basic background in ergodic theory) that a set $\Omega\subset M$ is called \emph{$(k,\varepsilon)$-spanning} if for every $x\in M$,
		there exists $y\in \Omega$ such that such that
		$$
		d\bigl(f^i(x),f^i(y)\bigr)<\varepsilon
		\quad\text{for all } i=0,1,\dots,k-1.$$ Denote by $r(f,k,\varepsilon)$ the minimal cardinality of a $(k,\varepsilon)$-spanning set of $M$.}\leq B_{p}+k\left(h_{\rm top }(f)+1\right),$$
	\item there is a family $\hat{\Theta}$ \footnote{Such a family $\hat{\Theta}$ can be obtained by adapting the argument in Lemma~\ref{Decomposition into varepsilon bounded curves}, yielding the bound $\#\hat{\Theta}\lesssim \frac{\varepsilon_{f}}{\varepsilon_{f^{p}}}$.}
	 of affine maps from $[0,1]$ into itself with $$\#\hat{\Theta}\leq B_{p},\quad\bigcup_{\theta\in\hat{\Theta}}\theta\left([0,1]\right)=[0,1]$$ such that for any $\varepsilon_{f}$-bounded curve $\widetilde{\sigma}$ and any $\theta\in\hat{\Theta}$, $\widetilde{\sigma}\circ\theta$ is an $\varepsilon_{f^{p}}$-bounded curve.
\end{itemize}

Define $$\Delta:=\left\{a_{k}:~a_{k+1}\in\Gamma^{2}_{l_{0}}\right\}.$$

	Given $1\leq s\leq n$ and a family of integers $\gamma:=\left\{\left(\gamma'_{i},\, \gamma''_{i}\right)\right\}_{i\in [0,n)}$, we define:
$$\begin{aligned}
	\Omega^{p,m,L,s}_{n, E,\gamma}:=\Big\{&x\in\sigma:~  \left(E(\hat{x})\cap [0,n)\right)^{L}=E,\,x\in B(z_{s}, m, \frac{\varepsilon_{f}}{100\left\|Df\right\|}) \text{ and }\\ &  \frac{1}{p}\left\lceil\log\left\|Df^{p}_{f^{i}(x)}\right\|\right\rceil= \gamma'_{i}, \,\frac{1}{p}\left\lceil\log\left\|Df^{p}\left(\hat{f}^{i}\left(\hat{x}\right)\right)\right\|\right\rceil=\gamma''_{i},\, i\in \left(\left([m,n)\cap E\right)\setminus \Gamma^{2}_{l_{0}}\right)\cup \Delta,\\ & \left\lceil\log\left\|Df_{f^{i}(x)}\right\|\right\rceil=\gamma'_{i}, \,\left\lceil\log\left\|Df\left(\hat{f}^{i}\left(\hat{x}\right)\right)\right\|\right\rceil=\gamma''_{i},\, i\in \left(\left([0,n)\setminus\left([m,n)\cap E\right)\right)\cup\Gamma^{2}_{l_{0}}\right)\setminus \Delta\Big \}.
\end{aligned}$$

The definition of $\Omega^{p,m,L,s}_{n,E,\gamma}$ differs from that of $O^{p,m,L}_{n,E,\beta}$ in two respects: first, it fixes a periodic source $z_s$; second, it modifies the definition at times belonging to $\Delta$ and $\Gamma^{2}_{l_0}$ for technical convenience. These times constitute only a small portion of $[0,n)$ (at most $2n/L$) and therefore do not affect the asymptotic estimates.


Next we carry out an inductive construction of a family of reparametrizations to cover $\Omega^{p,m,L,s}_{n, E,\gamma}$.


\begin{Claim}
	There are families $\{\Theta\}_{1\leq i\leq l_{0}}$ of affine maps from $[0,1]$ to itself such that for any $1\leq i\leq l_{0}$,


		\begin{enumerate}
		\item  $\Omega^{p,m,L,s}_{n, E,\gamma}\subset \cup_{\theta\in \Theta_{i}} \sigma\circ\theta$,
		\item for each $\theta_{i}\in\Theta_{i}$ ($i>1$) with $a_{i}\notin\Gamma^{2}_{i}$, there are some $\theta_{i-1}\in \Theta_{i-1}$ and some affine map $\theta:[0,1]\to[0,1]$ with $|\theta'|\leq e^{-\frac{\gamma'_{a_{i-1}}-\gamma''_{a_{i-1}}+3r}{r-1}}$ such that $\theta_{i}=\theta_{i-1}\circ\theta$,
		\item for any $\theta\in\Theta_{i}$, $f^{a_{i}}\circ\sigma\circ\theta$ is $\varepsilon_{f}$-bounded,
		\item we have \footnote{By convention, we set $[a,b)=\emptyset$ if $a>b$.}
		$$\log\# \Theta_{i}\leq  \#\left([m,a_{i}]\cap E\right)\cdot h_{\rm top}(f)+ \sum_{k<a_{i}}\frac{\gamma'_{k}-\gamma''_{k}}{r-1}+ \#\Gamma^{2}_{i}\cdot 2B_{p}+ \left(a_{i}+1\right)\cdot A_{r}$$	where $A_{r}$ is the constant given in Proposition~\ref{Burguet-Yomdin reparametrization}, which coincides with that in Proposition~\ref{Corollary 2 Burguet-Yomdin reparametrization}.

		\end{enumerate}



\end{Claim}
\begin{proof}[Proof of Claim]

	We start an inductive construction to build $\Theta_{i\leq l_{0}}$.
	\paragraph{\it{\bfseries\small  Initial step:}\\}
	For $i=1$ ($a_{1}=0$), the curve $\sigma$ is $\varepsilon_{f}$-bounded by our assumption. We define $\Theta_{1}:=\{{\rm Id:[0,1]\to[0,1]}\}$. The properties in the claim are trivially satisfied.
	\paragraph{\it{\bfseries \small Inductive step:}\\}
	Assume that we have families $\{\Theta_{k}\}_{1\leq k\leq i}$ of affine maps satisfying the four properties in the claim. We consider three cases:

	\begin{enumerate}
		\item $a_{i+1}\in \Gamma^{1}_{i+1}$.

		Note that in this case, $$a_{i+1}=a_{i}+1=i\in[0,m).$$ As a consequence, given any $\theta_{i}\in\Theta_{i}$, after composing with a suitable reparametrization, we may assume \footnote{This follows directly from the definition of $\Omega^{p,m,L,s}_{n, E,\gamma}$, since we only consider points lying in the dynamical ball $B(z_{s}, m, \frac{\varepsilon_{f}}{100\left\|Df\right\|})$.} \begin{equation}\label{trapping case}
			f^{a_{i}}\circ\sigma\circ\theta_{i}\subset B\left(f^{a_{i}}\left(z_{s}\right), \frac{\varepsilon_{f}}{10\left\|Df\right\|}\right).
		\end{equation} We apply Proposition \ref{Burguet-Yomdin reparametrization} with $a=\gamma'_{a_{i}}, b=\gamma''_{a_{i}}$, we get a family $\Theta$ of affine maps from $[0,1]$ to itself with $$\log \#\Theta\leq \frac{\gamma'_{a_{i}}-\gamma''_{a_{i}}}{r-1}+A_{r} $$ such that  each $f^{a_{i+1}} \circ \sigma \circ \theta_{i} \circ \theta, \,\theta \in \Theta$ is bounded and $$|\theta'|\leq e^{-\frac{\gamma'_{a_{i}}-\gamma''_{a_{i}}+3r}{r-1}}.$$ By (\ref{trapping case}), $$f^{a_{i+1}} \circ \sigma \circ \theta_{i} \circ \theta\subset B\left(f^{a_{i+1}}\left(z_{l}\right), \frac{\varepsilon_{f}}{10}\right).$$ Hence we may assume, after composing $\theta$ with a suitable reparametrization \footnote{This can be shown directly from the definitions; see \cite[Lemma~13]{Bur24} and the references therein for a precise argument.},  that $f^{a_{i+1}} \circ \sigma \circ \theta_{i} \circ \theta$ is $\varepsilon_{f}$-bounded.  We then define $$\Theta_{i+1}:=\{\theta_{i}\circ\theta:~  \,\theta\in\Theta, \,\theta_{i}\in \Theta_{i}\}.$$ Note in this case,  $$\Gamma^{1}_{i+1}=\Gamma^{1}_{i}\cup\{a_{i+1}\},\quad \Gamma^{2}_{i+1}=\Gamma^{2}_{i},\quad \Gamma^{3}_{i+1}=\Gamma^{3}_{i}, \quad a_{i+1}-a_{i}=1,$$ and $$\#\left([m,a_{i}]\cap E\right)=\#\left([m,a_{i+1}]\cap E\right)=0.$$ Hence we have the desired bound on $\log \# \Theta_{i+1}$:
		$$\begin{aligned}
			\log \# \Theta_{i+1}
			&\leq \log \# \Theta_{i}+\# \log \Theta\\
			&\leq\#\left([m,a_{i+1}]\cap E\right)\cdot h_{\rm top}(f)+ \sum_{k<a_{i+1}}\frac{\gamma'_{k}-\gamma''_{k}}{r-1}+\#\Gamma^{2}_{i+1}\cdot 2B_{p}+ \left(a_{i+1}+1\right)\cdot A_{r}.
		\end{aligned}$$

		The remaining properties asserted in the claim are obvious.


	\item $a_{i+1}\in \Gamma^{2}_{i+1}$.

	In this case, $a_{i}\notin \Gamma^{2}_{i}$ and any time in the interval $(a_{i},a_{i+1}]$ is a ($L$-expanded) geometric time.

	Recall the definitions of $\varepsilon_{f^{p}}$, $B_p$ and $\hat{\Theta}$ introduced before the claim.

	For any $\theta_{i}\in \Theta_{i}$ and $\hat{\theta}\in\hat{\Theta}$, $f^{a_{i}}\circ\sigma\circ\theta_{i}\circ\hat{\theta}$ is an $\varepsilon_{f^{p}}$-bounded curve. Next, we decompose each $f^{a_i}\circ \sigma \circ \theta_i \circ \hat{\theta}$ into smaller pieces, with the decomposition taken up to time $a_{i+1}-a_{i}$. We choose some $(a_{i+1}-a_{i}, \varepsilon_{f^{p}})$ spanning set $\Omega\subset f^{a_{i}}\circ\sigma\circ\theta_{i}\circ\hat{\theta}$ with $$\log\#\Omega=\log r(f,a_{i+1}-a_{i}, \varepsilon_{f^{p}})\leq B_{p}+\left(a_{i+1}-a_{i}\right)\cdot\left(h_{\rm top}(f)+1\right).$$ Note that we are only concerned with points $z\in f^{a_{i}}\circ\sigma\circ\theta_{i}\circ\hat{\theta}$ satisfying $$\frac{1}{p}\left\lceil\log\left\|Df^{p}_{f^{j}(z)}\right\|\right\rceil= \gamma'_{a_{i}+j}, \quad\frac{1}{p}\left\lceil\log\left\|Df^{p}\left(\hat{f}^{i}\left(\hat{z}\right)\right)\right\|\right\rceil=\gamma''_{a_{i}+j},\quad j\in [0, a_{i+1}-a_{i}).$$ where $\hat{z}:=(z,[v])$, with $v$ denoting the unit tangent vector to the curve
	$f^{a_{i}}\circ\sigma\circ\theta_{i}\circ\hat{\theta}$ at $z$. Hence, without loss of generality, we many assume that all points in $\Omega$ satisfy these relations.

	Applying Proposition \ref{Corollary 2 Burguet-Yomdin reparametrization} to each point $z\in \Omega$ and $n=a_{i+1}-a_{i}$, we have a family $\Theta_{z}$ of affine maps from $[0,1]$ to itself with the following properties:
	\begin{enumerate}
		\item $B(z,a_{i+1}-a_{i},\varepsilon_{f^{p}})\cap f^{a_{i}}\circ\sigma\circ\theta_{i}\circ\hat{\theta}\subset \cup_{\theta\in \Theta_{z}} f^{a_{i}}\circ\sigma\circ\theta_{i}\circ\hat{\theta}\circ\theta$,
		\item $f^{k}\circ f^{a_{i}}\circ\sigma\circ\theta_{i}\circ\hat{\theta}\circ\theta$ is $\varepsilon_{f^{p}}$-bounded for any $\theta\in\Theta_{z}$ and $k\leq a_{i+1}-a_{i}$,

		\item We have $$\log\#\Theta_{z}\leq B_{p}+\left(a_{i+1}-a_{i}\right)\cdot A_{r}+\frac{1}{r-1}\cdot\sum_{k=a_{i}}^{a_{i+1}-1}\left(\gamma'_{k}-\gamma''_{k}\right).$$

	\end{enumerate}
	Note that  $$\Gamma^{1}_{i+1}=\Gamma^{1}_{i},\quad \Gamma^{2}_{i+1}=\Gamma^{2}_{i}\cup\{a_{i+1}\},\quad \Gamma^{3}_{i+1}=\Gamma^{3}_{i}.$$ Since
	the half-open interval $(a_i,a_{i+1}]\subset E$, we have $$\#\left([m,a_{i+1}]\cap E\right)=\#\left([m,a_{i}]\cap E\right)+a_{i+1}-a_{i}.$$

We then define $$\Theta_{i+1}:=\left\{\theta_{i}\circ\hat{\theta}\circ\theta:~ \theta_{i}\in \Theta_{i},\,\hat{\theta}\in\hat{\Theta},\, \theta\in\Theta_{z}, \, z\in\Omega\right\}$$ and we have
$$\begin{aligned}
	\log \# \Theta_{i+1}
	&\leq \log \# \Theta_{i}+ \log \#\hat{\Theta}+\max_{z\in\Omega}\log\#\Theta_{z}+\log\#\Omega\\
	&\leq\#\left([m,a_{i+1})\cap E\right)\cdot h_{\rm top}(f)+ \sum_{k<a_{i+1}}\frac{\gamma'_{k}-\gamma''_{k}}{r-1} + \#\Gamma^{2}_{i+1}\cdot 2B_{p}+ \left(a_{i+1}+1\right)\cdot A_{r}.
\end{aligned}$$

The remaining properties asserted in the claim are obvious. The term "$2B_{p}$" arises from this case.


		\item $a_{i+1}\in \Gamma^{3}_{i+1}$.

		In this case, $a_{i+1}-a_{i}=1$.

		Given any $\theta_{i}\in\Theta_{i}$, for the $\varepsilon_{f}$-bounded curve $f^{a_{i}}\circ\sigma\circ\theta_{i}$, we apply Proposition \ref{Burguet-Yomdin reparametrization} with $a=\gamma'_{a_{i}}, b=\gamma''_{a_{i}}$ and we get a family $\Theta$ of affine maps from $[0,1]$ to itself with $$\log \#\Theta\leq \frac{\gamma'_{a_{i}}-\gamma''_{a_{i}}}{r-1}+ A_{r}$$ such that  each $f^{a_{i+1}} \circ \sigma \circ \theta_{i} \circ \theta, \,\theta \in \Theta$ is bounded and $$|\theta'|\leq e^{-\frac{\gamma'_{a_{i}}-\gamma''_{a_{i}}+3r}{r-1}}.$$  Next we show that $f^{a_{i+1}}\circ \sigma\circ\theta_{i}\circ\theta,\, \theta\in\Theta$ is actually $\varepsilon_{f}$-bounded. Define $$j=\max\left\{k:~  a_{k}\in  E, \,\,a_{k}<a_{i+1}\right\}.$$ If the set in the right side is empty, we set $j=1$ which gives $a_{j}=a_{1}=0$.	We also note that $(a_{j}, a_{i+1}]\cap E=\emptyset$ by the maximality of $j$.  Consequently, $$a_{i+1}-a_{j}=i+1-j.$$ For any $x\in \Omega^{p,m,l,s}_{n, E,\gamma}$, since $a_{i+1}$ is not a geometric time while $a_{j}$ is the largest geometric time before $a_{i+1}$ \footnote{$a_{j}$ is the right endpoint of a geometric interval and is not necessarily larger than $m$.}, by Lemma \ref{property of geometric time}, $$\sum_{k=a_{j}}^{a_{i+1}-1}\rho'\left(\hat{f}^{k}(\hat{x})\right)=\sum_{k=a_{j}}^{a_{i}}\rho'\left(\hat{f}^{k}(\hat{x})\right)< a_{i+1}-a_{j}.$$ Take the ceiling operation into consideration, we then have
		\begin{equation}\label{ceiling operation}
		\sum_{k=j}^{i}\gamma''_{a_{k}}-\frac{\gamma'_{a_{k}}}{r}<2\left(a_{i+1}-a_{j}\right)=2\left(i+1-j\right).
		\end{equation}


		Let $t\in[-1,1]$ be such that $\sigma\circ\theta_{i}\circ\theta(t)=x\in \Omega^{p,m,l,s}_{n, E,\gamma}$ for some $x$. By the second assumption in the claim, there are some $\theta_{j}\in \Theta_{j}$ and some affine maps $\widetilde{\theta}_{j},\widetilde{\theta}_{j+1},\cdots, \widetilde{\theta}_{i-1}$ with  $$|\widetilde{\theta}'_{k}|\leq e^{-\frac{\gamma'_{a_{k}}-\gamma''_{a_{k}}+3r}{r-1}},\quad j\leq k\leq i-1$$ such that $$\theta_{i}=\theta_{j}\circ \widetilde{\theta}_{j}\circ\widetilde{\theta}_{j+1}\circ\cdots\circ\widetilde{\theta}_{i-1}.$$ We have
		$$\begin{aligned}
		\left\|d\left(f^{a_{i+1}}\circ \sigma\circ\theta_{i}\circ\theta\right)\right\|
			&\leq 2\left\|d_{t}\left(f^{a_{i+1}}\circ \sigma\circ\theta_{i}\circ\theta\right)\right\|\text{\quad by (\ref{Property of boundedness})}\\
			&\leq 2\left\|df^{a_{i+1}-a_{j}}\left(\hat{f}^{a_{j}}(\hat{x})\right)\right\|\cdot \left\|d_{t}\left(f^{a_{j}}\circ \sigma\circ\theta_{i}\circ\theta\right)\right\|\\
			&\leq 2\left\|df^{a_{i+1}-a_{j}}\left(\hat{f}^{a_{j}}(\hat{x})\right)\right\|\cdot \left\|d\left(f^{a_{j}}\circ \sigma\circ\theta_{i}\right)\right\|\cdot|\theta'|\\
			&\leq 2\left\|df^{a_{i+1}-a_{j}}\left(\hat{f}^{a_{j}}(\hat{x})\right)\right\|\cdot \left\|d\left(f^{a_{j}}\circ \sigma\circ\theta_{j}\right)\right\|\cdot\left(\prod_{k=j}^{i-1}|\widetilde{\theta}'_{k}|\right)\cdot|\theta'|\\
			&\leq 2\exp{\left(\sum_{k=j}^{i}\gamma''_{a_{k}}\right)}\cdot\varepsilon_{f}\cdot \exp{\left(\sum_{k=j}^{i}-\frac{\gamma'_{a_{k}}-\gamma''_{a_{k}}+3r}{r-1}\right)}\\
			&=2\varepsilon_{f}\cdot\exp{\left(\sum_{k=j}^{i}\gamma''_{a_{k}}-\frac{\gamma'_{a_{k}}-\gamma''_{a_{k}}+3r}{r-1}\right)}\\
			&=2\varepsilon_{f}\cdot\exp{\left(\sum_{k=j}^{i}\frac{r\cdot\left(\gamma''_{a_{k}}-\frac{\gamma'_{a_{k}}}{r}\right)-3r}{r-1}\right)}\\
			&=2\varepsilon_{f}\cdot\exp{\left(\frac{r}{r-1}\cdot\left(\sum_{k=j}^{i}\gamma''_{a_{k}}-\frac{\gamma'_{a_{k}}}{r}\right)-\frac{3r\left(i-j+1\right)}{r-1}\right)}\\
			&\leq2\varepsilon_{f}\cdot\exp{\left(-\frac{r\cdot\left(i-j+1\right)}{r-1}\right)}\text{\quad by (\ref{ceiling operation})}\\
			&\leq2\varepsilon_{f}\cdot e^{-1}\\
			&\leq\varepsilon_{f}.
		\end{aligned}$$ This shows that $f^{a_{i+1}}\circ \sigma\circ\theta_{i}\circ\theta$ is not only bounded, but $\varepsilon_{f}$-bounded. One can also see from the above estimates the technical role played by the term " $3r$ " in Proposition~\ref{Burguet-Yomdin reparametrization}.


		We define $\Theta_{i+1}:=\{\theta_{i}\circ\theta:~\theta_{i}\in \Theta_{i}, \theta\in \Theta\}$. Then we have  $$\Gamma^{1}_{i+1}=\Gamma^{1}_{i},\quad \Gamma^{2}_{i+1}=\Gamma^{2}_{i},\quad \Gamma^{3}_{i+1}=\Gamma^{3}_{i}\cup\{a_{i+1}\},\quad a_{i+1}-a_{i}=1.$$ Note $$\#\left([m,a_{i+1}]\cap E\right)=\#\left([m,a_{i}]\cap E\right).$$  Hence we have the desired bound on $\log \# \Theta_{i+1}$:
		$$\begin{aligned}
			\log \# \Theta_{i+1}
			&\leq \log \# \Theta_{i}+\# \log \Theta\\
			&\leq\#\left([m,a_{i+1}]\cap E\right)\cdot h_{\rm top}(f)+ \sum_{k<a_{i+1}}\frac{\gamma'_{k}-\gamma''_{k}}{r-1} + \#\Gamma^{2}_{i+1}\cdot 2B_{p}+ \left(a_{i+1}+1\right)\cdot A_{r}.
		\end{aligned}$$
		The remaining properties asserted in the claim follow directly from Proposition \ref{Burguet-Yomdin reparametrization}.
	\end{enumerate}
The proof of the claim is complete.
\end{proof}

Define
$$\begin{aligned}
	O^{p,m,L,s}_{n, E,\beta}:=\Big\{x\in\sigma:~&  \left(E(\hat{x})\cap [0,n)\right)^{L}=E,\,\,x\in B(z_{s}, m, \frac{\varepsilon_{f}}{100\left\|Df\right\|}) \text{ and }\\ &  \frac{1}{p}\left\lceil\log\left\|Df^{p}_{f^{i}(x)}\right\|\right\rceil= \beta'_{i}, \,\frac{1}{p}\left\lceil\log\left\|Df^{p}\left(\hat{f}^{i}\left(\hat{x}\right)\right)\right\|\right\rceil=\beta''_{i},\, i\in [m,n)\cap E,\\ & \left\lceil\log\left\|Df_{f^{i}(x)}\right\|\right\rceil=\beta'_{i}, \,\left\lceil\log\left\|Df\left(\hat{f}^{i}\left(\hat{x}\right)\right)\right\|\right\rceil=\beta''_{i},\, i\in [0,n)\setminus\left([m,n)\cap E\right)\Big \}.
\end{aligned}$$
We note that by definition,
\begin{equation}\label{comparison between O and Omega}
	O^{p,m,L,s}_{n, E,\beta}\subset\bigcup_{\gamma}	\Omega^{p,m,L,s}_{n, E,\gamma}
\end{equation} where the union is taken over all $\gamma:=\left\{\left(\gamma'_{i},\, \gamma''_{i}\right)\right\}_{i\in [0,n)}$ such that $$\gamma'_{i}=\beta'_{i},\quad \gamma''_{i}=\beta''_{i},\quad i\in [0,n)\setminus\left(\Delta\cup\Gamma_{l_{0}}^{2}\right).$$ There are at most $$\left(\log\left\|Df\right\|+\log\left\|Df^{-1}\right\|+2\right)^{\frac{2n}{L}}$$ such $\gamma$, since both $\Delta$ and $\Gamma_{l_0}^2$ have cardinality at most $n/L$. Indeed, the set $\Gamma_{l_0}^2$ records the right endpoints of geometric intervals, and $\Delta$ has the same cardinality as $\Gamma_{l_0}^2$. The number of geometric intervals (including isolated ones) is relatively small—at most $n/L$—because the complement of $E$ in $[0,n)$ is a disjoint union of intervals whose lengths are all greater than $L$.

As a consequence of (\ref{comparison between O and Omega}) and the claim above, there is a family $\widetilde{\Theta}_{l_{0}}$ of affine maps from $[0,1]$ to itself such that


\begin{enumerate}
	\item  $O^{p,m,L,s}_{n, E,\beta}\subset \cup_{\theta\in \widetilde{\Theta}_{l_{0}}} \sigma\circ\theta$,
	\item for any $\theta\in\widetilde{\Theta}_{l_{0}}$, $f^{n_{E}}\circ\sigma\circ\theta$ is $\varepsilon_{f}$-bounded,
	\item we have
	\begin{equation}\label{log widetilde{Theta}{l_{0}}}
		\begin{aligned}
		\log\# \widetilde{\Theta}_{l_{0}}
		&\leq\#\left([m,n)\cap E\right)\cdot h_{\rm top}(f)+ \sum_{k<a_{l_{0}}}\frac{\beta'_{k}-\beta''_{k}}{r-1}+ \frac{n}{L}\cdot 2B_{p}+ n\cdot \left(A_{r}+1\right)\\
		&\quad + \frac{2n}{L}\cdot\log \left(\log\left\|Df\right\|+\log\left\|Df^{-1}\right\|+2\right)+\frac{2n}{L}\cdot \frac{\log\left\|Df\right\|+\log\left\|Df^{-1}\right\|+2}{r-1}.
	\end{aligned}
\end{equation}	Note that the last term above $$\frac{2n}{L}\cdot \frac{\log\left\|Df\right\|+\log\left\|Df^{-1}\right\|+2}{r-1}$$ is introduced to to compensate for the discrepancy between (at the times $\Delta\cup\Gamma_{l_{0}}^{2}$) $$\sum_{k<a_{l_{0}}}\frac{\gamma'_{k}-\gamma''_{k}}{r-1} \quad\text{ and} \quad\sum_{k<a_{l_{0}}}\frac{\beta'_{k}-\beta''_{k}}{r-1}.$$

\end{enumerate}




\noindent\textbf{\large Step 2: Estimates Along the Time Decomposition}

Next, we estimate from above the quantity: $$\sum_{k<a_{l_{0}}}\frac{\beta'_{k}-\beta''_{k}}{r-1}= \sum_{k\in[0, n_{E})}\frac{\beta'_{k}-\beta''_{k}}{r-1}.$$
\begin{itemize}

	\item {\bf \small Geometric Bound:}  The estimation of the geometric component $\sum_{k\in [m,n)\cap E}\frac{\beta'_{k}-\beta''_{k}}{r-1}$ is rather delicate. We will show in the claim of Proposition \ref{volume growth for time n} that this term is negligible, using a measure-theoretic argument. We retain this term in  Proposition~\ref{grow rate for major component in finite step n}.

	\item {\bf \small Neutral bound:}

	Recall that $n_{E}$ is the maximal geometric time in $[0,n)$ (if no such geometric time exists, we set $n_{E}=0$). We consider the times in $[0,n_{E})\setminus E$ \footnote{This part contains the neutral component of the trapping times $[0,m)$ and is treated here for technical convenience. The estimation of the remaining, geometric  component will be carried out in {\bf Trapping Bound} below.}. We decompose $[0,n_{E})\setminus E$  into maximal disjoint intervals: $$[0,n_{E})\setminus E=\bigcup[c_{i}, d_{i}).$$ We note that $d_{i}\in E$ by maximality and consequently, for any $x\in O^{p,m,l,s}_{n, E,\beta}$, $$0<d_{i}-c_{i}\leq \sum_{k\in [c_{i},d_{i})}\rho'\left(\hat{f}^{k}(\hat{x})\right)\leq 10(d_{i}-c_{i})+\sum_{k\in [c_{i},d_{i})}\beta''_{k}-\frac{\beta'_{k}}{r}.$$ Here the term "$10(d_{i}-c_{i})$" is introduced to compensate for the discrepancy between the ceiling operation and the actual value. The above inequalities imply $$\sum_{k\in [c_{i},d_{i})}\beta''_{k}\geq -10(d_{i}-c_{i})+\sum_{k\in [c_{i},d_{i})}\frac{\beta'_{k}}{r}.$$ Adding up all the intervals, we get $$\sum_{k\in [0,n_{E})\setminus E}\beta''_{k}\geq -10\# \left([0,n_{E})\setminus E\right)+\sum_{k\in [0,n_{E})\setminus E}\frac{\beta'_{k}}{r}.$$ Hence, inserting the above lower bound on $\beta''_{k}$ into the inequality below,

	 $$\begin{aligned}
	\sum_{k\in [0,n_{E})\setminus E}\frac{\beta'_{k}-\beta''_{k}}{r-1}
		&\leq \frac{10\# \left([0,n_{E})\setminus E\right)}{r-1}+\sum_{k\in [0,n_{E})\setminus E}\frac{\beta'_{k}}{r}\\
		&\leq \frac{10\# \left([0,n_{E})\setminus E\right)}{r-1}+\# \left([0,n_{E})\setminus E\right)\cdot\frac{\log\left\|Df\right\|}{r}\\
		&=\left(n_{E}-\#E\right)\cdot\left(\frac{10}{r-1}+\frac{\log\left\|Df\right\|}{r}\right).
	\end{aligned}$$

\item {\bf \small Trapping Bound:} The estimate of $\frac{\beta'_{k}-\beta''_{k}}{r-1}$ for all $k\in [0,m)\setminus E$ has already been addressed in the {\bf Neutral bound} above. It remains to treat the case where $k\in [0,m)\cap E$. We decompose $[0,m)\cap E$ into maximal disjoint intervals: $$[0,m)\cap E=\left(\bigcup [c_{i}, d_{i}]\right)\bigcup [m_{E}, m-1]$$ where $[m_{E}, m-1]$ denotes the last interval if $m-1, m\in E$ (which means that the geometric interval is truncated at $m-1$).
For each interval $[c_{i}, d_{i})$, by an almost identical argument as in the neutral bound above, we have $$\sum_{k\in [c_{i}, d_{i})}\beta''_{k}\geq -10(d_{i}-c_{i})+\sum_{k\in [c_{i}, d_{i})}\frac{\beta'_{k}}{r}.$$
Adding up all the intervals, we get $$\sum_{i}\sum_{k\in [c_{i}, d_{i})}\beta''_{k}\geq -10m + \sum_{i}\sum_{k\in [c_{i}, d_{i})}\frac{\beta'_{k}}{r}.$$

For the last interval $[m_{E}, m-1]$, there is some geometric time $j\in (m-1-L,m-1]$. As a consequence, \begin{equation}\label{tail bound for m}
	\sum_{k\in [m_{E}, m-1]}\beta''_{k}\geq -10m+\sum_{k\in [m_{E}, m-1]}\frac{\beta'_{k}}{r}-L\cdot \left(\log\left\|Df\right\|+\log\left\|Df^{-1}\right\|+2\right).
\end{equation}


Hence, inserting the above lower bound on $\beta''_{k}$ into the inequality below,
	 $$\begin{aligned}
	&\quad\sum_{k\in [0,m)\cap E}\frac{\beta'_{k}-\beta''_{k}}{r-1}\\
	&= \sum_{i}\sum_{k\in [c_{i}, d_{i})}\frac{\beta'_{k}-\beta''_{k}}{r-1}+\sum_{d_{i}}\frac{\beta'_{d_{i}}-\beta''_{d_{i}}}{r-1}+\sum_{k\in [m_{E}, m-1]}\frac{\beta'_{k}-\beta''_{k}}{r-1}\\
	&\leq \frac{20m}{r-1}+\sum_{i}\sum_{k\in [c_{i}, d_{i})}\frac{\beta'_{k}}{r}+\sum_{d_{i}}\frac{\beta'_{d_{i}}-\beta''_{d_{i}}}{r-1}+\sum_{k\in [m_{E}, m-1]}\frac{\beta'_{k}}{r}+L\cdot\frac{ \log\left\|Df\right\|+\log\left\|Df^{-1}\right\|+2}{r-1}\\
	&\leq \#\left([0,m)\cap E\right)\cdot\frac{\log\left\|Df\right\|}{r}\\
	&\quad+\frac{20m}{r-1}+\frac{m}{L}\cdot\frac{\log\left\|Df\right\|+\log\left\|Df^{-1}\right\|+2}{r-1}+L\cdot\frac{ \log\left\|Df\right\|+\log\left\|Df^{-1}\right\|+2}{r-1}.
\end{aligned}$$
Note that, in the last step, we use the fact that there are relatively few (at most $m/L$) right endpoints $d_{i}$ among the maximal intervals in $[0,m)\cap E$, since the complement of $E$ in $[0,m)$ is a disjoint union of intervals whose lengths exceed $L$ (except possibly for the first and the last ones).
\item {\bf \small Tail Bound:}

For the times in $[n_{E},n)$, if $n_{E}<n-1$ which implies that $n-1$ is not a geometric time,  for any $x\in O^{p,m,l,s}_{n, E,\beta}$, by Lemma \ref{property of geometric time}, $$\sum_{k\in [n_{E},n-1)}\rho'\left(\hat{f}^{k}(\hat{x})\right)< n-n_{E}.$$ 

Taking the ceiling operation into consideration, we then have
\begin{equation}\label{direct assumption}
	\sum_{k\in [n_{E},n)}\beta''_{k}-\frac{\beta'_{k}}{r}\leq 2n +10\left(\log\left\|Df\right\|+\log\left\|Df^{-1}\right\|\right).
\end{equation} Note that the extra term $10\left(\log\left\|Df\right\|+\log\left\|Df^{-1}\right\|\right)$ accounts for the inclusion of the time $n-1$ in the sum and for the mismatch between $\beta''_{n_{E}}-\frac{\beta'_{n_{E}}}{r}$ and $\rho'\!\left(\hat{f}^{n_{E}}(\hat{x})\right)$.


For the tail times, instead of estimating $\sum_{k\in [n_{E},n)}\frac{\beta'_{k}-\beta''_{k}}{r-1}$, we directly consider the volume growth. Given any reparametrization $\theta\in \widetilde{\Theta}_{l_{0}}$, recalling the last paragraph in {\bf Step 1},  $f^{n_{E}}\circ\sigma\circ\theta$ is an $\varepsilon_{f}$-bounded curve and consequently, its volume (length) is less than 1. Define \footnote{We ignore the value at time $n_E$ in the definition of $\Omega_\theta$ for technical convenience, which results in a slightly larger set. Since this omission concerns only a single time instance, it does not affect the asymptotic regime.}
%$$\begin{aligned}
%	\Omega_{\theta}:=\Big\{t\in [0,1]:~ &  \frac{1}{p}\left\lceil\log\left\|Df^{p}_{f^{k}\circ\sigma\circ\theta(t)}\right\|\right\rceil= \beta'_{i}, \,\frac{1}{p}\left\lceil\log\left\|Df^{p}_{f^{k}\circ\sigma\circ\theta(t)}\left(\sigma'(\theta(t)\right)\right\|\right\rceil=\beta''_{k},\, i=n_{E},\\ & \left\lceil\log\left\|Df_{f^{k}\circ\sigma\circ\theta(t)}\right\|\right\rceil=\beta'_{k}, \left\lceil\log\left\|Df_{f^{k}\circ\sigma\circ\theta(t)}\left(\sigma'(\theta(t)\right)\right\|\right\rceil=\beta''_{k},\, i\in (n_{E}, n)\Big \}.
%\end{aligned}$$
$$\begin{aligned}
	\Omega_{\theta}:=\Big\{t\in [0,1]:~& \left\lceil\log\left\|Df_{f^{k}\circ\sigma\circ\theta(t)}\right\|\right\rceil=\beta'_{k},\\ &  \left\lceil\log\left\|Df_{f^{k}\circ\sigma\circ\theta(t)}\left(\sigma'(\theta(t)\right)\right\|\right\rceil=\beta''_{k},\quad k\in (n_{E},n)\Big \}
\end{aligned}$$ where $\sigma'(s)$ denotes the unit tangent vector at $\sigma(s)$. 
Note that our focus is solely on the restriction $\sigma\circ\theta|_{\Omega_{\theta}}$. By (\ref{direct assumption}), we have
 $$\begin{aligned}
	&\quad\log{\rm Vol}\left(f^{n}\circ\sigma\circ\theta|_{\Omega_{\theta}}\right)\\
	&\leq\log{\rm Vol}\left(f^{n_{E}}\circ\sigma\circ\theta\right)+\sum_{k\in [n_{E},n)}\beta''_{k}+10\left(\log\left\|Df\right\|+\log\left\|Df^{-1}\right\|\right)\\
	&\leq\sum_{k\in [n_{E},n)}\frac{\beta'_{k}}{r}+2n+20\left(\log\left\|Df\right\|+\log\left\|Df^{-1}\right\|\right)\\
	&\leq \left(n-n_{E}\right)\cdot\frac{\log\left\|Df\right\|}{r}+2n+20\left(\log\left\|Df\right\|+\log\left\|Df^{-1}\right\|\right).
\end{aligned}$$ Note that the term "$10\left(\log\left\|Df\right\|+\log\left\|Df^{-1}\right\|\right)$" in the first inequality is introduced to compensate for the mismatch at time $n_{E}$.

\end{itemize}

	\medskip
	\noindent\textbf{\large Step 3: Conclusion.}
	\medskip


By the estimates in {\bf Step 2},
 $$\begin{aligned}
	\sum_{k\in [0,n_{E})}\frac{\beta'_{k}-\beta''_{k}}{r-1}
	&=\sum_{k\in [m,n_{E})\cap E}\frac{\beta'_{k}-\beta''_{k}}{r-1}+\sum_{k\in [0,n_{E})\setminus E}\frac{\beta'_{k}-\beta''_{k}}{r-1} + \sum_{k\in [0,m)\cap E}\frac{\beta'_{k}-\beta''_{k}}{r-1}\\
	&\leq\sum_{k\in [m,n_{E})\cap E}\frac{\beta'_{k}-\beta''_{k}}{r-1}+\left(m + \#\left([m,n_{E})\setminus E\right)\right)\cdot\frac{\log\left\|Df\right\|}{r}\\
	&\quad + \frac{30n}{r-1} +\frac{n}{L}\cdot\frac{\log\left\|Df\right\|+\log\left\|Df^{-1}\right\|+2}{r-1}+L\cdot\frac{ \log\left\|Df\right\|+\log\left\|Df^{-1}\right\|+2}{r-1}.
\end{aligned}$$

Since $\varepsilon_{f}$-bounded curve has volume (length) less than 1, by the {\bf Tail Bound} in {\bf Step 2}, $$\log{\rm Vol}\left(f^{n}\left(O^{p,m,l,s}_{n, E,\beta}\right)\right)\leq \log\# \widetilde{\Theta}_{l_{0}}+ \left(n-n_{E}\right)\cdot\frac{\log\left\|Df\right\|}{r}+2n+20\left(\log\left\|Df\right\|+\log\left\|Df^{-1}\right\|\right).$$

By the definition of $n_{E}$,  $$ m + \#\left([m,n_{E})\setminus E\right)+n-n_{E}= m + \#\left([m,n)\setminus E\right).$$
Hence, by (\ref{log widetilde{Theta}{l_{0}}}) we have
		 $$\begin{aligned}
	\log{\rm Vol}\left(f^{n}\left(O^{p,m,l}_{n, E,\beta}\right)\right)
	&\leq \log n + \max_{1\leq s\leq n}\log{\rm Vol}\left(f^{n}\left(O^{p,m,l,s}_{n, E,\beta}\right)\right)\\
	&\leq \log n + \log\# \Theta_{l_{0}}+ \left(n-n_{E}\right)\cdot\frac{\log\left\|Df\right\|}{r}+2n+20\left(\log\left\|Df\right\|+\log\left\|Df^{-1}\right\|\right)\\
	&\leq \#\left([m,n)\cap E\right)\cdot h_{\rm top}(f)+\left(m + \#\left([m,n)\setminus E\right)\right)\cdot\frac{\log\left\|Df\right\|}{r}+\sum_{k\in [m,n)\cap E}\frac{\beta'_{k}-\beta''_{k}}{r-1}\\
	&\quad + \frac{n}{L}\cdot 2B_{p} + n\cdot \left(A_{r}+4\right)+ \frac{30n}{r-1}\\
	&\quad+\left(\frac{3n}{L}\cdot\frac{1}{r-1}+\frac{L}{r-1}+\frac{2n}{L}+20\right)\cdot\left(\log\left\|Df\right\|+\log\left\|Df^{-1}\right\|+2\right).
\end{aligned}$$

Define $$C_{n,L}:=\frac{n}{L}\cdot 2B_{p} + n\cdot \left(A_{r}+4\right)+ \frac{30n}{r-1} +\left(\frac{3n}{L}\cdot\frac{1}{r-1}+\frac{L}{r-1}+\frac{2n}{L}+20\right)\cdot\left(\log\left\|Df\right\|+\log\left\|Df^{-1}\right\|+2\right)$$ and $$ C_{r}:=A_{r}+4+\frac{30}{r-1}$$ and note that $$\limsup_{L \to +\infty}\limsup_{n \to +\infty}\frac{C_{n,L}}{n}=C_{r}.$$

The proof of Proposition~\ref{grow rate for major component in finite step n} is then complete.

\end{proof}


\subsection{Estimation on the geometric component and proof of Theorem \ref{main theorem of decomposition of volume growth of submanifolds}}

Recall that given a subset $E\subset \mathbb{N}$ and a number $L\in\mathbb{N}$, we defined
$$E^{L}:=\bigcup [i,j]$$ where the union runs over all $i,j\in E$ with $0\leq j-i\leq L$.

Given a subset $E\subset[0,n)$ (assume that $n$ is given in advance), we write $$\delta_{E}\left(\hat{x}\right):=\frac{1}{n}\sum_{i\in E}\delta_{\hat{f}^{i}(\hat{x})}$$ and we let $\delta_{E}\left(x\right)$ denote the corresponding projection, i.e., $$\delta_{E}\left(x\right):=\frac{1}{n}\sum_{i\in E}\delta_{f^{i}(x)}.$$

The following proposition shows that limits of empirical measures along geometric times are $\hat f$-invariant and project to invariant measures with at least one positive Lyapunov exponent.

\begin{Proposition}\label{limit measures are hyperbolic}
		Let $f$ be a $C^1$ diffeomorphism on a compact surface $M$. We consider a doubly indexed family of points $\hat{x}_{n,p}\in \hat{M}$ together with a family of geometric time sets $E_{n}\subset [0,n)$ (common to all indices $p$). Let $0\leq m_{n}\leq n$ be a sequence of integers and assume $$\delta_{[m_{n},n)\cap E_{n}^{L}}\left(\hat{x}_{n,p}\right)\xrightarrow{\text{$\quad n \quad$}}\hat{\mu}^{L}_{p},\quad\quad \hat{\mu}^{L}_{p}\xrightarrow{\text{$\quad L \quad$}}\hat{\mu}_{p},\quad\quad \hat{\mu}_{p}\xrightarrow{\text{$\quad p \quad$}}\hat{\mu}$$ for some Borel measures (not necessarily probability measures) $\hat{\mu}^{L}_{p}, \hat{\mu}_{p}$ and $\hat{\mu}$. Then $\hat{\mu}_{p}$ and $\hat{\mu}$ are $\hat{f}$-invariant. Moreover, the projection $\mu$ of $\hat{\mu}$ carries at least one positive Lyapunov exponent (assuming $\mu$ is not zero measure) and $\hat{\mu}$ is an unstable lift of $\mu$.
\end{Proposition}
\begin{proof}
Given $L\in\mathbb{N}$, we decompose $[m_{n},n)\cap E_{n}^{L}$ into maximal disjoint intervals: $$[m_{n},n)\cap E_{n}^{L}=\bigcup_{i\geq 1}[a_{i}, b_{i}).$$ Note that there are relatively few (at most $n/L$) such intervals $[a_{i}, b_{i})$, as the complement of $E^{L}_{n}$ in $[0, n)$ consists of a disjoint union of intervals with lengths exceeding $L$, except possibly the first and the last intervals. Consequently, given any continuous function $h:\hat{M}\to\mathbb{R}$ and any $n,p\in\mathbb{N}$, $$\big|\delta_{[m_{n},n)\cap E_{n}^{L}}\left(\hat{x}_{n,p}\right)\left(h\right)-\delta_{[m_{n},n)\cap E_{n}^{L}}\left(\hat{x}_{n,p}\right)\left(h\circ\hat{f}\right)\big|\leq\frac{1}{n}\sum_{i}\big|h\left(\hat{f}^{a_{i}}\left(\hat{x}\right)\right)\big|+\big|h\left(\hat{f}^{b_{i}}\left(\hat{x}\right)\right)\big|\leq \frac{2}{L}\cdot\max_{\hat{y}}|h(\hat{y})|.$$ Passing to the limit in $n$, we have $$\big|\hat{\mu}^{L}_{p}\left(h\right)-\hat{\mu}^{L}_{p}\left(h\circ\hat{f}\right)\big|\leq \frac{2}{L}\cdot\max_{\hat{y}}|h(\hat{y})|.$$ Then passing to the limit in $L$ gives the $\hat{f}$-invariance of $\hat{\mu}_{p}$ and the $f$-invariant of $\mu_{p}$. As a consequence, $\hat{\mu}$ and $\mu$ are also invariant.

Next we show that $\mu$ has at least one positive Lyapunov exponent. We first note the following two facts:
\begin{itemize}
	\item Since $E_{n}^{L}\subset E_{n}^{L+1}$, the above convergence w.r.t. $L$ (i.e., $\hat{\mu}^{L}_{p}\xrightarrow{\text{$\quad L \quad$}}\hat{\mu}_{p}$) is an increasing convergence where \emph{increasing} means for any subset $A$, $$\hat{\mu}^{L}_{p}(A)\leq \hat{\mu}^{L+1}_{p}(A).$$
	\item By the convergences of the measures, we have $$\frac{\#\left([m_{n},n)\cap E_{n}^{L}\right)}{n} \xrightarrow{\text{$\quad n \quad$}} \alpha^{L}\quad \text{ and } \quad \alpha^{L}\xrightarrow{\text{$\quad L \quad$}} \alpha$$ where $\alpha^{L},\alpha\in[0,1]$ are defined (independent of $p$) by $$\alpha^{L}:= \hat{\mu}^{L}_{p}(\hat{M})=\mu^{L}_{p}(M),\quad\alpha:=\hat{\mu}_{p}(\hat{M})=\hat{\mu}(\hat{M})=\mu_{p}(M)=\mu(M).$$ Since we assume that $\mu$ is not zero measure, we may assume $\alpha^{L}, \alpha>0$. Also note that $\alpha^{L}$ converges increasingly to $\alpha$.
\end{itemize}


Note that by definition, for any point $\hat{x}_{n,p}$ and any integer $i$ with $i, i+1\in E_{n}^{L}$ \footnote{This additional condition $i+1\in E_n^{L}$ is a purely technical adjustment. It compensates for a minor asymmetry in the definition of geometric times: although $n$ itself may be declared a geometric time, the defining inequality only involves sums up to time $n-1$ (see Definition~\ref{definition of geometric time}). Requiring $i+1\in E_n^{L}$ avoids boundary issues caused by this convention and does not affect any asymptotic estimates.}, there is $1\leq l\leq L$ such that $l+i$ is a geometric time of $\hat{x}$. We then have (recalling $\rho(x, [v]):=\log\left\|Df_{x}\left(v\right)\right\|$) $$\hat{f}^{i}(\hat{x}_{n,p})\in \Omega^{L}:=\left\{\hat{x}\in\hat{M}:~ \exists 1\leq l\leq L \text{ such that } \sum_{m=0}^{l-1}\rho\left(\hat{f}^{m}\left(\hat{x}\right)\right)\geq l\right\}.$$ Note that in the definition of $\Omega^{L}$ above, we use the larger function $\rho$ rather than $\rho'$ for simplicity (since $\rho\geq\rho'$). Note that the number of indices $i$ satisfying $i\in E_n^{L}$ and $i+1\notin E_n^{L}$ is at most $\frac{n}{L}$, since each such $i$ corresponds to the right endpoint of a geometric interval and the complement of $E_n^{L}$ consists of intervals of length at least $L$. As a consequence, for any $n,p\in\mathbb{N}$, we have $$\big|\delta_{[m_{n},n)\cap E_{n}^{L}}\left(\hat{x}_{n,p}\right)\left(\Omega^{L}\right)-\delta_{[m_{n},n)\cap E_{n}^{L}}\left(\hat{x}_{n,p}\right)\left(\hat{M}\right)\big |\leq \frac{1}{L}.$$ By the compactness of $\Omega^{L}$, passing the limit in $n$, for any $p\in\mathbb{N}$,  $$\big|\hat{\mu}_{p}^{L}\left(\Omega^{L}\right)-\hat{\mu}^{L}_{p}\left(\hat{M}\right)\big|=\big|\hat{\mu}_{p}^{L}\left(\Omega^{L}\right)-\alpha^{L}\big|\leq \frac{1}{L}.$$ Given $\varepsilon>0$, choose $L_{0}\in\mathbb{N}$ large enough such that for any $p\in\mathbb{N}$, $$\hat{\mu}_{p}^{L_{0}}\left(\Omega^{L_{0}}\right)\geq \alpha-\varepsilon.$$ Recalling that $\hat{\mu}^{L}_{p}\xrightarrow{\text{$\quad L \quad$}}\hat{\mu}_{p}$ is an increasing convergence, for any $L\geq L_{0}$ and any $p\in\mathbb{N}$, we have $$\hat{\mu}_{p}^{L}\left(\Omega^{L_{0}}\right)\geq \hat{\mu}_{p}^{L_{0}}\left(\Omega^{L_{0}}\right)\geq \alpha-\varepsilon.$$ By the compactness of $\Omega^{L_{0}}$, passing the limit in $L$, then in $p$, we have $$\hat{\mu}\left(\Omega^{L_{0}}\right)\geq \alpha-\varepsilon.$$ By the arbitrariness of $\varepsilon$, we have  $$\hat{\mu}\left(\bigcup_{L}\Omega^{L}\right)=\hat{\mu}\left(\hat{M}\right)=\alpha.$$ Therefore the $\hat{f}$-invariant set $\bigcap_{i\in\mathbb{Z}}\hat{f}^{i}\left(\bigcup_{L}\Omega^{L}\right)$ has $\hat{\mu}$ full measure in which for any $\hat{x}$, by definition, there is an increasing sequence $0=m_{0}<m_{1}<m_{2}<\cdots$ such that $$\sum_{m=m_{i}}^{m_{i+1}-1}\rho\left(\hat{f}^{m}\left(\hat{x}\right)\right)\geq m_{i+1}-m_{i}.$$ Hence for $\hat{\mu}$-almost every $\hat{x}$, $$\limsup_{n\to+\infty}\frac{1}{n}\sum_{i=0}^{n-1}\rho\left(\hat{f}\left(\hat{x}\right)\right)\geq 1>0.$$ This implies the  projection $\mu$ of $\hat{\mu}$ has at least one positive Lyapunov exponent and $\hat{\mu}$ is an unstable lift of $\mu$.
\end{proof}


The following proposition provides an upper bound for the volume growth of a sequence of curves, already exhibiting the convex structure that interpolates between topological entropy and the Yomdin term. It can be viewed as a preliminary form of the convex upper bound in Theorem~\ref{main theorem of decomposition of volume growth of submanifolds}, where the parameter $\alpha$ arises from the relative proportions of different dynamical time regimes.

Recall the small constant $\varepsilon_{f}$ in Proposition \ref{Burguet-Yomdin reparametrization}.

\begin{Proposition}\label{volume growth for time n}
	Let $f$ be a $C^r$ ($r>1$) diffeomorphism on a compact surface $M$ and let $\sigma_{n}$ be a sequence of $\varepsilon_{f}$-bounded $C^r$ curves. There is a number $\alpha\in [0,1]$such that $$\limsup_{n \to +\infty}\frac{1}{n}\log{\rm Vol}\left(f^{n}\left(\sigma_{n}\right)\right)\leq \alpha\cdot h_{\rm top}(f)+\left(1-\alpha\right)\cdot\frac{\log\left\|Df\right\|}{r}+4\log\left(\log\left\|Df\right\|+\log\left\|Df^{-1}\right\|+2\right)+C_{r}$$ where  $C_{r}$ is a universal constant depending only on $r$.
\end{Proposition}
\begin{proof}



	Recall that the periodic sources are denoted by $z_{1}, z_{2}, z_{3}, \cdots$.

	Given $n\in\mathbb{N}$ and a point $x\in\sigma_{n}$, we define the \emph{trapping time} of $x$ by $$t_{n}(x):=\max\left\{0< k\leq n:~ \exists 1\leq s\leq n, \text{ s.t. } x\in B\left(z_{s}, k,\frac{\varepsilon_{f}}{100\left\|Df\right\|}\right)\right\}$$ and set $t_{n}(x):=0$ if for any $1\leq s\leq n$, $$x\notin B\left(z_{s}, \frac{\varepsilon_{f}}{100\left\|Df\right\|}\right).$$ Given $0\leq m\leq n$, define $$\sigma^{m}_{n}:=\{x\in\sigma_{n}:~ t_{n}(x)=m\}$$ and choose any $m_{n}$ (may not be unique) such that $${\rm Vol}\left(f^{n}\left(\sigma^{m_{n}}_{n}\right)\right)=\max_{0\leq m\leq n}{\rm Vol}\left(f^{n}\left(\sigma^{m}_{n}\right)\right).$$

	We define two partitions:
	\begin{itemize}
		\item Given $n\in\mathbb{N}$ and $E_{n}\subset[0,n)$, define $$\Lambda_{E_{n}}:=\left\{x\in\sigma^{m_{n}}_{n}:~E(\hat{x})\cap [0,n)=E_{n}\right\}$$ where, as previously defined, $E(\hat{x})$ is set of geometric times of $\hat{x}$. Let $\mathcal{E}_{n}$ be the partition of $\sigma^{m_{n}}_{n}$ with such elements $\Lambda_{E_{n}}$ for all possible $E_{n}$. Note that for the cardinality, we have $$\#\mathcal{E}_{n}\leq 2^{n}.$$ We fix an element $\Lambda_{E_{n}}\in\mathcal{E}_{n}$ that achieves the maximal volume growth among all possible $\widetilde{E}_{n}$, i.e., $${\rm Vol}\left(f^{n}\left(\Lambda_{E_{n}}\right)\right)=\max_{\widetilde{E}_{n}}{\rm Vol}\left(f^{n}\left(\Lambda_{\widetilde{E}_{n}}\right)\right).$$
		\item  Given $p\in\mathbb{N}$ and a family of integers
		$$\gamma:=\left\{\left(\gamma'_{i},\, \hat{\gamma}'_{i},\, \gamma''_{i},\, \hat{\gamma}''_{i}\right)\right\}_{i\in [0,n)},$$ define
		$$\begin{aligned}
			\Lambda^{p,m_{n}}_{n, E_{n},\gamma}:=\Big\{x\in\Lambda_{E_{n}}:~\,i\in [0,n),\,\,\,&  \frac{1}{p}\left\lceil\log\left\|Df^{p}_{f^{i}(x)}\right\|\right\rceil= \gamma'_{i}, \,\frac{1}{p}\left\lceil\log\left\|Df^{p}\left(\hat{f}^{i}\left(\hat{x}\right)\right)\right\|\right\rceil=\hat{\gamma}'_{i},\,\\ & \left\lceil\log\left\|Df_{f^{i}(x)}\right\|\right\rceil=\gamma''_{i}, \,\left\lceil\log\left\|Df\left(\hat{f}^{i}\left(\hat{x}\right)\right)\right\|\right\rceil=\hat{\gamma}''_{i}\, \Big \}.
		\end{aligned}$$
		Let $\mathcal{E}_{n, E_{n}}^{p,m_{n}}$ be the partition of $\Lambda_{E_{n}}$ with such elements  $\Lambda^{p,m_{n}}_{n, E_{n},\gamma}$ for all possible $\gamma$.  We note two facts:
		\begin{itemize}
			\item For the cardinality, we have $$\#\mathcal{E}_{n, E_{n}}^{p,m_{n}}\leq \left(\log\left\|Df\right\|+\log\left\|Df^{-1}\right\|+2\right)^{4n}.$$
			\item For any $L\in \mathbb{N}$, by definitions, $$\Lambda^{p,m_{n}}_{n, E_{n},\gamma}\subset O^{p,m_{n}, L}_{n, E^{L}_{n},\gamma}$$ where the corresponding $\beta=\beta(\gamma):=\left\{\left(\beta'_{i},\, \beta''_{i}\right)\right\}_{i\in [0,n)}$ is defined by
			$$\beta'_{i}:=\gamma'_{i},\,\,\beta''_{i}:=\hat{\gamma}'_{i}, \,\, i\in[m_{n},n)\cap E_{n}^{L} \,\,\text{ and }\,\, \beta'_{i}:=\gamma''_{i},\,\,\beta''_{i}:=\hat{\gamma}''_{i}, \,\, i\in[0,n)\setminus\left([m_{n},n)\cap E_{n}^{L}\right).$$
		\end{itemize}


	\end{itemize}

	For each $p$, we fix an element $\Lambda^{p,m_{n}}_{n, E_{n},\gamma}\in\mathcal{E}_{n, E_{n}}^{p,m_{n}}$ that achieves the maximal volume growth, i.e., $${\rm Vol}\left(f^{n}\left(\Lambda^{p,m_{n}}_{n, E_{n},\gamma}\right)\right)\geq\max_{\widetilde{\gamma}}\,{\rm Vol}\left(f^{n}\left(\Lambda^{p,m_{n}}_{n, E_{n},\widetilde{\gamma}}\right)\right).$$

	We fix any point $x_{n,p}\in \Lambda^{p,m_{n}}_{n, E_{n},\gamma}$. Up to considering sub-sequences in $n, L,p$, we assume
	\begin{equation}\label{definitions for measures}
		\delta_{[m_{n},n)\cap E_{n}^{L}}\left(\hat{x}_{n,p}\right)\xrightarrow{\text{$\quad n \quad$}}\hat{\mu}^{L}_{p},\quad\quad \hat{\mu}^{L}_{p}\xrightarrow{\text{$\quad L \quad$}}\hat{\mu}_{p},\quad\quad \hat{\mu}_{p}\xrightarrow{\text{$\quad p \quad$}}\hat{\mu}
	\end{equation} for some Borel measures (not necessarily probability measures) $\hat{\mu}^{L}_{p}, \hat{\mu}_{p}$ and $\hat{\mu}$ (denote by $\mu^{L}_{p}, \mu_{p}$ and $\mu$ the corresponding projections on $M$). By the convergences of these measures, we have $$\frac{\#\left([m_{n},n)\cap E_{n}^{L}\right)}{n} \xrightarrow{\text{$\quad n \quad$}} \alpha^{L}\quad \text{ and } \quad \alpha^{L}\xrightarrow{\text{$\quad L \quad$}} \alpha$$ where $\alpha^{L},\alpha\in[0,1]$ are defined (independent of $p$) by $$\alpha^{L}:= \hat{\mu}^{L}_{p}(\hat{M})=\mu^{L}_{p}(M),\quad\alpha:=\hat{\mu}_{p}(\hat{M})=\hat{\mu}(\hat{M})=\mu_{p}(M)=\mu(M).$$
	By Proposition \ref{grow rate for major component in finite step n}, for any $p,L\in\mathbb{N}$, recalling that $\gamma$ is determined by $\Lambda_{E_n}$ and $p$, and that $\beta$ is determined by $\gamma$,
	\begin{equation}\label{f n sigma volume intermediate step}
		\begin{aligned}
		\log{\rm Vol}\left(f^{n}\left(\sigma_{n}\right)\right)
		&\leq \log \left(n+1\right)+ \log{\rm Vol}\left(f^{n}\left(\sigma^{m_{n}}_{n}\right)\right)\\
		&\leq \log \left(n+1\right)+ \log\#\mathcal{E}_{n}+ \log{\rm Vol}\left(f^{n}\left(\Lambda_{E_{n}}\right)\right)\\
		&\leq \log \left(n+1\right)+ \log\#\mathcal{E}_{n}+\log\#\mathcal{E}_{n,E_{n}}^{p,m_{n}}+\log{\rm Vol}\left(f^{n}\left(\Lambda^{p,m_{n}}_{n, E_{n},\gamma}\right)\right)\\
		&\leq \log \left(n+1\right)+ \log\#\mathcal{E}_{n}+\log\#\mathcal{E}_{n,E_{n}}^{p,m_{n}}+\log{\rm Vol}\left(f^{n}\left(O^{p,m_{n}, L}_{n, E^{L}_{n},\beta}\right)\right)\\
		&\leq \log \left(n+1\right) + n\log2+4n\log\left(\log\left\|Df\right\|+\log\left\|Df^{-1}\right\|+2\right)+\#\left([m_{n},n)\cap E^{L}_{n}\right)\cdot h_{\rm top}(f)\\
		&+\left(m_{n}+\#\left([m_{n},n)\setminus E^{L}_{n}\right)\right)\cdot\frac{\log\left\|Df\right\|}{r} + \sum_{k\in [m_{n},n)\cap E^{L}_{n}}\frac{\beta'_{k}-\beta''_{k}}{r-1}+ C_{n,L}
	\end{aligned}
	\end{equation} where the error term $C_{n,L}$ satisfies $$\limsup_{L \to +\infty}\limsup_{n \to +\infty}\frac{C_{n,L}}{n}=C_{r}$$ for some universal constant $C_{r}$ depending only on $r$.

	Note that by definition, recalling $\rho(x, [v]):=\log\left\|Df_{x}\left(v\right)\right\|$, $$\frac{1}{n}\sum_{k\in [m_{n},n)\cap E^{L}_{n}}\frac{\beta'_{k}-\beta''_{k}}{r-1}\leq \frac{1}{r-1}\int\frac{\log\left\|Df^{p}_{(\cdot)}\right\|-\sum_{i=0}^{p-1}\rho\circ\hat{f}^{i}}{p} d\,\left(\delta_{[m_{n},n)\cap E_{n}^{L}}\left(\hat{x}_{n,p}\right)\right)+10.$$ Here the term "$10$" is introduced to compensate for the average discrepancy between the ceiling operation and the actual value.

	Putting $\frac{1}{n}$ on both sides in (\ref{f n sigma volume intermediate step}), we have

\begin{equation}\label{frac 1 n f n sigma volume intermediate step}
		\frac{1}{n}\log{\rm Vol}\left(f^{n}\left(\sigma_{n}\right)\right)\leq \alpha_{n,L}\cdot h_{\rm top}(f)+\left(1-\alpha_{n,L}\right)\cdot\frac{\log\left\|Df\right\|}{r}+\gamma_{n,L,p}
\end{equation} where $$\alpha_{n,L}:=\frac{\#\left([m_{n},n)\cap E^{L}_{n}\right)}{n}$$ and
	$$\begin{aligned}
\gamma_{n,L,p}
	&:= \frac{\log \left(n+1\right)}{n} + \log2+4\log\left(\log\left\|Df\right\|+\log\left\|Df^{-1}\right\|+2\right)+\frac{C_{n,L}}{n}+10\\
	&\quad+\frac{1}{r-1}\int\frac{\log\left\|Df^{p}_{(\cdot)}\right\|-\sum_{i=0}^{p-1}\rho\circ\hat{f}^{i}}{p} d\,\left(\delta_{[m_{n},n)\cap E_{n}^{L}}\left(\hat{x}_{n,p}\right)\right).
	\end{aligned}$$


	We take $\limsup_{n \to +\infty}$, then take $\limsup_{L \to +\infty}$ on both sides in (\ref{frac 1 n f n sigma volume intermediate step}): $$\limsup_{n \to +\infty}\frac{1}{n}\log{\rm Vol}\left(f^{n}\left(\sigma_{n}\right)\right)\leq \alpha\cdot h_{\rm top}(f)+\left(1-\alpha\right)\cdot\frac{\log\left\|Df\right\|}{r}+\limsup_{L\to+\infty}\limsup_{n \to +\infty}\gamma_{n,L,p}$$ where $$\alpha:=\limsup_{L\to+\infty}\limsup_{n \to +\infty}\alpha_{n,L}.$$

	It remains to show the following claim.

	\begin{Claim} $\gamma_{n,L,p}$ satisfies $$\limsup_{p\to+\infty}\limsup_{L\to+\infty}\limsup_{n \to +\infty}\gamma_{n,L,p}\leq 4\log\left(\log\left\|Df\right\|+\log\left\|Df^{-1}\right\|+2\right)+C_{r}$$ for some universal constant $C_{r}$ \footnote{The constant $C_{r}$ was defined earlier. We reuse the same notation here for simplicity. This causes no ambiguity, since the present $C_{r}$ may differ from the previous one only by a constant depending only on $r$.} depending only on $r$.
	\end{Claim}
	\begin{proof}[Proof of Claim]
		Let us first show the following two properties:
		\begin{itemize}
			\item $\hat{\mu}_{p}$ and $\hat{\mu}$ defined in (\ref{definitions for measures}) are invariant measures,
			\item $\hat{\mu}$ is the unique unstable lift of $\mu$, provided that $\mu$ is not zero measure.
		\end{itemize}
		To see this, by Proposition \ref{limit measures are hyperbolic},
		\begin{itemize}
		\item $\hat{\mu}_{p}, \hat{\mu}$ are invariant measures,
		\item $\mu$ has at least one positive Lyapunov exponent,
		\item $\hat{\mu}$ is an unstable lift of $\mu$.
		\end{itemize} We claim that $\mu$ must have a unique positive Lyapunov exponent (hence $\hat{\mu}$ is the unique unstable lift of $\mu$), due to the choice of $m_{n}$.   Indeed, if not, $\mu$ would have an ergodic component with two positive Lyapunov exponents and this ergodic component must support on some periodic source $z$ with $\mu$ positive measure ( see \cite[Proposition 4.4]{Pol93} or \cite[Fact in Claim 7.3]{BCS22}). Let $r>0$ be small enough such that if $y\notin B\left(f^{i}(z), \frac{\varepsilon_{f}}{100\left\|Df\right\|}\right)$ for some $i\in\mathbb{N}$,  then $f^{k}(y)\notin B(z, r)$ for all $k\in\mathbb{N}$ \footnote{This is possible since $z$ is a periodic source. Let $T_{z}$ denote the period of $z$.  We can find a small number $0<\delta<\frac{\varepsilon_{f}}{100\left\|Df\right\|}$ such that for $0\leq k \leq T_{z}-1$,
		$$f^{-T_{z}}\left(B\left(f^{k}(z),\delta\right)\right)\subset B\left(f^{k}(z),\delta\right).$$ Then choose $r>0$ small enough such that for $0\leq k \leq T_{z}-1$, $$f^{k}\left(B\left(z,r\right)\right)\subset B\left(f^{k}(z),\delta\right).$$}.   Since $\mu$ is not zero measure, we may assume for all large $n$, $$m_{n}<n.$$

		Since $x_{n,p}\in\sigma_{n}^{m_{n}}$, by the definition of $\sigma_{n}^{m_{n}}$, there must exist some $0\leq i< m_{n}+1$ such that  $$f^{i}(x_{n,p})\notin B\left(f^{i}(z), \frac{\varepsilon_{f}}{100\left\|Df\right\|}\right).$$  As a consequence, for any $k\in [m_{n},n)\cap E_{n}^{L}$, $$f^{k}(x_{n,p})\notin B(z, r).$$ This implies for large $n$, $$\delta_{[m_{n},n)\cap E_{n}^{L}}\left(x_{n,p}\right)\left(B(z,r)\right)=0$$ which is a contradiction. Hence $\mu$ must have a unique positive Lyapunov exponent and  $\hat{\mu}$ is the unique unstable lift of $\mu$.




		 By taking limit in $n$, then in $L$, we have

		\begin{equation*}
			\begin{aligned}
				&\quad\limsup_{L\to+\infty}\limsup_{n \to +\infty}\int\frac{\log\left\|Df^{p}_{(\cdot)}\right\|-\sum_{i=0}^{p-1}\rho\circ\hat{f}^{i}}{p} d\,\left(\delta_{[m_{n},n)\cap E_{n}^{L}}\left(\hat{x}_{n,p}\right)\right)\\
				&= \int\frac{\log\left\|Df^{p}_{(\cdot)}\right\|-\sum_{i=0}^{p-1}\rho\circ\hat{f}^{i}}{p} d\,\hat{\mu}_{p} .\\
				&=\int\frac{\log\left\|Df^{p}_{(\cdot)}\right\|}{p} d\,\mu_{p}- \int \rho d\,\hat{\mu}_{p}.
			\end{aligned}
		\end{equation*}
		Taking limit in $p$ on both sides, by Lemma \ref{upper continuity of subadditive functions}, we then have
			$$\begin{aligned}
				&\quad\limsup_{p\to+\infty}\limsup_{L\to+\infty}\limsup_{n \to +\infty}\int\frac{\log\left\|Df^{p}_{(\cdot)}\right\|-\sum_{i=0}^{p-1}\rho\circ\hat{f}^{i}}{p} d\,\left(\delta_{[m_{n},n)\cap E_{n}^{L}}\left(\hat{x}_{n,p}\right)\right)\\
			&\leq \int\limsup_{p\to+\infty}\frac{\log\left\|Df^{p}_{(\cdot)}\right\|}{p} d\,\mu- \int \rho d\,\hat{\mu}.
		\end{aligned}$$
		Since $\hat{\mu}$ is the unique unstable lift of $\mu$, we have (see Section \ref{tangent dynamics}) $$\int\limsup_{p\to+\infty}\frac{\log\left\|Df^{p}_{(\cdot)}\right\|}{p} d\,\mu- \int \rho d\,\hat{\mu}=0.$$ Hence $$\limsup_{p\to+\infty}\limsup_{L\to+\infty}\limsup_{n \to +\infty}\gamma_{n,L,p}\leq 4\log\left(\log\left\|Df\right\|+\log\left\|Df^{-1}\right\|+2\right)+\log 2+10+C_{r}.$$ We can put $\log 2+10$ into $C_{r}$ which completes the proof of the claim.


	\end{proof}
The proof of Proposition \ref{volume growth for time n} is then complete.
\end{proof}

Now we can use Proposition \ref{volume growth for time n} to prove Theorem \ref{main theorem of decomposition of volume growth of submanifolds}.
\begin{proof}[Proof of Theorem \ref{main theorem of decomposition of volume growth of submanifolds}]

Define
$$\Phi(\alpha):=\alpha\cdot h_{\rm top}(f)+(1-\alpha)\cdot\frac{\log\left\|Df\right\|}{r}+4\log\left(\log\left\|Df\right\|+\log\left\|Df^{-1}\right\|+2\right)+C_r,\qquad\alpha\in[0,1]$$ where $C^{r}$ is the universal constant depending only $r$ from Proposition \ref{volume growth for time n}.

		In the following arguments, the notation $\sup_{\sigma}$ refers to the supremum taken over all
$C^{r}$ embedded curves $\sigma:[0,1]\to M$ with
$\left\|\sigma\right\|_{C^{r}}\le 1$ and $\left\|d_{t}\sigma\right\| \geq 1/2, \,t\in[0,1]$.

We claim \begin{equation}\label{uniform volume growth}
	\limsup_{n\to+\infty}\frac{1}{n}\log\sup_{\sigma}{\rm Vol}\left(f^n(\sigma)\right)\leq\max_{\alpha\in[0,1]}\Phi(\alpha).
\end{equation} Assume by contradiction that
$$\limsup_{n\to+\infty}\frac{1}{n}\log\sup_{\sigma}{\rm Vol}\left(f^n(\sigma)\right)>\max_{\alpha\in[0,1]}\Phi(\alpha).$$ Then there exist $n_k\to+\infty$ and $C^r$ embedded curves
$\sigma_{n_k}$ with
$\left\|\sigma\right\|_{C^{r}}\le 1$ and $\left\|d_{t}\sigma\right\| \geq 1/2, \,t\in[0,1]$ such that $$\limsup_{k\to+\infty}\frac{1}{n_k}\log{\rm Vol}\left(f^{n_k}(\sigma_{n_k})\right)>\max_{\alpha\in[0,1]}\Phi(\alpha).$$
Since each $\sigma_{n_k}$ can be decomposed into a uniformly bounded number of $\varepsilon_f$-bounded curves (see Lemma~\ref{Decomposition into varepsilon bounded curves}), we may, without loss of generality, assume that the sequence $\{\sigma_{n_k}\}$ itself consists of $\varepsilon_f$-bounded curves.
 Hence, by Proposition \ref{volume growth for time n},
there exists $\alpha\in[0,1]$ such that $$\limsup_{k\to+\infty}\frac{1}{n_k}\log{\rm Vol}\left(f^{n_k}(\sigma_{n_k})\right)\le\Phi(\alpha)\le\max_{\alpha\in[0,1]}\Phi(\alpha),$$
which is a contradiction.




\begin{Claim}
For any $q\geq 1$, $$\limsup_{n\to+\infty}\frac{1}{n}\log\sup_{\sigma}{\rm Vol}\left(f^n(\sigma)\right)=\limsup_{n\to+\infty}\frac{1}{qn}\log\sup_{\sigma}{\rm Vol}\left(f^{qn}(\sigma)\right).$$
\end{Claim}
\begin{proof}[Proof of Claim]
	Set
	$$ a_n :=\sup_{\sigma}{\rm Vol}\left(f^n(\sigma)\right).$$
	The inequality ``$\ge$'' follows since $\{qn\}$ is a sub-sequence of $\mathbb{N}$.
	For the reverse inequality, write $n=qk+m$ with $0\le m<q$. Since $f^m$ has
	uniformly bounded $C^1$ norm for $0\le m<q$, there exists $C>0$ such that
	$$a_n \le C\, a_{qk}.$$
	Hence
	$$\frac{1}{n}\log a_n\le\frac{qk}{n}\cdot\frac{1}{qk}\log a_{qk}+\frac{1}{n}\log C,$$
	and letting $n\to+\infty$ yields the result.
\end{proof}

Replacing $f$ by $f^{q}$ and putting $\frac{1}{q}$ on both sides in (\ref{uniform volume growth}), by the arbitrariness of $q$, we get $$\limsup_{n\to+\infty}\frac{1}{n}\log\sup_{\sigma}{\rm Vol}\left(f^n(\sigma)\right)\leq \max_{\alpha\in[0,1]}\left\{\alpha\cdot h_{\rm top}(f) + (1-\alpha)\cdot\frac{\lambda^{+}(f)}{r}\right\}.$$
The proof of Theorem \ref{main theorem of decomposition of volume growth of submanifolds} is then complete.
\end{proof}

	\section{Volume growth of the tangent cocycle}\label{Volume growth of the tangent cocycle}
	In this section, we study the relationship between the volume growth of the tangent cocycle and entropy. More precisely, we establish the chain of inequalities
	$$h_{\rm top}(f)\leq\liminf_{n\to+\infty}\frac{1}{n}\log\int_{M}\left\|Df^{n}_{x}\right\|\,dx\leq  \limsup_{n\to+\infty}\frac{1}{n}\log\int_{M}\left\|Df^{n}_{x}\right\|\,dx\leq \alpha\cdot h_{\rm top}(f) + (1-\alpha)\cdot\frac{\lambda^{+}(f)}{r}.$$
	The rightmost inequality is Theorem~\ref{main theorem of decomposition of volume growth of derivatives}, while the leftmost inequality will be proved in Theorem~\ref{volume of tangent cocycles bounds entropy}.

\subsection{Upper bound of the volume growth rate}\label{Upper bound of the volume growth rate}
The following lemma asserts the existence of a point at which the exponential growth rate of the integral of $\left\|Df^{n}_{x}\right\|$ is fully realized locally. In other words, the global asymptotic growth of $\int_{M}\left\|Df^{n}_{x}\right\|\,dx$ can be detected on arbitrarily small neighborhoods of a single point.
\begin{Lemma}\label{existence of a maximal point}
	Let $f$ be a $C^{1}$ diffeomorphism on a compact manifold $M$. There is a point $z$ such that for any neighborhood $U$ of $z$, $$\limsup_{n\to+\infty}\frac{1}{n}\log\int_{U}\left\|Df^{n}_{x}\right\|\,dx=\limsup_{n\to+\infty}\frac{1}{n}\log\int_{M}\left\|Df^{n}_{x}\right\|\,dx.$$
\end{Lemma}
\begin{proof}
	Let $$\lambda(S) := \limsup_{n\to+\infty} \frac{1}{n} \log \int_{S} \left\|Df^n_x\right\| \, dx.$$ Suppose for contradiction that for every $z \in M$, there exists an open neighborhood $U_z$ such that $\lambda(U_z) < \lambda(M)$.

	By the compactness of $M$, there exists a finite sub-cover $\{U_{z_i}\}_{i=1}^k$ such that $M = \bigcup_{i=1}^k U_{z_i}$. Using the property that for non-negative sequences $a_{n,i}$,
	$$ \limsup_{n\to\infty} \frac{1}{n} \log \left( \sum_{i=1}^k a_{n,i} \right) = \max_{1 \le i \le k} \left( \limsup_{n\to\infty} \frac{1}{n} \log a_{n,i} \right), $$
	we have:
	$$ \lambda(M) = \lambda\left( \bigcup_{i=1}^k U_{z_i} \right) = \max_{1 \le i \le k} \lambda(U_{z_i}). $$
	Since $\lambda(U_{z_i}) < \lambda(M)$ for all $i$, we obtain $\lambda(M) < \lambda(M)$, a contradiction.
\end{proof}

The main idea to prove Theorem \ref{main theorem of decomposition of volume growth of derivatives} is to reduce the estimation of the volume growth of the tangent cocycle to the estimation of the volume growth of the one-dimensional sub-manifolds. Using Fubini's theorem, the two-dimensional integral of $\left\|Df_{x}^n\right\|$ is decomposed into integrals along horizontal and vertical lines, reducing the problem to estimating the volume growth of one-dimensional curves. The conclusion then follows by applying the curve volume growth estimate from Proposition~\ref{volume growth for time n}.
\begin{proof}[Proof of Theorem \ref{main theorem of decomposition of volume growth of derivatives}]


Let $\mathcal{A}:=\left\{\left(U_{i},\varphi_{i}\right):~\varphi_{i}:U_{i}\to[0,1]^{2}\right\}_{0\leq i<N_{0}}$ be a finite family of charts that cover $M$. Let $z$ be the point in Lemma \ref{existence of a maximal point}. Without loss of generality, we assume $z\in U_{0}$  and $U_{0}$ is a ball at $z$ with diameter far less than $\varepsilon_{f}$ ($\varepsilon_{f}$ is from Proposition \ref{Burguet-Yomdin reparametrization}) such that for any line $\sigma$ in $[0,1]^{2}$, $\varphi^{-1}_{0}\left(\sigma\right)$ is an $\varepsilon_{f}$-bounded curve on $M$. Let $K$ be a sufficiently large number (independent of $n$) such that

\begin{itemize}
	\item for any $0\leq i< N_{0}$ and any curve $\sigma$ on $U_{i}$, $$K^{-1}\cdot {\rm Vol}\left(\varphi_{i}\circ\sigma\right)\leq {\rm Vol}\left(\sigma\right)\leq K\cdot {\rm Vol}\left(\varphi_{i}\circ\sigma\right).$$
	\item for any $i$ and $j$, assuming $x\in U_{i}$ and $f^{n}(x)\in U_{j}$, $$K^{-1} \cdot \left\|D\left(\varphi_{j}\circ f^{n}\circ\varphi^{-1}_{i}\right)_{\varphi_{i}(x)}\right\|\leq \left\|Df_{x}^{n}\right\|\leq K \cdot \left\|D\left(\varphi_{j}\circ f^{n}\circ\varphi^{-1}_{i}\right)_{\varphi_{i}(x)}\right\|.$$
\end{itemize}


Given $n\in\mathbb{N}$, we fix any $0\leq i<N_{0}$ such that
\begin{equation}\label{major U0 i}
	\int_{U_{0}^{i}}\left\|Df^{n}_{x}\right\|\,dx\geq\frac{1}{N_{0}}\int_{U_{0}}\left\|Df^{n}_{x}\right\|\,dx
\end{equation}  where $$U_{0}^{i}:=\left\{x\in U_{0}:~ f^{n}\left(x\right)\in U_{i}\right\}.$$ Let $\left(x_{1}, x_{2}\right)$ be the standard coordinates in $\mathbb{R}^{2}$. By the choice of $K$, we note that 
\begin{equation}\label{volume growth in local coordinates}
	\int_{U_{0}^{i}}\left\|Df^{n}_{x}\right\|\,dx\leq K \int_{\varphi_{0}\left( U_{0}^{i}\right)}\left\|DF^{n}_{\left(x_{1},x_{2}\right)}\right\|\,dx_{1}dx_{2}
\end{equation} where $$F^{n}:=\varphi_{i}\circ f^{n}\circ\varphi^{-1}_{0}.$$  Given a $C^{1}$ map $G:=(G_{1}, G_{2})$ on $\mathbb{R}^{2}$, denote partial derivative vectors of $G$ by $$\partial_{1}G_{(x_{1},x_{2})}:=\left(\frac{\partial G_{1}}{\partial x_{1}}\left(x_{1},x_{2}\right), \frac{\partial G_{2}}{\partial x_{1}}\left(x_{1},x_{2}\right)\right),\quad \partial_{2}G_{(x_{1},x_{2})}:=\left(\frac{\partial G_{1}}{\partial x_{2}}\left(x_{1},x_{2}\right), \frac{\partial G_{2}}{\partial x_{2}}\left(x_{1},x_{2}\right)\right).$$  Note that, by standard linear algebra, $$\left\|DG_{\left(x_{1},x_{2}\right)}\right\|\leq\left\|\partial_{1}G_{(x_{1},x_{2})}\right\|+\left\|\partial_{2}G_{(x_{1},x_{2})}\right\|$$ where $\left\|\cdot\right\|$ denotes the usual operator norm for matrices and the usual Euclidean norm for vectors.  Consider the foliation of horizontal lines
$$
\left\{[0,1]\times x_{2}:\ x_{2}\in[0,1]\right\}.
$$


By Fubini's theorem and by the choice of $K$, we have

$$\begin{aligned}
	\int_{\varphi_{0}\left(U_{0}^{i}\right)}
	\left\|\partial_{1} F^{n}_{\left(x_{1},x_{2}\right)}\right\|\,dx_{1}dx_{2}
	&=\int_{0}^{1}
	\left(
	\int_{\varphi_{0}\left(U_{0}^{i}\right)\cap([0,1]\times x_{2})}
	\left\|\partial_{1} F^{n}_{\left(x_{1},x_{2}\right)}\right\|\,dx_{1}
	\right)dx_{2}\\
	&=\footnotemark\int_{0}^{1}
	{\rm Vol}\left(F^{n}\left(\varphi_{0}\left(U_{0}^{i}\right)\cap\left([0,1]\times x_{2}\right)\right)\right)dx_{2}\\
	&\leq \max_{0\le x_{2}\leq 1}{\rm Vol}\left(F^{n}\left(\varphi_{0}\left(U_{0}^{i}\right)\cap\left([0,1]\times x_{2}\right)\right)\right)\\
	&\leq K\cdot \max_{0\le x_{2}\leq 1}{\rm Vol}\left(f^{n}\left(U_{0}^{i}\cap\varphi_{0}^{-1}\left([0,1]\times x_{2}\right)\right)\right)\\
	&\leq K\cdot \max_{0\le x_{2}\leq 1}{\rm Vol}\left(f^{n}\left(\varphi_{0}^{-1}\left([0,1]\times x_{2}\right)\right)\right).
\end{aligned}$$\footnotetext{Consider a $C^{1}$ map $G:=(G_{1}, G_{2})$ on $\mathbb{R}^{2}$. Fix $x_{2}\in[0,1]$ and consider the $C^{1}$ curve
$\gamma_{x_{2}}(x_{1}):=(x_{1},x_{2})$.
Then $G\circ\gamma_{x_{2}}(x_{1})=G(x_{1},x_{2})$ and
$$
\frac{d}{dx_{1}}\big(G\circ\gamma_{x_{2}}\big)(x_{1})
=
\left(\frac{\partial G_{1}}{\partial x_{1}}(x_{1},x_{2}),
\frac{\partial G_{2}}{\partial x_{1}}(x_{1},x_{2})\right)
=:
\partial_{1}G_{(x_{1},x_{2})}.
$$
By the length formula for $C^{1}$ curves, the length of
$G(\gamma_{x_{2}})$ is therefore given by
$$
{\rm Vol}\big(G(\gamma_{x_{2}})\big)
=
\int \left\|\partial_{1}G_{(x_{1},x_{2})}\right\|\,dx_{1}.
$$}
Similarly, by considering vertical lines, $$\int_{\varphi_{0}\left(U_{0}^{i}\right)}
\left\|\partial_{2} F^{n}_{\left(x_{1},x_{2}\right)}\right\|\,dx_{1}dx_{2}\leq K\cdot \max_{0\le x_{1}\leq 1}{\rm Vol}\left(f^{n}\left(\varphi_{0}^{-1}\left(x_{1}\times[0,1] \right)\right)\right).$$ We define $\sigma_n$ to be the pre-image under $\varphi_0$ of a horizontal or vertical line in $[0,1]^2$ whose image under $f^n$ attains the maximal volume among all such coordinate lines above, namely, $${\rm Vol}\left(f^{n}\left(\sigma_{n}\right)\right)=\max_{0\leq x_{1},x_{2}\leq 1}\left\{{\rm Vol}\left(f^{n}\left(\varphi_{0}^{-1}\left([0,1]\times x_{2}\right)\right)\right),\,{\rm Vol}\left(f^{n}\left(\varphi_{0}^{-1}\left(x_{1}\times[0,1] \right)\right)\right)\right\}.$$
Hence
$$\begin{aligned}
	\int_{\varphi_{0}\left(U_{0}^{i}\right)}
	\left\| DF^{n}_{(x_{1},x_{2})} \right\| \, dx_{1} dx_{2}
	&\leq
	\int_{\varphi_{0}\left(U_{0}^{i}\right)}
	\left(
	\left\| \partial_{1} F^{n}_{(x_{1},x_{2})} \right\|
	+
	\left\| \partial_{2} F^{n}_{(x_{1},x_{2})} \right\|
	\right)
	\, dx_{1} dx_{2}\\
	&\leq
	2K\cdot{\rm Vol}\left(f^{n}\left(\sigma_{n}\right)\right).
\end{aligned}$$


	Applying Proposition \ref{volume growth for time n} to $\sigma_{n}$, recalling (\ref{major U0 i}) and (\ref{volume growth in local coordinates}), we have
	\begin{equation}\label{rough estimations of volume grwoth 2}
	\begin{aligned}
		\limsup_{n\to+\infty}\frac{1}{n}\log\int_{M}\left\|Df^{n}_{x}\right\|\,dx&=\limsup_{n\to+\infty}\frac{1}{n}\log\int_{U_{0}}\left\|Df^{n}_{x}\right\|\,dx\\
		&\leq\limsup_{n\to+\infty}\frac{1}{n}\log{\rm Vol}\left(f^{n}\left(\sigma_{n}\right)\right)\\
		&\leq \max_{\alpha\in[0,1]}\left\{\alpha\cdot h_{\rm top}(f)+\left(1-\alpha\right)\cdot\frac{\log\left\|Df\right\|}{r}\right\}\\
		&\quad+4\log\left(\log\left\|Df\right\|+\log\left\|Df^{-1}\right\|+2\right)+C_{r}
	\end{aligned}
\end{equation} where  $C_{r}$ is a universal constant depending only on $r$.


	Note that $$\limsup_{n\to+\infty}\frac{1}{n}\log\int_{M}\left\|Df^{n}_{x}\right\|\,dx=\limsup_{n\to+\infty}\frac{1}{qn}\log\int_{M}\left\|Df^{qn}_{x}\right\|\,dx,\quad h_{\rm top}(f)=\frac{h_{\rm top}(f^{q})}{q}.$$ Replacing $f$ by $f^{q}$ and putting $\frac{1}{q}$ on both sides in (\ref{rough estimations of volume grwoth 2}), by the arbitrariness of $q$, we get the conclusion: $$\limsup_{n\to+\infty}\frac{1}{n}\log\int_{M}\left\|Df^{n}_{x}\right\|\,dx\leq \max_{\alpha\in[0,1]}\left\{\alpha\cdot h_{\rm top}(f) + (1-\alpha)\cdot\frac{\lambda^{+}(f)}{r}\right\}.$$

	The proof of Theorem \ref{main theorem of decomposition of volume growth of derivatives} is then complete.
\end{proof}
\subsection{Lower bound of the volume growth rate}
The goal of this section is to establish the following result, which can be viewed as a consequence of Pesin theory and is, in a certain sense, already known.
The distinction here lies in the formulation: Theorem~\ref{volume of tangent cocycles bounds entropy} is stated directly in terms of the derivative $Df^{n}_{x}$, rather than the induced map $(Df^{n}_{x})^{\wedge}$ on exterior algebras of tangent spaces, as considered by Przytycki~\cite{Prz80} and Kozlovski~\cite{Koz98}.

\begin{Theorem}\label{volume of tangent cocycles bounds entropy}
	Let $f$ be a $C^r$ ($r>1$) diffeomorphism on a compact surface $M$. $$ h_{\rm top}(f)\leq\liminf_{n\to+\infty}\frac{1}{n}\log\int_{M}\left\|Df^{n}_{x}\right\|\,d x.$$
\end{Theorem}

We first recall a result of Katok (Theorem 1.1 in \cite{Kat80}).

Let $\mu$ be an ergodic measure. Given $\lambda\in(0,1)$, we define $S_{\lambda}(n,\varepsilon)\subset M$ to be a finite set such that
$$\mu\left(\bigcup_{x\in S_{\lambda}(n,\varepsilon)} B(x,n,\varepsilon)\right)\ge \lambda,$$
and whose cardinality is minimal among all finite subsets of $M$ with this property. Recall (see \cite{Wal82} for basic background in ergodic theory)  that a subset $S\subset M$ is called a \emph{$(n,\varepsilon)$ separated set} if for any $x,y\in S$ with $x\neq y$, there is $0\leq k<n$ such that $d(f^{k}(x), f^{k}(y))>\varepsilon$. Moreover, a separated set $S$ is called \emph{maximal} if its cardinality is larger than or equal to the cardinality of any other separated set. Given a subset $\Omega$, let $S(n,\varepsilon,\Omega)$ denote a maximal $(n,\varepsilon)$ separated set of $\Omega$. By maximality, the union of the dynamical balls $\{B(x,n,\varepsilon)\}_{x\in S(n,\varepsilon,\Omega)}$ covers $\Omega$.

\begin{Lemma}[Katok, \cite{Kat80}]\label{Katok entropy formula}
	Let $f$ be a homeomorphism on a compact metric space $X$ and let $\mu$ be an ergodic measure.  Then for any $\lambda\in(0,1)$, $$h(f,\mu)=\lim_{\varepsilon\to 0}\liminf_{n\to +\infty}\frac{1}{n}\log\# S_{\lambda}(n,\varepsilon)=\lim_{\varepsilon\to 0}\limsup_{n\to +\infty}\frac{1}{n}\log\# S_{\lambda}(n,\varepsilon).$$
\end{Lemma}
The following lemma, due to Przytycki, as an application of Pesin theory, provides a uniform lower bound on the integral of the derivative over dynamical balls centered at typical points of a hyperbolic measure.
\begin{Lemma}\cite[Lemma 2.1]{Prz80}\label{Przytycki Lemma}
		Let $f$ be a $C^r$ ($r>1$) diffeomorphism on a compact surface $M$ and let $\mu$ be a hyperbolic ergodic measure. There are a subset $\Omega\subset M$ with $\mu\left(\Omega\right)>\frac{1}{2}$ and a constant $C>0$ such that for any $z\in\Omega$, any $n\in\mathbb{N}$ and any $0<\varepsilon<2026$, $$\int_{B(z,n,\varepsilon)}\left\|Df^{n}_{x}\right\|\,dx\geq C\cdot\varepsilon^{2}\cdot\frac{1}{2^{n}}.$$
\end{Lemma}
\begin{Remark}
	The original statement of Lemma~\ref{Przytycki Lemma} is formulated for the induced map  $\left(Df^{n}_{x}\right)^{\wedge}$ acting on the exterior algebra of the tangent space.
	However, one can see from the proof that the same conclusion holds when
	$\left(Df^{n}_{x}\right)^{\wedge}$ is replaced by $\left\|Df^{n}_{x}\right\|$, since for a hyperbolic measure on a surface the unstable dimension is equal to one.
\end{Remark}

\begin{proof}[Proof of Theorem \ref{volume of tangent cocycles bounds entropy}]
	Given a hyperbolic ergodic measure $\mu$, by Lemma \ref{Przytycki Lemma}, there are a subset $\Omega\subset M$ with $\mu\left(\Omega\right)\geq\frac{1}{2}$ and a constant $C>0$ such that for any $z\in\Omega$, any $n\in\mathbb{N}$ and any $0<\varepsilon<2026$, $$\int_{B(z,n,\varepsilon)}\left\|Df^{n}_{x}\right\|\,dx\geq C\cdot\varepsilon^{2}\cdot\frac{1}{2^{n}}.$$ Let $S(n,2\varepsilon,\Omega)$ be a maximal $(n,2\varepsilon)$ separated set of $\Omega$. By definition, for any $z_{1}, z_{2}\in S(n,2\varepsilon,\Omega)$, $$B(z_{1}, n,\varepsilon)\cap B(z_{2}, n,\varepsilon)=\emptyset.$$
We have
$$\begin{aligned}
	\int_{M}\left\|Df_{x}\right\|\,d x
	&\geq \# S(n,2\varepsilon,\Omega)\cdot\min_{z\in S(n,2\varepsilon,\Omega)}\int_{B(z,n,\varepsilon)} \left\|Df_{x}\right\|\,dx\\
	&\geq \# S(n,2\varepsilon,\Omega)\cdot C\cdot\varepsilon^{2}\cdot\frac{1}{2^{n}}.
\end{aligned}$$
Then $$\liminf_{n\to+\infty}\frac{1}{n}\log\int_{M}\left\|Df_{x}\right\|\,d x\geq \liminf_{n\to+\infty}\frac{1}{n}\log \#S(n,2\varepsilon
,\Omega)-\log 2.$$

Since $\mu(\Omega)>\frac{1}{2}$, we have $\#S(n,2\varepsilon,
\Omega)\geq \#S_{\frac{1}{2}}(n,2\varepsilon)$. By Lemma \ref{Katok entropy formula}, by the arbitrariness of $\varepsilon$, we then have $$\liminf_{n\to+\infty}\frac{1}{n}\log\int_{M}\left\|Df^{n}_{x}\right\|\,d x\geq h(f,\mu)-\log 2.$$ By the variational principle, we have
\begin{equation}\label{rough estimations of volume grwoth 3}
\liminf_{n\to+\infty}\frac{1}{n}\log\int_{M}\left\|Df^{n}_{x}\right\|\,d x\geq h_{\rm top}(f)-\log 2.
\end{equation}
Note that $$\liminf_{n\to+\infty}\frac{1}{n}\log\int_{M}\left\|Df^{n}_{x}\right\|\,dx=\liminf_{n\to+\infty}\frac{1}{qn}\log\int_{M}\left\|Df^{qn}_{x}\right\|\,dx,\quad h_{\rm top}(f)=\frac{h_{\rm top}(f^{q})}{q}.$$ Replacing $f$ by $f^{q}$ and putting $\frac{1}{q}$ on both sides in (\ref{rough estimations of volume grwoth 3}), by the arbitrariness of $q$, we finally get the conclusion: $$\liminf_{n\to+\infty}\frac{1}{n}\log\int_{M}\left\|Df^{n}_{x}\right\|\,dx\geq h_{\rm top}(f).$$

\end{proof}

\section{Appendix}

The following lemma establishes an upper semi-continuity property for integrals of sub-additive potentials along converging sequences of invariant measures. This result is a special case of \cite[Lemma~2.3]{CFH08}.
	\begin{Lemma}\label{upper continuity of subadditive functions}
	Let $f$ be a homeomorphism on a compact metric space $X$. Let $\mu_{n}$ be a sequence of invariant probability measures converging to some invariant probability measure $\mu$ and let $\varphi_{n}$ be a sequence of sub-additive continuous functions. We have $$\limsup_{n\to+\infty}\int\frac{\varphi_{n}}{n}d\,\mu_{n}\leq \int\limsup_{n\to+\infty}\frac{\varphi_{n}}{n}d\,\mu.$$
	\end{Lemma}

	\begin{proof}
		Fix $m \ge 1$. For any $n$, write $n = km + r$ with $0 \le r < m$. By the sub-additivity of $\varphi_n$, we have
		\[
		\varphi_n \le \sum_{j=0}^{k-1} \varphi_m \circ f^{jm} + \varphi_r \circ f^{km}.
		\]
		Integrating w.r.t. the $f$-invariant measure $\mu_n$, we obtain
		\[
		\int \varphi_n \, d\mu_n \le k \int \varphi_m \, d\mu_n + C_m,
		\]
		where $C_m:= \max_{0 \le r < m} \max_{x}\varphi_{r}(x)$. Dividing by $n$ yields
		\[
		\int \frac{\varphi_n}{n} \, d\mu_n \le \frac{k}{n} \int \varphi_m \, d\mu_n + \frac{C_m}{n} \le \frac{1}{m} \int \varphi_m \, d\mu_n + \frac{C_m}{n}.
		\]
		Since $\mu_n \to \mu$ and $\varphi_m$ is continuous, letting $n \to +\infty$ gives
		\[
		\limsup_{n \to +\infty} \int \frac{\varphi_n}{n} \, d\mu_n \le \frac{1}{m} \int \varphi_m \, d\mu.
		\]
		Taking the infimum over $m \ge 1$, and applying Kingman's Sub-additive Ergodic Theorem (specifically, $\inf_m \frac{1}{m} \int \varphi_m \, d\mu = \int \lim_n \frac{\varphi_n}{n} \, d\mu$), we conclude
		\[
		\limsup_{n \to +\infty} \int \frac{\varphi_n}{n} \, d\mu_n \le \inf_{m \ge 1} \frac{1}{m} \int \varphi_m \, d\mu = \int \limsup_{n \to +\infty} \frac{\varphi_n}{n} \, d\mu.
		\]
	\end{proof}
	\begin{Lemma}[Reverse Fatou Lemma]\label{Reverse Fatou Lemma}
		Let $X$ be a compact metric space and let $\mu$ be a Borel probability measure on $X$.
		Let $\{f_n\}_{n\ge1}$ be a sequence of measurable functions on $X$ such that
		$\sup_{n\ge1}\left\|f_n\right\|_\infty < +\infty$.
		Then
		$$	\int_X \limsup_{n\to+\infty} f_n\,d\mu \geq\limsup_{n\to+\infty}\int_X f_n\,d\mu.$$
	\end{Lemma}



















	\begin{Lemma}[Decomposition into $\varepsilon$-bounded curves]\label{Decomposition into varepsilon bounded curves}
		Let $M$ be a compact manifold and let $r>1$.
		Then for any $0<\varepsilon\le \frac{1}{100}$, any $C^{r}$ curve $\sigma:[0,1]\to M$ with
		$$
		\left\|\sigma\right\|_{C^{r}}=\max_{1\le s\le r}\left\|d^{s}\sigma\right\|\le 1 \,\text{ and }\, \left\|d_{t}\sigma\right\| \geq 1/2, \,t\in[0,1],
		$$ there are $\varepsilon$-bounded $C^{r}$ curves $\{\sigma_{j}\}_{1\leq j \leq \left\lceil \frac{1}{\varepsilon}\right\rceil}$ such that
		$$
		\sigma\left([0,1]\right)=\bigcup_{j=1}^{\left\lceil \frac{1}{\varepsilon}\right\rceil}\sigma_{j}\left([0,1]\right).
		$$
	\end{Lemma}

	\begin{proof}
		Take $N = \left\lceil 1/\varepsilon \right\rceil$ and define the step size $\delta = \frac{1}{N}$. Note that by assumption $\varepsilon \le \frac{1}{100}$, we have $\delta \le \varepsilon \le \frac{1}{100}$.

		Define the translation constants $b_j$ for $j=1, \dots, N$ as follows:
		$$ b_j = \begin{cases} (j-1)\delta, & 1 \le j < N \\ 1 - \delta, & j = N \end{cases} $$
		The affine maps $\theta_j: [0,1] \to [0,1]$ are given by $\theta_j(t) = \delta t + b_j$. By construction, $\theta_j([0,1]) = [b_j, b_j+\delta] \subset [0,1]$, and the union of these images covers $[0,1]$ (with the last interval $[1-\delta, 1]$ potentially overlapping with the previous one).

		Let $\sigma_j = \sigma \circ \theta_j$. We verify the $\varepsilon$-boundedness for each $1 \le j \le N$:

		\begin{enumerate}
			\item \textbf{First derivative bound.}

			Since $\left\|d_t\sigma\right\| \le 1$ and $\theta_j' = \delta$, we have
			$$ \left\|d\sigma_j\right\| = \sup_{t \in [0,1]} \left\|d_t\sigma \cdot \theta_j'\right\| \le 1 \cdot \delta \le \varepsilon. $$

			\item \textbf{Higher order derivative bounds.}

			For $2 \le s \le r$, the chain rule gives $\left\|d^s\sigma_j\right\| \le \delta^s \left\|d^s\sigma\right\| \le \delta^s$. Recalling the lower bound $\left\|d_t\sigma\right\| \ge 1/2$, we have $\left\|d\sigma_j\right\| \ge \frac{1}{2}\delta$. Consequently,
			$$ \frac{\left\|d^s\sigma_j\right\|}{\left\|d\sigma_j\right\|} \le \frac{\delta^s}{\frac{1}{2}\delta} = 2\delta^{s-1} \le 2\delta \le \frac{2}{100} < \frac{1}{6}, $$
			which satisfies the boundedness condition for all $2 \le s \le r$.
		\end{enumerate}
		Therefore, each $\sigma_j$ is $\varepsilon$-bounded, and $\sigma([0,1]) = \bigcup_{j=1}^N \sigma_j([0,1])$.
	\end{proof}




































	The following example shows that the local volume growth rate can still be positive in a small neighborhood of a repeller. Here the positivity of the local volume growth is a purely local phenomenon: although the curve is exponentially restricted to an increasingly small neighborhood of the repeller, the rapidly growing oscillation frequency produces a strong transverse amplification of derivatives. The resulting growth reflects a subtle balance between geometric restriction and oscillatory stretching, rather than uniform expansion.
	\begin{Example}\label{example repeller positive volume growth}
Let $A = \operatorname{diag}(a, 3)$ with $a > 1$, and let $\sigma: [0, 1] \to \mathbb{R}^2$ be a $C^2$ curve defined by $\sigma(x) = (x, x^5 \sin(1/x))$. The volume (i.e., arc length) growth rate restricted in $[-1,1]^2$, defined by
\[ {\rm Vol}(A, \sigma)_{[-1,1]^2} := \limsup_{n \to +\infty} \frac{1}{n} \log \operatorname{Length}(A^n \sigma \cap [-1,1]^2), \]
is given by:
\[ {\rm Vol}(A, \sigma)_{[-1,1]^2} =
\begin{cases}
	\frac{1}{5} \log 3, & 1 < a \le 3^{1/5} \\
	\log 3 - 4 \log a, & 3^{1/5} < a < 3^{1/4} \\
	0, & a \ge 3^{1/4}
\end{cases} \]
In particular, the growth rate is positive if and only if $1 < a < 3^{1/4}$.
	\end{Example}

	\begin{proof}
		\textbf{Step 1. Parametrization and derivatives.}

		Inside $[-1,1]^2$, the image curve $A^n\sigma$ admits the parametrization
		$$
		\gamma_n(u)=(u,g_n(u)), \qquad u\in[0,1],
		$$
		where
		$$
		g_n(u)=3^n a^{-5n}u^5\sin\!\left(\frac{a^n}{u}\right).
		$$

		The admissible domain to have $\gamma_n(u)=(u,g_n(u))\in[-1,1]^2$ is determined by
		$$
		u\in[0,1],\qquad |g_n(u)|\le1.
		$$
		Denote by $I_n$ the set of $u$ satisfying these conditions. The length of
		$A^n\sigma\cap[-1,1]^2$ is
		$$
		L_n:=\int_{I_n}\sqrt{1+|g_n'(u)|^2}\,du.
		$$
		A direct computation gives
		$$
		g_n'(u)
		=5\cdot3^n a^{-5n}u^4\sin\!\left(\frac{a^n}{u}\right)
		-3^n a^{-4n}u^3\cos\!\left(\frac{a^n}{u}\right).
		$$

		\medskip
		\textbf{Step 2. Case analysis.}

	The figure \ref{example} below illustrates how the local volume growth transitions from bounded oscillations to amplified and eventually explosive behavior as the parameter $a$ varies across different regimes.
		\begin{figure}[htbp]
			\centering
			\includegraphics[width=1\textwidth]{example}
			\caption{Local volume growth for different value of $a$}
			\label{example}
		\end{figure}



		\smallskip

		\begin{itemize}
			\item 	\emph{Case 1: $a\ge 3^{1/4}$.}

			In this range, $3a^{-4}\le1$. For $u\in[0,1]$, we have
			$$
			|g_n(u)|\le (3a^{-5})^n\leq 1,
			$$
			which implies $I_{n}=[0,1]$. Also note
			$$
			|g_n'(u)|
			\le 5\cdot3^n a^{-5n}+3^n a^{-4n}
			=5(3a^{-5})^n+(3a^{-4})^n.
			$$
			As $a>1$ and $3a^{-4}\le1$, there exists $C>0$ such that
			$|g_n'(u)|\le C$ for all $n$ and $u\in I_n$. Hence
			$$
			1\le \sqrt{1+|g_n'(u)|^2}\le\sqrt{1+C^2},
			$$
			which implies
			$$
			{\rm Vol}(A, \sigma)_{[-1,1]^2}=\limsup_{n\to+\infty}\frac1n\log L_n\leq0.
			$$
			Since $I_{n}=[0,1]$, the image curve has length at least one. Hence $${\rm Vol}(A, \sigma)_{[-1,1]^2}=0$$

			\item  \emph{Case 2: $3^{1/5}<a<3^{1/4}$.}

			In this range, $3a^{-5}<1$ and $3a^{-4}>1$. Since
			$$
			|g_n(u)|\le (3a^{-5})^n\leq 1,
			$$
			we have $I_n=[0,1]$ as in the previous case.

			Note $$5\cdot3^n a^{-5n}u^4\sin\!\left(\frac{a^n}{u}\right)\leq 5\cdot3^n a^{-5n}\to 0.$$
			As a consequence, for all large $n$, $$\sqrt{1+|g_n'(u)|^2}\le1+|g_n'(u)|\leq 2+ 3^n a^{-4n}.$$ Hence $${\rm Vol}(A, \sigma)_{[-1,1]^2}\leq \log3-4\log a.$$ It remains to prove the reverse inequality.


			 For all large $n$, by Lemma \ref{Calculus results} with $b=1$, $$L_{n}\geq \int_{0}^{1}|g_n'(u)|du\geq \int_{0}^{1}\Big|3^n a^{-4n}u^3\cos\!\left(\frac{a^n}{u}\right)\Big|du-1\geq 3^n a^{-4n}\cdot \frac{1}{2\pi(1+2\pi)^{4}} -1.$$ Hence $${\rm Vol}(A, \sigma)_{[-1,1]^2}\geq \log3-4\log a.$$

			\item 	\emph{Case 3: $1<a\le 3^{1/5}$.}

			In this range, $3a^{-5}\ge1$.
			For a lower bound, consider $[0,u_n]$ with
			$$
			u_n:=(3^{n}a^{-5n})^{-1/5}=3^{-n/5}a^n.
			$$
			Then $|g_n(u)|\le1$ on $[0,u_n]$, hence $[0,u_n]\subset I_n$. Also note $$\sqrt{1+|g_n'(u)|^2}\geq |g_n'(u)|\geq  \left|3^n a^{-4n}u^3\cos\!\left(\frac{a^n}{u}\right)\right|-5\cdot3^n a^{-5n}u^4.$$ Applying Lemma \ref{Calculus results} with $b=u_n=3^{-n/5}a^n$, we have $$\int_{0}^{u_{n}}\left|3^n a^{-4n}u^3\cos\!\left(\frac{a^n}{u}\right)\right|du\geq \frac{1}{2\pi\left(1+2\pi\right)^{4}}\cdot 3^{n/5}.$$ Hence $$L_{n}\geq \int_{0}^{u_{n}}\left|3^n a^{-4n}u^3\cos\!\left(\frac{a^n}{u}\right)\right|du-\int_{0}^{u_{n}}5\cdot3^n a^{-5n}u^4du\geq \frac{1}{2\pi\left(1+2\pi\right)^{4}}\cdot 3^{n/5}-1$$ which gives $${\rm Vol}(A, \sigma)_{[-1,1]^2}\geq \frac15\log3.$$

			It remains to prove the reverse inequality. On $[0, u_{n}]$, $$\sqrt{1+|g_n'(u)|^2}\leq |g_n'(u)|+1\leq 3^n a^{-4n}u^3+5\cdot3^n a^{-5n}u^4 +1.$$ By a direct computation,  $$\limsup_{n \to +\infty}\frac{1}{n}\log\int_{0}^{u_{n}}\left(3^n a^{-4n}u^3+5\cdot3^n a^{-5n}u^4 +1\right)du \leq \frac15\log3.$$ So we only have to focus on the volume growth on the interval $[u_{n},1]$.

			We next estimate the number of intervals of monotonicity for $g_n$ on $[u_{n},1]$. Note that the critical points (i.e., $g'_{n}(u)=0$) are determined by
			\[
			5u\sin\!\frac{a^n}{u}
			=
			a^n\cos\!\frac{a^n}{u}.
			\]
			Introducing the variable $t=a^n/u$, this equation is equivalent to
			\[
			\tan t=\frac{t}{5}.
			\]
			For $u\in [u_{n},1]$, we have
			\[
			t\in\Bigl[a^n,\frac{a^n}{u_n}\Bigr]=[a^n,3^{n/5}].
			\]
			On each interval $(k\pi-\frac{\pi}{2},k\pi+\frac{\pi}{2})$, the function
			$\tan t - t/5$ is strictly increasing, hence the above equation admits at most one solution.
			Therefore the number of solutions in $[a^n,3^{n/5}]$ is bounded by
			\[
			\frac{3^{n/5}-a^n}{\pi}+1 \le \frac{3^{n/5}}{\pi}+1.
			\]
			Each critical point contributes at most one change of monotonicity, so the number of intervals of monotonicity on $[u_{n},1]$ is bounded by
			\[
			\frac{3^{n/5}}{\pi}+2.
			\]

			Restricted on $[-1,1]^2$, the length of the curve on each interval of monotonicity is at most $4$. Hence the volume growth rate on the interval $[u_{n},1]$ restricted on $[-1,1]^2$ is bounded above by also $\frac15\log3.$ This proves the reverse inequality.

		\end{itemize}

	\end{proof}

		\begin{Lemma}\label{Calculus results}
		For any $a,b>0$ and $n\in\mathbb{N}$,
		$$
		\int_{0}^{b} u^{3}\left|\cos\frac{a^{n}}{u}\right|\,du
		\ge
		\frac{a^{4n} b^{4}}{2\pi\left(a^{n}+2\pi b\right)^{4}}.
		$$
	\end{Lemma}

	\begin{proof}
		By the change of variables $t=a^{n}/u$, we obtain
		$$
		\int_{0}^{b} u^{3}\left|\cos\frac{a^{n}}{u}\right|\,du
		=
		a^{4n}\int_{a^{n}/b}^{\infty} t^{-5}|\cos t|\,dt .
		$$

		Let
		$$
		k_{0}:=\left\lceil \frac{a^{n}}{b\pi}\right\rceil .
		$$
		Since $t^{-5}|\cos t|\ge 0$, we have
		$$
		\int_{a^{n}/b}^{\infty} t^{-5}|\cos t|\,dt
		\ge
		\sum_{k\ge k_{0}}
		\int_{k\pi}^{(k+1)\pi} t^{-5}|\cos t|\,dt .
		$$
		For each $k\ge k_{0}$, using that $t^{-5}$ is decreasing on $[k\pi,(k+1)\pi]$ and
		$$
		\int_{k\pi}^{(k+1)\pi}|\cos t|\,dt=2,
		$$
		we obtain
		$$
		\int_{k\pi}^{(k+1)\pi} t^{-5}|\cos t|\,dt
		\ge
		2\,((k+1)\pi)^{-5}.
		$$
		Therefore,
		$$
		\int_{a^{n}/b}^{\infty} t^{-5}|\cos t|\,dt
		\ge
		2\sum_{k\ge k_{0}}((k+1)\pi)^{-5}
		\ge
		\frac{2}{\pi}\int_{(k_{0}+1)\pi}^{\infty} t^{-5}\,dt
		=
		\frac{1}{2\pi}((k_{0}+1)\pi)^{-4}.
		$$
		Since $(k_{0}+1)\pi\le a^{n}/b+2\pi$, it follows that
		$$
		\int_{a^{n}/b}^{\infty} t^{-5}|\cos t|\,dt
		\ge
		\frac{1}{2\pi}(a^{n}/b+2\pi)^{-4}.
		$$
		Consequently,
		$$
		\int_{0}^{b} u^{3}\left|\cos\frac{a^{n}}{u}\right|\,du
		\ge
		\frac{a^{4n} b^{4}}{2\pi\left(a^{n}+2\pi b\right)^{4}}
		.$$
	\end{proof}

	\begin{Acknowledgements}
		We are grateful to Mingyang Xia from Dalian University of Technology for many helpful and stimulating discussions during the early stages of this work.
	\end{Acknowledgements}



\begin{thebibliography}{999}


\bibitem{BeB25}
\newblock S. Ben Ovadia and D. Burguet,
\newblock Generalized $u$-Gibbs measures for $C^{\infty}$ diffeomorphisms.
\newblock {\em 	 arXiv:2506.18238v1}, (2025).

\bibitem{Bur20}
\newblock D. Burguet,
\newblock Periodic expansiveness of smooth surface diffeomorphisms and applications.
\newblock {\em J. Eur. Math. Soc.}, \textbf{22} (2020), 413--454.

\bibitem{Bur24}
\newblock D. Burguet,
\newblock Maximal measure and entropic continuity of Lyapunov exponents for $C^{r}$ surface diffeomorphisms with large entropy.
\newblock {\em Ann. Henri Poincar\'e}, \textbf{25} (2024), 1485--1510.


\bibitem{Bur24'}
\newblock D. Burguet,
\newblock SRB measures for $C^{\infty}$ surface diffeomorphisms.
\newblock {\em Invent. Math.}, \textbf{235} (2024), 1019--1062.

\bibitem{BLY25'}
\newblock D. Burguet, C. Luo and D. Yang,
\newblock Effective SPR property for surface diffeomorphisms and three-dimensional vector fields.
\newblock {\em 	arXiv:2512.03515v1}, (2025).

\bibitem{Buz96}
\newblock J. Buzzi,
\newblock Entropy, volume growth and Lyapunov exponents.
\newblock \emph{unpublished}, 1996.

\bibitem{BCS22'}
\newblock J. Buzzi, S. Crovisier and O. Sarig,
\newblock Measures of maximal entropy for surface diffeomorphisms.
\newblock \emph{Ann. of Math.}, \textbf{195}(2022), 421--508.

\bibitem{BCS22}
\newblock J. Buzzi, S. Crovisier and O. Sarig,
\newblock Continuity properties of Lyapunov exponents for surface diffeomorphisms.
\newblock \emph{Invent. Math.}, \textbf{230}(2022), 767--849.

\bibitem{BCS25}
\newblock J. Buzzi, S. Crovisier and O. Sarig,
\newblock Strong positive recurrence and exponential mixing for diffeomorphisms.
\newblock {\em 	arXiv:2501.07455}, (2025).

\bibitem{BLY25}
\newblock J. Buzzi, C. Luo and D. Yang,
\newblock Continuity properties of ergodic measures of maximal entropy for $C^r$ surface diffeomorphisms.
\newblock {\em 	arXiv: 2412.19658v3}, (2025).

\bibitem{CFH08}
\newblock Y. Cao, D. Feng and W. Huang,
\newblock The thermodynamic formalism for sub-additive potentials.
\newblock \emph{Discrete and Continuous Dynamical Systems}, \textbf{20}(2008), 639--657.

\bibitem{Cog00}
\newblock K. Cogswell,
\newblock Entropy and volume growth.
\newblock \emph{Ergodic Theory Dynam. Systems}, \textbf{20}(2000), 77--84.



\bibitem{GLS18}
\newblock  X. Guo, G. Liao, W. Sun and D. Yang,
\newblock On the hybrid control of metric entropy for dominated splittings.
\newblock \emph{Discrete and Continuous Dynamical Systems}, \textbf{38}(2018), 5011--5019.











\bibitem{Koz98}
\newblock O. Kozlovski,
\newblock An integral formula for topological entropy of $C^{\infty}$ maps.
\newblock \emph{Ergodic Theory Dynam. Systems}, \textbf{18}(1998), 405--424.

\bibitem{Kat80}
\newblock A. Katok,
\newblock Lyapunov exponents, entropy and periodic orbits for diffeomorphisms.
\newblock \emph{Inst. Hautes \'{E}tudes Sci. Publ. Math.}, \textbf{51}(1980), 137--173.

\bibitem{KlY95}
\newblock I. Klapper and L.-S. Young,
\newblock Rigorous bounds on the fast dynamo growth rate involving topological entropy.
\newblock \emph{Comm. Math. Phys.}, \textbf{173}(1995), 623--646.

\bibitem{LlS09}
\newblock J. Llibre and R. Saghin,
\newblock Results and open questions on some invariants measuring the dynamical complexity of a map.
\newblock \emph{Fundamenta Mathematicae}, \textbf{206}(2009), 307--327.

\bibitem{LuY25}
\newblock C. Luo and D. Young,
\newblock Upper semi-continuity of metric entropy for $C^{1,\alpha}$ diffeomorphisms.
\newblock {\em 	 arXiv:2504.07746v2}, (2025).

\bibitem{MaX20}
\newblock X. Ma,
\newblock Estimation of topological entropy in random dynamical systems.
\newblock \emph{J. Differential Equations}, \textbf{269}(2020), 892--911.


\bibitem{New88}
\newblock S. E. Newhouse,
\newblock Entropy and volume.
\newblock \emph{Ergodic Theory Dynam. Systems}, \textbf{8}(1988), 283--299.

\bibitem{Pol93}
\newblock M. Pollicott,
\newblock Lectures on Ergodic Theory and Pesin Theory on Compact Manifolds.
\newblock \emph{London Mathematical Society Lecture Note Series}, Cambridge University Press, Cambridge (1993).



\bibitem{Prz80}
\newblock F. Przytycki,
\newblock An upper estimation for topological entropy of diffeomorphisms.
\newblock \emph{Invent. Math.}, \textbf{59}(1980), 205--213.

\bibitem{SaS78}
\newblock R. Sacksteder and M. Shub,
\newblock Entropy of a differentiable map.
\newblock \emph{Advances in Math.}, \textbf{28}(1978), 181--185.

\bibitem{Sag14}
\newblock R. Saghin,
\newblock Volume growth and entropy for $C^{1}$ partially hyperbolic diffeomorphisms.
\newblock \emph{Discrete and Continuous Dynamical Systems}, \textbf{34}(2014), 3789--3801.

\bibitem{Wal82}
\newblock P. Walters,
\newblock An introduction to ergodic theory.
\newblock \emph{volume 79 of Graduate Texts in Mathematics. Springer-Verlag, New York-Berlin}, (1982).

\bibitem{YaZ23}
\newblock D. Yang and Y. Zang,
\newblock Volume growth and topological entropy of partially hyperbolic systems.
\newblock \emph{Israel J. Math.}, \textbf{255}(2023), 325--347.

\bibitem{Yom87}
\newblock Y. Yomdin,
\newblock Volume growth and entropy.
\newblock \emph{Israel J. Math.}, \textbf{57}(1987), 285--300.

\bibitem{Zan22}
\newblock Y. Zang,
\newblock Entropies and volume growth of unstable manifolds.
\newblock \emph{Ergodic Theory Dynam. Systems}, \textbf{42}(2022), 1576--1590.

\bibitem{Zan25}
\newblock Y. Zang,
\newblock Measures of maximal entropy for $C^\infty$ three-dimensional flows.
\newblock \emph{ arXiv:2503.21183v1}, (2025).


\end{thebibliography}

\vskip 5pt



\flushleft{\bf Yuntao Zang} \\
\small Soochow College,  Soochow University, Suzhou, 215006, P.R. China\\
\textit{E-mail:} \texttt{ytzang@suda.edu.cn}\\


































\end{document}














