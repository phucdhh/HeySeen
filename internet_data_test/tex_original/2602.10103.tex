\documentclass[12pt,numbers,sort&compress]{elsarticle}
\journal{}

\makeatletter
\def\ps@pprintTitle{%
 \let\@oddhead\@empty
 \let\@evenhead\@empty
 \def\@oddfoot{\hfill\thepage}%
 \let\@evenfoot\@oddfoot}
\makeatother

%% Base packages
\usepackage{amsmath,amsfonts,amssymb,amsthm,graphicx}
\providecommand{\doi}[1]{\href{https://doi.org/#1}{DOI:#1}}
\usepackage{xurl} % Loads url package with options to break everywhere
\renewcommand{\doi}[1]{%
 \href{https://doi.org/#1}{\nolinkurl{DOI:#1}}%
}

%% Author packages
\usepackage{mathtools} % for \vcentcolon in \leqdef and \reqdef
\usepackage{appendix} % to create appendices
\usepackage{xcolor} % for color in comments
\usepackage{enumerate} % for flexible enumerate labeling
\usepackage{dsfont} % for \mathds command in \ind
\usepackage{natbib} % for plainnat bibliography
\usepackage{float} % for flexible placement of figures
\usepackage{hyperref} % hyperlinks
\usepackage{subcaption} % for subfigures
\usepackage{geometry} % to control the margins
\geometry{top=1in,bottom=1.2in,left=0.8in,right=0.8in}

\theoremstyle{plain}
\newtheorem{theorem}{Theorem}
\newtheorem{proposition}[theorem]{Proposition}
\newtheorem{lemma}[theorem]{Lemma}
\newtheorem{corollary}[theorem]{Corollary}

\theoremstyle{definition}
\newtheorem{definition}{Definition}
\newtheorem{remark}{Remark}

\newcommand{\N}{\mathbb{N}}
\newcommand{\R}{\mathbb{R}}
\newcommand{\PP}{\mathsf{P}}
\newcommand{\EE}{\mathsf{E}}
\newcommand{\Var}{\mathsf{Var}}
\newcommand{\bb}[1]{\boldsymbol{#1}}
\newcommand{\rd}{\mathrm{d}}
\newcommand{\ind}{\mathds{1}}
\newcommand{\e}{\varepsilon}
\newcommand{\oo}{\mathrm{o}}
\newcommand{\OO}{\mathcal{O}}
\newcommand{\leqdef}{\vcentcolon=}
\newcommand{\reqdef}{=\vcentcolon}

\newcommand{\fred}[1]{{\color{red} #1}}
\newcommand{\Guanjie}[1]{{\color{magenta} #1}}

\begin{document}

\begin{frontmatter}

\title{Minimax properties of gamma kernel density estimators \\ under $L^p$ loss and $\beta$-H\"older smoothness of the target}

\author[a1]{Fr\'ed\'eric Ouimet}\ead{frederic.ouimet2@uqtr.ca}

\address[a1]{D\'epartement de math\'ematiques et d'informatique, Universit\'e du Qu\'ebec \`a Trois-Rivi\`eres, Trois-Rivi\`eres, Canada}

\begin{abstract}
This paper considers the asymptotic behavior in $\beta$-H\"older spaces, and under $L^p$ loss, of the gamma kernel density estimator introduced by Chen [Ann.\ Inst.\ Statist.\ Math.\ 52 (2000), 471--480] for the analysis of nonnegative data, when the target's support is assumed to be upper bounded. It is shown that this estimator can achieve the minimax rate asymptotically for a suitable choice of bandwidth whenever $(p,\beta)\in [1,3)\times(0,2]$ or $(p,\beta)\in [3,4)\times ((p-3)/(p-2),2]$. It is also shown that this estimator cannot be minimax when either $p\in [4,\infty)$ or $\beta\in (2,\infty)$.
\end{abstract}

\begin{keyword}
Density estimation \sep gamma kernel \sep \texorpdfstring{$L^p$}{Lp} loss \sep minimax estimation \sep nonnegative data \sep nonparametric estimation.
\MSC[2020]{Primary: 62G07; Secondary: 62G05 \sep 62G20}
\end{keyword}

\end{frontmatter}

\section{Introduction}\label{sec:intro}

Let $X_1,\ldots,X_n$ be independent and identically distributed (iid) random variables taking values in $[0,\infty)$ with unknown density $f$. Estimating $f$ nonparametrically is a classical problem, and kernel density estimators (KDEs) in the sense of \citet{Rosenblatt1956} and \citet{Parzen1962} remain a standard tool; see, e.g., \cite{Silverman1986,WandJones1995,ChaconDuong2018}. When the support of $f$ is constrained (for instance $[0,\infty)$ or $[0,1]$), the use of symmetric kernels in the usual KDE creates a boundary bias near the endpoints, which can significantly deteriorate global $L^p$ risks. A large literature addresses this issue using reflection and pseudodata mechanisms \citep{Schuster1985,CowlingHall1996,KarunamuniAlberts2005}, transformation approaches \citep{MarronRuppert1994}, or various boundary correction techniques \citep{GasserMuller1979,Rice1984,Muller1991,Jones1993,ChengFanMarron1997,ZhangKarunamuniJones1999}.

An alternative approach is to use \emph{asymmetric} kernels whose support matches that of the target density and whose shape adapts with the evaluation point. A prominent contribution is due to \citet{Chen1999}, who introduced \emph{beta kernel} estimators for densities supported on $[0,1]$, and to \citet{Chen2000Gamma}, who proposed a \emph{gamma kernel} estimator for densities supported on $[0,\infty)$. In addition to density estimation, \citet{Chen2000BetaReg} developed beta-kernel smoothers for regression curves, and \citet{Chen2002LocalLinear} studied local linear regression smoothers based on asymmetric kernels, including gamma-type constructions. Many extensions and refinements of gamma kernel estimators have appeared since then; see, for example, the semiparametric density estimation approach of \citet{BouezmarniRombouts2009} based on copulas, applications in econometrics \citep{BouezmarniScaillet2005,FernandesGrammig2005,BouezmarniRombouts2010,HirukawaSakudo2016,SongHouZhou2019}, and many other relevant references such as \citep{FernandesMonteiro2005,HirukawaSakudo2015,IgarashiKakizawa2018,FauziMaesono2020,IgarashiKakizawa2020,Some2020Bayesian,SomeEtAl2022CombinedGamma,FunkeHirukawa2025Uniform,FunkeHirukawa2025Splicing}.

Some of these techniques have notably been adapted to more complex supports such as the simplex \citep{AitchisonLauder1985,ChaconMateuFiguerasMartinFernandez2011,OuimetTolosanaDelgado2022,BertinGenestKlutchnikoffOuimet2023,GenestOuimet2025,Bouzebda2024,DaayebGenestKhardaniKlutchnikoffOuimet2025,DaayebKhardaniOuimet2025}, product spaces like the unit hypercube or positive orthant \citep{BouezmarniRombouts2009,BouezmarniRombouts2010b,FunkeKawka2015,Hirukawa2018,FunkeHirukawa2025Uniform}, half-spaces \citep{BelzileDesgagneGenestOuimet2025}, and the cone of positive definite matrices \citep{Ouimet2022a,BelzileGenestOuimetRichards2025}. Parallel to these specific developments, general frameworks for multivariate asymmetric kernels (called associated kernels) have been proposed and studied; see, e.g., \cite{KokonendjiSome2018,KokonendjiSome2021,AboubacarKokonendji2025,EsstafaKokonendjiNgo2025} and references therein.

While pointwise asymptotic expansions (bias, variance, limit distributions) for asymmetric kernel density estimators are by now well documented, their \emph{minimax} performance under integrated losses is more delicate. The difficulty stems from the fact that the smoothing induced by such kernels is spatially inhomogeneous, which complicates the control of global risk. In particular, depending on the support of the target density and on the loss under consideration, the variance or higher-order moments of the estimator may fail to be uniformly integrable. This lack of global moment control can prevent attainment of the classical minimax rate even when boundary bias is substantially reduced. This phenomenon was analyzed for beta kernel density estimators by~\citet{BertinKlutchnikoff2011}, who identified regimes of smoothness and loss for which minimax optimality holds or fails. Those conclusions were later extended to Dirichlet kernel density estimators on the simplex by \citet{BertinGenestKlutchnikoffOuimet2023}.


The purpose of this paper is to establish parallel minimax and non-minimax results for the gamma kernel density estimator introduced by \citet{Chen2000Gamma}. To isolate the boundary behavior at $0$ and to avoid additional tail phenomena in global $L^p$ risks, we work over H\"older-type classes of densities supported on a compact interval (throughout, on $[0,1]$, noting that any density supported on $[0,C]$ for $C\in (0,\infty)$ can be rescaled to this domain). For such classes, the benchmark minimax rate under $L^p$ loss is of order $n^{-\beta/(2\beta+1)}$; see, e.g., \citet{Tsybakov2009}. Our main results show that, with a suitable bandwidth choice, the gamma kernel density estimator achieves this rate for all $p\in[1,3)$ and $\beta\in (0,2]$, and also for a nontrivial subset of $(p,\beta)$ with $p\in [3,4)$ and $\beta\in (0,2]$. On the other hand, we prove two complementary non-minimaxity statements: the estimator cannot be minimax for any $p$ when $\beta\in (2,\infty)$, and it cannot be minimax when $p\in [4,\infty)$ even if $\beta\in (0,2]$. These conclusions mirror those obtained for beta and Dirichlet kernel density estimators.

The paper is organized as follows. Section~\ref{sec:definitions} introduces the gamma kernel density estimator, the risk criteria, the definition of $\beta$-smoothness, and other relevant notational conventions. Section~\ref{sec:main.results} states the main results, i.e., the regions of $(p,\beta)$ where we have minimaxity and non-minimaxity. Section~\ref{sec:regularity} briefly examines the regularity of mirrored gamma densities truncated to be supported on $[0,1]$ within the proposed functional classes. Section~\ref{sec:future} outlines several directions for future work. Proofs are gathered in Section~\ref{sec:proofs}, with proofs of some technical lemmas relegated to Appendix~\ref{app}.

\section{Definitions and notation}\label{sec:definitions}

Let $X_1,\ldots,X_n$ be a sequence of independent and identically distributed (iid) random variables with an unknown density $f$ supported on $[0,\infty)$. The goal is to estimate $f$ from the sample.

For a smoothing parameter (bandwidth) $b \in (0,\infty)$ and a point $x\in [0,\infty)$, define the \emph{gamma kernel}
\begin{equation}\label{eq:kernel}
K_b(x,t) \equiv K_{x/b + 1,b}(t) = \frac{t^{x/b} e^{-t/b}}{b^{x/b + 1} \Gamma(x/b + 1)} \ind_{[0,\infty)}(t), \quad t\in [0,\infty),
\end{equation}
where $\Gamma(\cdot)$ is Euler's gamma function. The associated gamma kernel density estimator is
\begin{equation}\label{eq:estimator}
\hat{f}_{n,b}(x) = \frac{1}{n} \sum_{i=1}^n K_{x/b + 1,b}(X_i), \quad x\in [0,\infty).
\end{equation}
The family $\{\hat{f}_{n,b}: b > 0\}$ is indexed by the smoothing parameter $b$, which is typically taken to depend on the sample size $n$.

Let $\xi_x$ be a gamma random variable with density $t\mapsto K_{x/b + 1,b}(t)$ on $[0,\infty)$. Then
\[
\EE[\hat{f}_{n,b}(x)] = \int_0^{\infty} K_{x/b + 1,b}(t) f(t) \rd t = \EE[f(\xi_x)].
\]
Moreover, for this parametrization, one has
\[
\EE(\xi_x) = x + b, \quad \Var(\xi_x) = x b + b^2.
\]
Hence, the smoothing induced by $\hat{f}_{n,b}$ is \emph{spatially inhomogeneous}: the typical fluctuation scale of $\xi_x$ around $x$ is of order $\sqrt{xb}$ when $x/b$ is large. This feature makes the choice of the functional class (and the control of global $L^p$ risks) more delicate than in compact-support settings.

For $p\in [1,\infty)$ and a measurable function $g:[0,\infty)\to \R$, write
\[
\|g\|_p = \left(\int_0^{\infty} |g(x)|^p \rd x\right)^{1/p}.
\]
For an estimator $f_n$ of $f$, define its $L^p$ risk at $f$ by
\[
R_n(f_n,f) = \left\{\EE\big(\|f_n - f\|_p^p\big)\right\}^{1/p},
\]
whenever the expectation exists. For a class of densities $\mathcal{F}$, define the maximal risk
\[
R_n(f_n,\mathcal{F}) = \sup_{f\in \mathcal{F}} R_n(f_n,f),
\]
and the minimax risk
\[
r_n(\mathcal{F}) = \inf_{f_n} R_n(f_n,\mathcal{F}),
\]
where the infimum is over all estimators based on $(X_1,\ldots,X_n)$.

For $\beta\in (0,\infty)$, let
\[
m = \sup\{\ell\in \N_0 : \ell < \beta\},
\]
and define the $\beta$-H\"older class $\Sigma(\beta,L)$ as the set of all densities $f$ supported on $[0,1]$ that are $m$-times differentiable on $(0,\infty)$ (so a jump discontinuity is allowed at $0$ but not at $1$) and such that
\[
\max_{0 \leq k \leq m} \sup_{u\in (0,\infty)}|f^{(k)}(u)| \leq L \quad \text{and} \quad
\sup_{\substack{u,v\in (0,\infty) \\ u\neq v}} \frac{|f^{(m)}(u) - f^{(m)}(v)|}{|u-v|^{\beta-m}} \leq L,
\]
where $f^{(k)}$ denotes the $k$th derivative of $f$ (with $f^{(0)} = f$).

\begin{remark}
Here, the definition of $\Sigma(\beta,L)$ is stated for densities supported on $[0,1]$ for simplicity. Indeed, a new space, say $\widetilde{\Sigma}(\beta,L)$, could be defined analogously for densities supported on $[0,C]$ for $C\in (0,\infty)$. All results in the paper would still be valid, given that $\tilde{f}\in \widetilde{\Sigma}(\beta,\tilde{L})$ if and only if $\tilde{f}(\cdot) = C f(C \, \cdot)$ for some $f\in \Sigma(\beta,L)$.
\end{remark}

The main question is whether the family $\{\hat{f}_{n,b}: b > 0\}$ can achieve the minimax rate over $\Sigma(\beta,L)$ under $L^p$ loss, for an appropriate choice of $b = b_n$.

Throughout the paper, expectation is taken with respect to the joint law of the mutually independent copies $X_1,\ldots,X_n$ of $X$. The notation $u = \OO(v)$ means that $\limsup |u / v| < B < \infty$ as $n \to \infty$ or $b \to 0$, depending on the context. The positive constant $B$ may depend on the risk exponent $p$, the smoothness parameter $\beta$, the Lipschitz constant $L$, and the target density $f$, but no other variable unless explicitly written as a subscript. Similarly, throughout the proofs, $c,C\in (0,\infty)$ denote generic positive constants whose value may change from expression to expression and which may depend on $p$, $\beta$, $L$, and $f$, but not on $n$,~$b$. If both $u = \OO(v)$ and $v = \OO(u)$ hold, then one writes $u \asymp v$. Similarly, the notation $u = \oo(v)$ means that $\lim |u / v| = 0$ as $n\to \infty$ or $b\to 0$. Subscripts indicate which parameters the convergence rate can depend on. If $f_n$ is any estimator of $f$, then $\EE[f_n]$ is a shorthand for the map $x\mapsto \EE[f_n(x)]$. The gamma distribution always has the shape/scale parametrization.

\section{Main results}\label{sec:main.results}

For one-dimensional density estimation under $\beta$-H\"older smoothness assumptions, the minimax rate under $L^p$ loss is
\begin{equation}\label{eq:minimax.rate}
r_n\{\Sigma(\beta,L)\} \asymp n^{-\beta/(2\beta + 1)};
\end{equation}
see, e.g., \citet[Theorem~5.1]{IbragimovHasminskii1981} and \citet[Theorem~2.8]{Tsybakov2009}.

The following theorem states a minimax property of the gamma kernel density estimator \eqref{eq:estimator} when the bandwidth is tuned according to the smoothness parameter $\beta$.

\begin{theorem}\label{thm:minimax}
Let $L > 0$ be given. Define
\[
\mathcal{S} = \left\{(p,\beta)\in [3,4)\times(0,2] : \frac{p-3}{p-2} < \beta \leq 2\right\}.
\]
Assume that $(p,\beta)\in [1,3)\times(0,2]$ or that $(p,\beta)\in \mathcal{S}$. Let $b_n = c \, n^{-2/(2\beta + 1)}$ for all $n\in \N$ and some constant $c\in (0,\infty)$. Then
\[
\limsup_{n\to \infty} \frac{R_n\{\hat{f}_{n,b_n}, \Sigma(\beta,L)\}}{r_n\{\Sigma(\beta,L)\}} < \infty,
\]
i.e., the sequence $\{\hat{f}_{n,b_n}: n\in \N\}$ achieves the minimax rate over $\Sigma(\beta,L)$ under $L^p$ loss.
\end{theorem}

\begin{remark}
The somewhat restrictive definition of $\Sigma(\beta,L)$ (smoothness on $(0,\infty)$ even though $\mathrm{supp}(f)\subseteq[0,1]$) implicitly enforces a \emph{compatibility condition at the upper endpoint} $x=1$: since $f(x)=0$ for $x>1$ and the derivatives are assumed to exist and be bounded on $(0,\infty)$, one necessarily has $f(1)=0$ and, when $\beta>1$, also $f'(1)=0$, etc. This prevents an additional ``endpoint leakage'' bias coming from the fact that the gamma kernel has support on $[0,\infty)$ and is not truncated at~$1$.

To see what would happen without this restriction, imagine working instead with the more standard H\"older-type class of densities supported on $[0,1]$ that are $\beta$-H\"older on $(0,1)$ but with \emph{no} matching condition at $x=1$ (so that, for instance, $f(1^-)$ may be strictly positive). In Step~4 of the proof of Theorem~\ref{thm:minimax}, the bound $|\EE[f(\xi_x)]-f(x)| = \OO(b^{\beta/2})$ is obtained by controlling $|f(\xi_x)-f(x)|$ through H\"older regularity. If $f$ is allowed to have a nonzero left limit at $1$, then for $x$ close to $1$, we must also account for the event $\{\xi_x>1\}$, on which $f(\xi_x)=0$ while $f(x)$ can be of order one. A simple decomposition is
\[
|\EE[f(\xi_x)] - f(x)| \leq \EE\left(|f(\xi_x) - f(x)| \ind_{\{\xi_x\leq 1\}}\right) + f(x) \, \PP(\xi_x > 1).
\]
The first term behaves as before (of order $b^{\beta/2}$), but the second term can be much larger: when $x$ lies within the kernel's typical fluctuation scale of the endpoint, i.e. $|1-x| = \OO(b^{1/2})$, the tail probability $\PP(\xi_x>1)$ is not small (it is typically of constant order), so the pointwise bias can be $\asymp 1$ on an interval of length $\asymp b^{1/2}$ around $1$ (on both sides of~$1$ when we integrate over $[0,\infty)$). This yields an additional integrated bias contribution of order
\[
\left(\int_{1 - c \, b^{1/2}}^{1 + c \, b^{1/2}} |\EE[\hat f_{n,b}(x)] - f(x)|^p \, \rd x\right)^{1/p} \asymp b^{1/(2p)}.
\]
Heuristically, the maximal risk bound in the proof of Theorem~\ref{thm:minimax} would therefore become
\[
R_n\{\hat f_{n,b},\Sigma(\beta,L)\} \leq C \Big(n^{-1/2} b^{-1/4} + b^{\beta/2} + b^{1/(2p)}\Big),
\]
where the extra term $b^{1/(2p)}$ comes from the lack of regularity at $x=1$.

If $\beta\leq 1/p$, then $b^{\beta/2}$ dominates $b^{1/(2p)}$ as $b\to 0$, so this endpoint effect is negligible and the choice $b\asymp n^{-2/(2\beta+1)}$ still leads to the usual rate $n^{-\beta/(2\beta+1)}$. In contrast, if $\beta > 1/p$, then $b^{1/(2p)}$ dominates $b^{\beta/2}$, and the best achievable rate for the gamma kernel density estimator comes from balancing $n^{-1/2} b^{-1/4}$ with $b^{1/(2p)}$, which gives $b\asymp n^{-2p/(p+2)}$ and a resulting rate $n^{-1/(p+2)}$ (the rate corresponding to an ``effective'' smoothness $1/p$). This explains why imposing a smooth matching at $x=1$ is important if one wants to recover the faster minimax rate for smoother targets.
\end{remark}

The following results identify regimes of the loss exponent $p$ and smoothness parameter $\beta$ for which the gamma kernel density estimator $\hat{f}_{n,b}$ defined in~\eqref{eq:estimator} cannot achieve the minimax rate over the $\beta$-H\"older class $\Sigma(\beta,L)$, irrespective of the choice of bandwidth sequence $(b_n)_{n\in \N}$.

\begin{proposition}\label{prop:2}
Let $p\in [1,\infty)$ and $\beta\in (2,\infty)$ be given. There exists $L > 1$ such that for every bandwidth sequence $(b_n)_{n\in \N}\subseteq (0,1)$, the family $\{\hat{f}_{n,b_n}:n\in \N\}$ satisfies
\[
\liminf_{n\to \infty}\frac{R_n\{\hat{f}_{n,b_n},\Sigma(\beta,L)\}}{r_n\{\Sigma(\beta,L)\}} = + \infty.
\]
\end{proposition}

\begin{proposition}\label{prop:3}
Let $p\in [4,\infty)$ and $\beta\in (0,2]$ be given. There exists $L > 1$ such that for every bandwidth sequence $(b_n)_{n\in \N}\subseteq (0,1)$, the family $\{\hat{f}_{n,b_n}:n\in \N\}$ satisfies
\[
\liminf_{n\to \infty} \frac{R_n\{\hat{f}_{n,b_n},\Sigma(\beta,L)\}}{r_n\{\Sigma(\beta,L)\}} = + \infty.
\]
\end{proposition}

\section{Regularity of mirrored gamma densities truncated to be supported on \texorpdfstring{$[0,1]$}{[0,1]}}\label{sec:regularity}

The critical technical condition under which Theorem~\ref{thm:minimax} is established is the assumption that the underlying univariate density $f$ belongs to the $\beta$-H\"older space $\Sigma(\beta,L)$ for appropriate choices of smoothness parameter $\beta\in(0,2]$ and Lipschitz constant $L\in(0,\infty)$. In view of the definition of $\Sigma(\beta,L)$, this entails not only compact support on $[0,1]$, but also a \emph{smooth matching at the upper endpoint} $x=1$, since $f(x)=0$ for $x>1$ and the derivatives are required to exist and be bounded on $(0,\infty)$. To provide a concrete illustration of these constraints in a familiar parametric family, we consider in this section target densities obtained by mirroring a gamma density around $1$ and truncating to~$[0,1]$.

Let $g=g_{\alpha,\theta}$ be a gamma density with shape parameter $\alpha\in(0,\infty)$ and scale parameter $\theta\in(0,\infty)$, given by
\begin{equation}\label{eq:gamma-density}
g_{\alpha,\theta}(s) = \frac{s^{\alpha - 1} e^{-s/\theta}}{\theta^{\alpha} \Gamma(\alpha)} \ind_{(0,\infty)}(s).
\end{equation}
Define the normalizing constant
\[
c_{\alpha,\theta} \leqdef \left(\int_0^1 g_{\alpha,\theta}(u) \, \rd u\right)^{-1}\in (1,\infty),
\]
and the associated \emph{mirrored gamma density truncated to $[0,1]$} by
\begin{equation}\label{eq:mirrored-gamma}
f_{\alpha,\theta}(x) \leqdef c_{\alpha,\theta} \, g_{\alpha,\theta}(1-x)\ind_{[0,1]}(x), \qquad x\in [0,\infty);
\end{equation}
see Figure~\ref{fig:mirrortruncgamma}.

\begin{figure}[!ht]
\centering
\begin{subfigure}[t]{0.49\textwidth}
\centering
\includegraphics[width=\textwidth]{graph/mirrortruncgamma.pdf}
\end{subfigure}
\hfill
\begin{subfigure}[t]{0.49\textwidth}
\centering
\includegraphics[width=\textwidth]{graph/mirrortruncgamma_zoom.pdf}
\end{subfigure}
\caption{Visualization of the mirrored gamma densities truncated to $[0,1]$ defined in \eqref{eq:mirrored-gamma} for $\theta=0.2$ and various values of the shape parameter $\alpha$ (from $2.0$ to $6.0$ with $0.2$ increments). The left panel shows the full range $x\in[0,1]$, while the right panel zooms in on $x\in[0.9,1.1]$.}
\label{fig:mirrortruncgamma}
\end{figure}

By construction, $f_{\alpha,\theta}$ is supported on $[0,1]$ and integrates to one. Moreover, since $g_{\alpha,\theta}$ is $C^{\infty}$ on $(0,\infty)$, it follows that $f_{\alpha,\theta}$ is $C^{\infty}$ on $(0,1)$. For every $k\in\N_0$ and every $x\in (0,1)$,
\begin{equation}\label{eq:mirrored-derivatives}
\smash{f_{\alpha,\theta}^{(k)}}(x) = c_{\alpha,\theta}(-1)^k g_{\alpha,\theta}^{(k)}(1-x).
\end{equation}
On the other hand, $f_{\alpha,\theta}(x)=0$ for $x>1$, and hence $\smash{f_{\alpha,\theta}^{(k)}}(x)=0$ for all $k\in\N_0$ and all $x>1$. Therefore, the only potential obstruction to $f_{\alpha,\theta}\in\Sigma(\beta,L)$ arises at the endpoint $x=1$, and it is governed by the behavior of $\smash{g_{\alpha,\theta}^{(k)}}(s)$ as $s\downarrow 0$.

To describe this behavior, note that as $s\downarrow 0$,
\[
g_{\alpha,\theta}(s) = \frac{s^{\alpha-1}}{\theta^{\alpha}\Gamma(\alpha)}\left\{1 + \OO(s)\right\}.
\]
Differentiating $s^{\alpha-1}e^{-s/\theta}$ repeatedly shows that for each fixed $k\in\N_0$, there exists a function $h_{\alpha,\theta,k}$ that is bounded on $(0,1]$ (and depends only on $\alpha,\theta,k$) such that
\begin{equation}\label{eq:gamma-derivative-factor}
g_{\alpha,\theta}^{(k)}(s) = s^{\alpha-1-k} \, h_{\alpha,\theta,k}(s), \qquad s\in (0,1].
\end{equation}
Combining \eqref{eq:mirrored-derivatives} and \eqref{eq:gamma-derivative-factor} yields that, for every $k\in\N_0$,
\begin{equation}\label{eq:mirrored-endpoint-rate}
\smash{f_{\alpha,\theta}^{(k)}}(x) = \OO\bigl((1-x)^{\alpha-1-k}\bigr), \qquad x\uparrow 1.
\end{equation}
In particular, if $\alpha>k+1$, then $\smash{f_{\alpha,\theta}^{(k)}}(x)\to 0$ as $x\uparrow 1$, which ensures the matching condition $\smash{f_{\alpha,\theta}^{(k)}}(1)=0$ required for $k$th differentiability at $x=1$ (given that $\smash{f_{\alpha,\theta}^{(k)}}(x)\equiv 0$ for $x>1$). Conversely, if $\alpha\leq k+1$, then $\smash{f_{\alpha,\theta}^{(k)}}$ fails to vanish at $x=1$ (and may even diverge), preventing membership in $\Sigma(\beta,L)$ once derivatives up to order $k$ are required.

These observations lead to a simple characterization of the largest H\"older smoothness index that can be accommodated by the family $\{f_{\alpha,\theta}:\alpha,\theta>0\}$.

\begin{proposition}\label{prop:regularity}
Let $g_{\alpha,\theta}$ be the gamma density \eqref{eq:gamma-density} and let $f_{\alpha,\theta}$ be its mirrored-and-truncated version \eqref{eq:mirrored-gamma}.
\begin{itemize}
\item[{\rm (i)}]
If $\alpha>1$, then for every $\beta\in (0,\alpha-1]$ there exists $L\in(0,\infty)$ (depending on $\alpha,\theta,\beta$) such that $f_{\alpha,\theta}\in \Sigma(\beta,L)$.

\item[{\rm (ii)}]
If $\beta>\alpha-1$, then for every $L\in(0,\infty)$ one has $f_{\alpha,\theta}\notin \Sigma(\beta,L)$.
\end{itemize}
\end{proposition}

The proof is straightforward from \eqref{eq:mirrored-endpoint-rate}. For~(i), let $\beta\in(0,\alpha-1]$ and let $m=\sup\{\ell\in\N_0:\ell<\beta\}$. Then $\beta-m\in(0,1]$ and $\alpha-1-m\geq \beta-m>0$, so \eqref{eq:mirrored-endpoint-rate} implies that $\smash{f_{\alpha,\theta}^{(k)}}(x)\to 0$ as $x\uparrow 1$ for all $0\leq k\leq m$, and that $\smash{f_{\alpha,\theta}^{(m)}}(x)=\OO((1-x)^{\alpha-1-m})$ near~$1$. Since the map $t\mapsto t^{\alpha-1-m}$ is $(\beta-m)$-H\"older on $[0,1]$ whenever $\alpha-1-m\geq \beta-m$, it follows that $\smash{f_{\alpha,\theta}^{(m)}}$ is $(\beta-m)$-H\"older in a neighborhood of~$1$; away from~$1$, the function is smooth and hence (locally) Lipschitz, which is stronger than $(\beta-m)$-H\"older because $\beta-m\leq 1$. This yields $f_{\alpha,\theta}\in\Sigma(\beta,L)$ for some finite~$L$ after taking suprema over the compact support. For~(ii), if $\alpha\leq m+1$ (with $m$ associated to $\beta$) then $f_{\alpha,\theta}$ fails to be $m$-times differentiable at $x=1$ by \eqref{eq:mirrored-endpoint-rate}. If instead $\alpha>m+1$, then $\smash{f_{\alpha,\theta}^{(m)}}(1)=0$ but, writing $x=1-h$ with $h\downarrow 0$, \eqref{eq:mirrored-endpoint-rate} gives
\[
|f_{\alpha,\theta}^{(m)}(1-h) - f_{\alpha,\theta}^{(m)}(1)| \asymp h^{\alpha - 1 - m}
\quad\Longrightarrow\quad
\frac{|f_{\alpha,\theta}^{(m)}(1-h) - f_{\alpha,\theta}^{(m)}(1)|}{h^{\beta - m}} \asymp h^{\alpha - 1 - \beta},
\]
so the H\"older quotient with exponent $\beta-m$ diverges whenever $\beta>\alpha-1$.

Finally, note that the inclusion property
\begin{equation}\label{eq:inclusion}
f \in \Sigma(\beta,L) \quad \Longrightarrow \quad \forall_{\beta^{\star}\in(0,\beta)} \exists_{L^{\star}\in(0,\infty)} \quad f \in \Sigma(\beta^{\star},L^{\star})
\end{equation}
holds in full generality. In particular, if one believes that the target density has smoothness $\beta>2$, then Theorem~\ref{thm:minimax} can still be invoked by working with any $\beta^{\star}\in(0,2]$ for which the target also belongs to $\Sigma(\beta^{\star},L^{\star})$, at the expense of using the slower minimax rate $n^{-\beta^{\star}/(2\beta^{\star}+1)}$ associated with the larger class $\Sigma(\beta^{\star},L^{\star})$.

\section{Future research directions}\label{sec:future}

The present work treats the iid setting and assumes that the smoothness index $\beta$ (hence the bandwidth order) is known. A natural continuation is to devise \emph{fully data-driven} choices of $b$ that remain rate-optimal for the global $L^p$ risk, and to clarify how the spatially inhomogeneous smoothing of gamma kernels impacts oracle-type inequalities and adaptation. Related Goldenshluger--Lepski-type selection ideas and oracle inequalities have been developed for local polynomial density estimation on complicated supports, including domains with local pinches; see, e.g., \citet{Bertin2025}. Closely related is the problem of relaxing the compact-support/matching assumption at the upper endpoint by studying truncated or renormalized gamma kernels that could be minimax over standard H\"older-type classes on bounded intervals without compatibility constraints at the upper endpoint.

Another direction is to extend the minimax analysis to other asymmetric-kernel smoothing problems, such as nonparametric regression or conditional density estimation with nonnegative responses, building on the gamma/beta kernel regression and local-linear constructions of \citet{Chen2000BetaReg,Chen2002LocalLinear} (see also \citealp{BouezmarniScaillet2005}). It would also be valuable to understand to what extent the regions of $(p,\beta)$ identified here persist when the sample is \emph{dependent} (e.g., stationary strongly mixing sequences), where the stochastic term must reflect long-run variance contributions and require dependence-adapted moment inequalities.

Finally, extending global $L^p$ minimax results to higher-dimensional associated kernels (see, e.g., \citealp{KokonendjiSome2021}) and to matrix supports remains largely open. In particular, obtaining analogues of Theorem~\ref{thm:minimax} for the Wishart kernel density estimator on the cone of symmetric positive definite matrices \citep{BelzileGenestOuimetRichards2025} appears promising, but the underlying geometry and boundary structure make the control of bias and integrated moments substantially more challenging.

\section{Proofs}\label{sec:proofs}

\subsection{Proof of Theorem~\ref{thm:minimax}}\label{sec:proof.thm.1}

\noindent\textbf{Step 1: Risk decomposition.}
Fix $n\in \N$, $b\in (0,1]$ and $f\in \Sigma(\beta,L)$. Decompose
\[
\hat{f}_{n,b} - f = (\hat{f}_{n,b} - \EE[\hat{f}_{n,b}]) + (\EE[\hat{f}_{n,b}] - f).
\]
By applying the triangle inequality for the $L^p([0,\infty))$ and $L^p(\Omega)$ norms, respectively, we have
\begin{equation}\label{eq:decomp}
\begin{aligned}
\big\{\EE(\|\hat{f}_{n,b} - f\|_p^p)\big\}^{1/p}
&\leq \big\{\EE\big[\big(\|\hat{f}_{n,b} - \EE[\hat{f}_{n,b}]\|_p + \|\EE[\hat{f}_{n,b}] - f\|_p\big)^p\big]\big\}^{1/p} \\
&\leq \big\{\EE(\|\hat{f}_{n,b} - \EE[\hat{f}_{n,b}]\|_p^p)\big\}^{1/p} + \|\EE[\hat{f}_{n,b}] - f\|_p \\
&\equiv A_n(b,f) + B_n(b,f).
\end{aligned}
\end{equation}
Taking suprema over $f\in \Sigma(\beta,L)$ yields
\[
R_n\{\hat{f}_{n,b},\Sigma(\beta,L)\} \leq \sup_{f\in \Sigma(\beta,L)} A_n(b,f) + \sup_{f\in \Sigma(\beta,L)} B_n(b,f).
\]

\bigskip
\noindent\textbf{Step 2: Kernel bounds.}
For the next steps, we register bounds below on the kernel: (i) a uniform bound; (ii) an $L^2([0,\infty))$ bound; and (iii) an exponential tail bound (valid for $x\geq 3$).
\begin{enumerate}[(a)]
\item For $x > 0$, the gamma density $t\mapsto K_b(x,t)$ is unimodal with mode at $t = x$, so $\|K_b(x,\cdot)\|_{\infty} = K_b(x,x)$. Using Stirling's lower bound
\begin{equation}\label{eq:Stirling.bound}
\Gamma(u + 1) \geq \sqrt{2\pi} \, u^{u + 1/2} e^{-u}, \quad u > 0,
\end{equation}
\citep[see, e.g.,][]{Batir2008} with $u = x/b$,
\[
K_b(x,x) = \frac{(x/b)^{x/b} e^{-x/b}}{b \, \Gamma(x/b + 1)} \leq \frac{1}{\sqrt{2\pi}} \, b^{-1/2} x^{-1/2}, \quad x > 0.
\]
Fix $x\in (0,1]$ and $b\in (0,1]$. If $x\geq b$, then $x+b\leq 2x$ and thus $x^{-1/2}\leq \sqrt{2} \, (x+b)^{-1/2}$, which implies
\[
K_b(x,x) \leq \frac{1}{\sqrt{\pi}} \, b^{-1/2}(x+b)^{-1/2}.
\]
If instead $0<x<b$, write $u=x/b\in (0,1)$. Using the integral representation $\Gamma(u+1)=\int_0^{\infty} t^{u}e^{-t}\rd t$ and restricting to $[u,u+1]$, we get
\[
\Gamma(u+1)\geq \int_u^{u+1} t^{u}e^{-t}\rd t \geq \int_u^{u+1} u^{u}e^{-(u+1)}\rd t
= u^{u}e^{-(u+1)},
\]
hence
\[
K_b(x,x)=\frac{1}{b} \, \frac{e^{-u}u^{u}}{\Gamma(u+1)}\leq \frac{e}{b}.
\]
Since $x<b$ implies $x+b\leq 2b$, we have $b^{-1}\leq \sqrt{2} \, b^{-1/2}(x+b)^{-1/2}$, and therefore
\[
K_b(x,x)\leq e\sqrt{2} \, b^{-1/2}(x+b)^{-1/2}.
\]
Combining the previous cases yields that, for all $x\in (0,1]$ and $b\in (0,1]$,
\begin{equation}\label{eq:sup-local}
\|K_b(x,\cdot)\|_{\infty} \leq C \, b^{-1/2}(x + b)^{-1/2}.
\end{equation}
In particular, $\|K_b(x,\cdot)\|_{\infty} \leq C \, b^{-1}$ for all $x\in (0,1]$.

\item A direct calculation yields
\begin{equation}\label{eq:Bb.def}
\int_0^{\infty} K_b(x,t)^2 \rd t = \frac{b^{-1} \, \Gamma(2x/b + 1)}{2^{2x/b + 1} \Gamma(x/b + 1)^2} \reqdef B_b(x).
\end{equation}
Using Stirling's formula, it follows that
\begin{equation}\label{eq:Bb}
B_b(x) \leq C \, b^{-1/2}(x + b)^{-1/2}, \quad x\in (0,1], ~b\in (0,1].
\end{equation}

\item Fix $b\in (0,1]$ and $x \geq 3$. Since the gamma density $t\mapsto K_b(x,t)$ is unimodal and the mode is at $t = x \geq 1$, it is increasing on $[0,1]$. Hence
\[
\sup_{t\in [0,1]} K_b(x,t) = K_b(x,1) = \frac{e^{-1/b}}{b^{x/b + 1} \Gamma(x/b + 1)}.
\]
Applying Stirling's lower bound $\Gamma(u+1)\geq \sqrt{2\pi} \, u^{u+1/2}e^{-u}$ with $u=x/b$ gives
\[
K_b(x,1)
\leq \frac{C}{\sqrt{x b}}\exp\left(-\frac{x\ln x - x + 1}{b}\right).
\]
Moreover, the map $x\mapsto (x\ln x - x + 1)/x = \ln x - 1 + 1/x$ is increasing for $x\geq 1$, hence for all $x\geq 3$,
\[
x\ln x - x + 1 \geq c^{\star} x,
\qquad c^{\star} \leqdef \ln 3 - 1 + \tfrac{1}{3} > 0.
\]
Therefore, there exist constants $c^{\star},C > 0$ such that
\begin{equation}\label{eq:exp-tail}
\sup_{t\in [0,1]} K_b(x,t) \leq \frac{C}{\sqrt{x b}} \exp(-c^{\star} x/b), \quad x \geq 3, ~b\in (0,1].
\end{equation}
\end{enumerate}

\bigskip
\noindent\textbf{Step 3: Control of the stochastic term $A_n(b,f)$.}
Write
\[
Z_b(x) \leqdef \hat{f}_{n,b}(x) - \EE[\hat{f}_{n,b}(x)] = \frac{1}{n} \sum_{i=1}^n \{K_b(x,X_i) - \EE[K_b(x,X_i)]\}.
\]
Since $\mathrm{supp}(f)\subseteq [0,1]$, we have $X_i\in [0,1]$ almost surely; hence
\begin{equation}\label{eq:Mb}
|K_b(x,X_i) - \EE[K_b(x,X_i)]| \leq 2 M_b(x), \quad M_b(x) \leqdef \sup_{t\in [0,1]} K_b(x,t).
\end{equation}
Moreover,
\[
\Var\bigl(K_b(x,X_1)\bigr) \leq \EE[K_b(x,X_1)^2] \leq \|f\|_{\infty} \int_0^1 K_b(x,t)^2 \rd t \leq L B_b(x),
\]
where we used $\|f\|_{\infty} \leq L$ and \eqref{eq:Bb.def}.

For $p\in [2,\infty)$, we use Rosenthal's inequality \citep[see, e.g.,][Theorem~3]{Rosenthal1970}: if $Y_1,\ldots,Y_n$ are iid centered, $|Y_1| \leq M$ almost surely and $\Var(Y_1) = \sigma^2$, then
\[
\EE\left(\left|\frac{1}{n} \sum_{i=1}^n Y_i\right|^p\right) \leq C_p\left\{\left(\frac{M}{n}\right)^{p-2}\frac{\sigma^2}{n} + \left(\frac{\sigma^2}{n}\right)^{p/2}\right\}.
\]
Applying this with $Y_i = K_b(x,X_i) - \EE[K_b(x,X_i)]$, $M = 2M_b(x)$ and $\sigma^2 = \Var(K_b(x,X_1))$ yields, for $p \geq 2$,
\begin{equation}\label{eq:pointwise-moment}
\EE(|Z_b(x)|^p) \leq C_p\left\{\left(\frac{M_b(x)}{n}\right)^{p-2}\frac{L B_b(x)}{n} + \left(\frac{L B_b(x)}{n}\right)^{p/2}\right\}.
\end{equation}
For $p\in [1,2)$, we simply use Lyapunov's inequality
\begin{equation}\label{eq:Lyapunov}
\EE(|Z_b(x)|^p) \leq (\EE[Z_b(x)^2])^{p/2} = (\Var(\hat{f}_{n,b}(x)))^{p/2} \leq \left(\frac{L B_b(x)}{n}\right)^{p/2},
\end{equation}
so \eqref{eq:pointwise-moment} remains valid with the first term dropped inside the braces.

Now, we want to integrate $\EE(|Z_b(x)|^p)$ over $x\geq 0$. We split the integral as follows:
\[
\int_0^{\infty} \EE(|Z_b(x)|^p) \rd x
= \int_0^1 \EE(|Z_b(x)|^p) \rd x
+ \int_1^3 \EE(|Z_b(x)|^p) \rd x
+ \int_3^{\infty} \EE(|Z_b(x)|^p) \rd x.
\]

\medskip
\noindent\emph{Integral over $[0,1]$.}
For $x\in (0,1]$, we have $M_b(x) \leq \|K_b(x,\cdot)\|_{\infty}$ and thus \eqref{eq:sup-local} gives $M_b(x) \leq C \, b^{-1/2}(x + b)^{-1/2}$; moreover $B_b(x) \leq C \, b^{-1/2}(x + b)^{-1/2}$ by \eqref{eq:Bb}. For $p \geq 2$ and any $q\in [0,1]$, interpolate the bounds $M_b(x) \leq C \, b^{-1}$ and $M_b(x) \leq C \, b^{-1/2}(x + b)^{-1/2}$ to obtain
\begin{equation}\label{eq:interp}
M_b(x) \leq C \, b^{-1 + q/2}(x + b)^{-q/2}, \quad x\in (0,1].
\end{equation}
Combining \eqref{eq:pointwise-moment}, \eqref{eq:Bb} and \eqref{eq:interp} gives, for $p \geq 2$,
\begin{equation}\label{eq:An-integral-01}
\begin{aligned}
&\int_0^1 \EE(|Z_b(x)|^p) \rd x \\
&\quad\leq C \left[n^{-(p-1)} b^{-\frac{(p-2)(2-q) + 1}{2}} \int_0^1 (x + b)^{-\frac{(p-2)q + 1}{2}} \rd x + n^{-p/2} b^{-p/4} \int_0^1 (x + b)^{-p/4} \rd x\right].
\end{aligned}
\end{equation}
The second integral on the right-hand side is finite uniformly in $b\in (0,1]$ provided $p < 4$. The first integral on the right-hand side is finite provided
\begin{equation}\label{eq:q-integrability}
\frac{(p-2)q + 1}{2} < 1 \quad \Longleftrightarrow \quad (p-2) q < 1.
\end{equation}

\medskip
\noindent\emph{Integral over $[1,3]$.}
For $x\in [1,3]$, the global supremum satisfies $M_b(x)\leq \|K_b(x,\cdot)\|_\infty = K_b(x,x)\leq C b^{-1/2}$ (since $x^{-1/2}\leq 1$ on $[1,3]$), and similarly $B_b(x)\leq C b^{-1/2}$ (e.g., from \eqref{eq:Bb.def} and Stirling's formula). Plugging these bounds in \eqref{eq:pointwise-moment} yields, for $p\geq 2$,
\[
\EE(|Z_b(x)|^p) \leq C_p\Big\{n^{-(p-1)} b^{-(p-1)/2} + n^{-p/2} b^{-p/4}\Big\},
\quad x\in[1,3],
\]
and for $p\in[1,2)$, \eqref{eq:Lyapunov} gives $\EE(|Z_b(x)|^p)\leq C n^{-p/2}b^{-p/4}$. Hence, for $p\geq 2$,
\begin{equation}\label{eq:integral.1.3}
\int_1^3 \EE(|Z_b(x)|^p) \, \rd x
\leq C\Big\{n^{-(p-1)} b^{-(p-1)/2} \ind_{[2,\infty)}(p) + n^{-p/2} b^{-p/4}\Big\},
\end{equation}

\medskip
\noindent\emph{Tail integral over $[3,\infty)$.}
For the tail part $x \geq 3$, we use \eqref{eq:exp-tail} and \eqref{eq:Mb}: since $|Z_b(x)| \leq 2M_b(x)$ almost surely,
\begin{equation}\label{eq:integral.3.inf}
\begin{aligned}
\int_3^{\infty} \EE(|Z_b(x)|^p) \rd x
&\leq \int_3^{\infty} (2M_b(x))^p \rd x \\
&\leq C \, b^{-p/2} \int_3^{\infty} x^{-p/2} \exp(-c^{\star} p x/b) \, \rd x \\
&\leq C \, b^{1-p/2}\exp(-3c^{\star} p/b),
\end{aligned}
\end{equation}
which is negligible compared to any power of $b$ as $b\to 0$.

\medskip
\noindent\emph{Conclusion for $A_n(b,f)$.}
If $p\in [1,2)$, then only the variance-type term contributes and we obtain
\[
\int_0^{\infty} \EE(|Z_b(x)|^p) \rd x \leq C \, n^{-p/2}b^{-p/4}.
\]
It follows that, for $p\in [1,2)$,
\begin{equation}\label{eq:A.n.1}
A_n(b,f) \leq C \, n^{-1/2}b^{-1/4}, \quad \text{for all } f\in \Sigma(\beta,L).
\end{equation}

If $p\in [2,3)$, take $q = 1$ in \eqref{eq:An-integral-01}; then \eqref{eq:q-integrability} holds and both integrals are finite on the right-hand side of \eqref{eq:An-integral-01}. Combining the bounds on $[0,1]$, $[1,3]$ and $[3,\infty)$ yields
\[
\int_0^{\infty} \EE(|Z_b(x)|^p) \rd x
\leq C\Big\{n^{-(p-1)} b^{-(p-1)/2} + n^{-p/2}b^{-p/4}\Big\}.
\]
For the bandwidth choice $b = b_n = c \, n^{-2/(2\beta + 1)}$ in Step~5 below, one has $n b_n^{1/2}\to\infty$, and therefore
\[
\frac{n^{-(p-1)} b_n^{-(p-1)/2}}{n^{-p/2} b_n^{-p/4}}
= (n b_n^{1/2})^{-(p-2)/2} \longrightarrow 0
\quad \text{if }p>2,
\]
(and equals $1$ if $p=2$). Hence, for $p\in [2,3)$,
\begin{equation}\label{eq:A.n.2}
A_n(b_n,f) \leq C \, n^{-1/2}b_n^{-1/4}, \quad \text{for all } f\in \Sigma(\beta,L).
\end{equation}

Now let $p\in [3,4)$ and assume $(p,\beta)\in \mathcal{S}$, i.e., $\beta > (p-3)/(p-2)$. Choose $q$ such that
\[
1 - \beta < q < \frac{1}{p-2},
\]
which is possible precisely because $\beta > (p-3)/(p-2)$. With this choice, the condition in \eqref{eq:q-integrability} holds. Moreover, for $b = b_n = c \, n^{-2/(2\beta + 1)}$, we have
\[
n^{-(p-1)} b_n^{-\frac{(p-2)(2-q) + 1}{2}}
\asymp \bigl(n^{-p/2}b_n^{-p/4}\bigr) \, n^{1 - p/2} b_n^{\frac{(p-2)(2q-3)}{4}}
\asymp \bigl(n^{-p/2}b_n^{-p/4}\bigr) \, n^{1 - p/2 + \frac{(p-2)(3-2q)}{2(2\beta + 1)}}.
\]
The exponent of $n$ in the last display is
\[
1 - \frac{p}{2} + \frac{(p-2)(3-2q)}{2(2\beta + 1)} \leq 1 - \frac{p}{2} + \frac{(p-2)(1 + 2\beta)}{2(2\beta + 1)} = 0,
\]
where we used $q > 1 - \beta$, i.e., $3-2q < 1 + 2\beta$. Hence the first term on the right-hand side of \eqref{eq:An-integral-01} is dominated by the second, uniformly in $n$. Together with the bounds \eqref{eq:integral.1.3} and \eqref{eq:integral.3.inf}, we get, for $p\in [3,4)$,
\begin{equation}\label{eq:A.n.3}
A_n(b_n,f) \leq C \, n^{-1/2}b_n^{-1/4}, \quad \text{for all } f\in \Sigma(\beta,L).
\end{equation}
Combining \eqref{eq:A.n.1}, \eqref{eq:A.n.2}, and \eqref{eq:A.n.3}, and taking the supremum over $f\in \Sigma(\beta,L)$ yields, for all $p\in [1,4)$,
\begin{equation}\label{eq:An - final}
\sup_{f\in \Sigma(\beta,L)} A_n(b_n,f) \leq C \, n^{-1/2}b_n^{-1/4}.
\end{equation}

\bigskip
\noindent\textbf{Step 4: Control of the bias term $B_n(b,f)$.}
Recall that, for each $x > 0$,
\[
\EE[\hat{f}_{n,b}(x)] = \int_0^1 K_b(x,t) f(t) \rd t = \EE[f(\xi_x)],
\]
where $\xi_x\sim \mathrm{Gamma}(x/b + 1,b)$. Hence $B_n(b,f) = \| \EE f(\xi_\cdot) - f(\cdot)\|_p$.

\medskip
\noindent\emph{Bias on $(0,3]$.}
If $\beta\in (0,1]$, then $|f(t) - f(x)| \leq L|t-x|^{\beta}$ and therefore
\[
|\EE[f(\xi_x)] - f(x)| \leq L \, \EE(|\xi_x - x|^{\beta}).
\]
If $\beta\in (1,2]$, then by Taylor's theorem with remainder and the definition of $\Sigma(\beta,L)$,
\[
f(\xi_x) - f(x) = f'(x)(\xi_x - x) + R_x, \quad |R_x| \leq L |\xi_x - x|^{\beta},
\]
so, using $|f'(x)| \leq L$,
\[
|\EE[f(\xi_x)] - f(x)| \leq L |\EE(\xi_x - x)| + L \, \EE(|\xi_x - x|^{\beta}) = L \, b + L \, \EE(|\xi_x - x|^{\beta}).
\]
In both cases, it remains to bound $\EE(|\xi_x - x|^{\beta})$. Since $\EE(\xi_x) = x + b$ and
\[
\Var(\xi_x) = x b + b^2 \leq 4 b, \quad x\in (0,3], ~b\in (0,1],
\]
Lyapunov's inequality gives $\EE(|\xi_x - \EE\xi_x|^{\beta}) \leq C \, b^{\beta/2}$. Hence, by the triangle inequality,
\[
\EE(|\xi_x - x|^{\beta}) \leq C \, \left(\EE(|\xi_x - \EE(\xi_x)|^{\beta}) + b^{\beta}\right) \leq C \, b^{\beta/2}.
\]
Therefore, for all $\beta\in (0,2]$ and all $x\in (0,3]$,
\[
|\EE[f(\xi_x)] - f(x)| \leq C \, b^{\beta/2},
\]
and consequently
\begin{equation}\label{eq:bias-01}
\int_0^3 |\EE[f(\xi_x)] - f(x)|^p \rd x \leq C \, b^{p\beta/2}.
\end{equation}

\medskip
\noindent\emph{Bias on $[3,\infty)$.}
For $x\geq 3$, $f(x)=0$ and
\[
0 \leq \EE[\hat{f}_{n,b}(x)] = \int_0^1 K_b(x,t)f(t) \rd t \leq \|f\|_{\infty} \int_0^1 K_b(x,t) \rd t \leq L M_b(x).
\]
Using \eqref{eq:exp-tail} and integrating yields
\[
\int_3^{\infty} |\EE[\hat{f}_{n,b}(x)]|^p \rd x
\leq C \, b^{-p/2}\int_3^{\infty} x^{-p/2} e^{-c^{\star} p x/b} \, \rd x
\leq C \, b^{1-p/2}e^{-3c^{\star} p/b},
\]
which is negligible compared to any power of $b$.

Combining the two regions $[0,3]$ and $[3,\infty)$ with \eqref{eq:bias-01} yields
\begin{equation}\label{eq:Bn - final}
\sup_{f\in \Sigma(\beta,L)} B_n(b,f) \leq C \, b^{\beta/2}.
\end{equation}

\bigskip
\noindent\textbf{Step 5: Choice of $b_n$ and conclusion.}
From \eqref{eq:decomp}, \eqref{eq:An - final} and \eqref{eq:Bn - final}, for the choice $b = b_n = c \, n^{-2/(2\beta + 1)}$, we obtain
\[
R_n\{\hat{f}_{n,b_n},\Sigma(\beta,L)\} \leq C (n^{-1/2}b_n^{-1/4} + b_n^{\beta/2}) \asymp n^{-\beta/(2\beta + 1)},
\]
which is the minimax rate stated in \eqref{eq:minimax.rate}. Therefore,
\[
\limsup_{n\to \infty} \frac{R_n\{\hat{f}_{n,b_n}, \Sigma(\beta,L)\}}{r_n\{\Sigma(\beta,L)\}} < \infty.
\]
This completes the proof of Theorem~\ref{thm:minimax}.

\subsection{Proof of Proposition~\ref{prop:2}}\label{sec:proof.prop.2}

For a density $f$ supported on $[0,1]$, write $\PP_f$, $\EE_f$ and $\Var_f$ for probability, expectation and variance computed under the joint law of $(X_1,\ldots,X_n)$ with iid marginals of density $f$.

Let $L > 1$. We begin with a simple device to lower bound the $L^p([0,\infty))$ norm by an $L^1([0,1])$ norm. For any measurable function $g$,
\begin{equation}\label{eq:Lp.L1}
\|g\|_p \geq \left(\int_0^1 |g(x)|^p \rd x\right)^{1/p} \geq \int_0^1 |g(x)| \rd x \equiv \|g\|_{L^1([0,1])},
\end{equation}
so by Jensen's inequality,
\begin{equation}\label{eq:Lp-to-L1}
R_n(\hat{f}_{n,b},f) = \left\{\EE_f(\|\hat{f}_{n,b} - f\|_p^p)\right\}^{1/p} \geq \EE_f(\|\hat{f}_{n,b} - f\|_{L^1([0,1])}).
\end{equation}

We first introduce the following simple (non-smooth) test functions:
\begin{equation}\label{eq:test-densities}
f_0(x) = \ind_{[0,1]}(x), \qquad f_3(x) = \{1 + \e \cdot(2x-1)\} \ind_{[0,1]}(x),
\end{equation}
where
\[
\e = \e(L) \leqdef \min\left\{\frac{1}{2},\frac{L-1}{2}\right\}\in (0,1/2];
\]
see Figure~\ref{fig:test-densities}. Then $f_0$ and $f_3$ are nonnegative and integrate to one on $[0,1]$. However, due to the hard cutoff at $x=1$, they do not satisfy the smoothness requirements imposed by the definition of $\Sigma(\beta,L)$.

\begin{figure}[!ht]
\centering
\begin{subfigure}[t]{0.49\textwidth}
\centering
\includegraphics[width=\textwidth]{graph/f0.pdf}
\end{subfigure}
\hfill
\begin{subfigure}[t]{0.49\textwidth}
\centering
\includegraphics[width=\textwidth]{graph/f3.pdf}
\end{subfigure}
\caption{Visualization of the non-smooth test densities $f_0$ and $f_3$ defined in \eqref{eq:test-densities} with $L=2$ (hence $\e=1/2$).}
\label{fig:test-densities}
\end{figure}

Let $\widetilde{f}_0$ and $\widetilde{f}_3$ denote $C^{\infty}$ modifications of $f_0$ and $f_3$ on $[7/8,1]$ such that:
\begin{itemize}\setlength\itemsep{0em}
\item $\widetilde{f}_j = f_j$ on $[0,7/8]$ for $j\in\{0,3\}$;
\item $\widetilde{f}_j$ is supported on $[0,1]$ and $\int_0^1 \widetilde{f}_j(x) \, \rd x=1$;
\item $\widetilde{f}_j$ is $C^{\infty}$ on $(0,\infty)$ with $\widetilde{f}_j^{(k)}(1)=0$ for all $k\in\N_0$;
\item $\widetilde{f}_0,\widetilde{f}_3\in \Sigma(\beta,\widetilde{L})$ for an appropriate $\widetilde{L} > 1$.
\end{itemize}
This is a standard mollification construction and we omit the details, since all subsequent lower bounds are obtained by integrating over $x$ in subsets of $[0,1/2]$ and by using exponential tail bounds for the gamma kernel; the smoothing on $[7/8,1]$ does not affect the conclusions, up to exponentially small error terms that are absorbed in the constants.

The next lemma provides the two lower bounds (bias and stochastic fluctuation) needed to prove Proposition~\ref{prop:2}.

\begin{lemma}\label{lem:prop2-key}
Let $\widetilde{f}_0$ and $\widetilde{f}_3$ be as above. There exist constants $b_0\in (0,1)$ and $c_0,c_1,c_2\in (0,\infty)$ (depending at most on $\widetilde{L}$) such that the following statements hold for all $b\in (0,b_0]$.
\begin{enumerate}[(a)]
\item \textbf{Bias lower bound at $\widetilde{f}_3$.} One has
\[
\|\EE_{\widetilde{f}_3}[\hat{f}_{n,b}] - \widetilde{f}_3\|_{L^1([0,1])} \geq c_1 b.
\]
\item \textbf{Fluctuation lower bound at $\widetilde{f}_0$.} If in addition $n b^{1/2} \geq c_0$, then
\[
\EE_{\widetilde{f}_0}\left[\int_{1/4}^{1/2}\big|\hat{f}_{n,b}(x) - \EE_{\widetilde{f}_0}[\hat{f}_{n,b}(x)]\big| \rd x\right] \geq c_2 n^{-1/2}b^{-1/4}.
\]
Moreover,
\[
\sup_{x\in [1/4,1/2]}\big|\EE_{\widetilde{f}_0}[\hat{f}_{n,b}(x)] - \widetilde{f}_0(x)\big| \leq \exp(-c_0/b).
\]
\end{enumerate}
\end{lemma}

\begin{proof}[Proof of Lemma~\ref{lem:prop2-key}]
\noindent\textbf{Proof of $(a)$.}
Fix $x\in [0,1/2]$ and let $\xi_x\sim\mathrm{Gamma}(x/b + 1,b)$, so that
\[
\EE_{\widetilde{f}_3}[\hat{f}_{n,b}(x)] = \int_0^1 K_b(x,t)\widetilde{f}_3(t) \, \rd t = \EE[\widetilde{f}_3(\xi_x)].
\]
Since $\widetilde{f}_3=f_3$ on $[0,7/8]$ and $x\in[0,1/2]\subseteq[0,7/8]$, we have $\widetilde{f}_3(x)=f_3(x)$. Moreover,
\begin{equation}\label{eq:tilde-f3-error}
\big|\EE[\widetilde{f}_3(\xi_x)] - \EE[f_3(\xi_x)]\big|
= \big|\EE[(\widetilde{f}_3-f_3)(\xi_x)\ind_{\{\xi_x>7/8\}}]\big|
\leq \|\widetilde{f}_3-f_3\|_{\infty} \, \PP(\xi_x>7/8).
\end{equation}
A Chernoff bound (as in \eqref{eq:tail.bound.xi.x}, with threshold $7/8$ instead of $1$) yields
\[
\PP(\xi_x>7/8)\leq C\exp(-c/b)
\]
uniformly over $x\in[0,1/2]$ for some constants $c,C>0$. Hence, the right-hand side of \eqref{eq:tilde-f3-error} is $\OO(\exp(-c/b))$ uniformly in $x\in[0,1/2]$.
Therefore it suffices to derive a lower bound for $\EE[f_3(\xi_x)]-f_3(x)$, since the above error term is negligible as $b\to 0$.

Since $f_3(t) = \{1 + \e \cdot (2t-1)\} \ind_{[0,1]}(t)$, we have
\[
\EE[f_3(\xi_x)] = \PP(\xi_x \leq 1) + \e\left\{2 \EE(\xi_x \ind_{\{\xi_x \leq 1\}}) - \PP(\xi_x \leq 1)\right\}.
\]
Hence, using $f_3(x) = 1 + \e \cdot (2x-1)$ and the identity $\EE(\xi_x) = x + b$, we obtain
\begin{align*}
\EE[f_3(\xi_x)] - f_3(x)
&= (1 - \e) \, \{\PP(\xi_x \leq 1)-1\} + 2 \e\{\EE(\xi_x \ind_{\{\xi_x \leq 1\}})-x\} \\
&= -(1 - \e) \, \PP(\xi_x > 1) + 2 \e\left\{b - \EE(\xi_x \ind_{\{\xi_x > 1\}})\right\} \\
&\geq 2 \e b -2 \e \, \EE(\xi_x \ind_{\{\xi_x > 1\}}) - \PP(\xi_x > 1).
\end{align*}
We now bound the tail probability $\PP(\xi_x > 1)$ uniformly over $x\in [0,1/2]$. For $\lambda = 1/(2 b)$ (so that $\lambda < 1/b$, which implies the finiteness of the moment generating function), Chernoff's inequality yields
\begin{equation}\label{eq:tail.bound.xi.x}
\PP(\xi_x > 1)
\leq e^{-\lambda}\EE(e^{\lambda\xi_x}) = e^{-1/(2 b)}(1 - b\lambda)^{-(x/b + 1)}
= e^{-1/(2 b)} 2^{x/b + 1} \leq 2 \exp\left(-\frac{1 - \ln 2}{2 b}\right).
\end{equation}
Next, by Cauchy--Schwarz,
\[
\EE(\xi_x \ind_{\{\xi_x > 1\}}) \leq \{\EE(\xi_x^2)\}^{1/2} \PP(\xi_x > 1)^{1/2}.
\]
Using $\EE(\xi_x^2) = \Var(\xi_x) + \{\EE(\xi_x)\}^2 = (xb + b^2) + (x + b)^2 \leq C$ for $x\in [0,1/2]$ and $b \leq 1$, we get $\EE(\xi_x \ind_{\{\xi_x > 1\}}) \leq C \, \exp(-c/b)$ for some $c > 0$. Therefore, for $b$ small enough,
\[
\EE[f_3(\xi_x)] - f_3(x) \geq \e b, \quad x\in [0,1/2].
\]
Combining this with \eqref{eq:tilde-f3-error} yields, for $b$ small enough,
\[
\EE_{\widetilde{f}_3}[\hat{f}_{n,b}(x)] - \widetilde{f}_3(x)
= \EE[\widetilde{f}_3(\xi_x)] - f_3(x)
\geq \e b - C\exp(-c/b) \geq \frac{\e}{2}b,
\quad x\in[0,1/2].
\]
Integrating over $x\in [0,1/2]$ yields
\[
\|\EE_{\widetilde{f}_3}[\hat{f}_{n,b}] - \widetilde{f}_3\|_{L^1([0,1])}
\geq \int_0^{1/2} \big|\EE_{\widetilde{f}_3}[\hat{f}_{n,b}(x)] - \widetilde{f}_3(x)\big| \rd x
\geq \frac{\e}{4} b,
\]
which is $(a)$ with $c_1 = \e/4$.

\bigskip
\noindent\textbf{Proof of $(b)$.}
Fix $x\in [1/4,1/2]$. Under $\widetilde{f}_0$, the variable $\zeta_1(x) \leqdef K_b(x,X_1)$ satisfies
\[
\EE_{\widetilde{f}_0}[\zeta_1(x)] = \int_0^1 K_b(x,t)\widetilde{f}_0(t) \, \rd t, \quad
\EE_{\widetilde{f}_0}[\zeta_1(x)^2] = \int_0^1 K_b(x,t)^2\widetilde{f}_0(t) \, \rd t.
\]
Let $Y_1(x) = \zeta_1(x) - \EE_{\widetilde{f}_0}[\zeta_1(x)]$ and set $\sigma_x^2 \leqdef \Var_{\widetilde{f}_0}\{Y_1(x)\} = \Var_{\widetilde{f}_0}\{\zeta_1(x)\}$. Also let
\[
Z_n(x) \leqdef \hat{f}_{n,b}(x) - \EE_{\widetilde{f}_0}[\hat{f}_{n,b}(x)] = \frac{1}{n} \sum_{i=1}^n Y_i(x),
\]
with iid copies $Y_i(x)$ of $Y_1(x)$.

\medskip
\noindent\emph{Lower bound on $\sigma_x^2$.}
Introduce the Stirling ratio
\begin{equation}\label{eq:Stirling.ratio}
R(u) \leqdef \frac{\sqrt{2\pi} \, e^{-u} u^{u + 1/2}}{\Gamma(u + 1)}, \quad u \geq 0.
\end{equation}
Substituting $\Gamma(u + 1) = \sqrt{2\pi} e^{-u}u^{u + 1/2}/R(u)$ into~\eqref{eq:Bb.def} with $u = x/b$ and $u = 2x/b$ yields
\begin{equation}\label{eq:Bb-R}
\int_0^{\infty} K_b(x,t)^2 \rd t \equiv B_b(x) = \frac{1}{2\sqrt{\pi}} \, b^{-1/2} x^{-1/2} \frac{R(x/b)^2}{R(2x/b)}.
\end{equation}
Since $R$ is increasing and $R(u) \leq 1$ for $u \geq 1$, for $b \leq 1/4$ we have $x/b \geq 1$ on $x\in [1/4,1/2]$ and thus $R(x/b)^2 / R(2x/b) \geq R(1)^2$. Hence, for $b \leq 1/4$ and all $x\in [1/4,1/2]$,
\[
B_b(x) \geq c \, b^{-1/2}
\]
for some constant $c > 0$.

Next, since $t\mapsto K_b(x,t)$ is decreasing on $[x,\infty)$ when $x \leq 1/2$, we have
\[
\int_{7/8}^{\infty} K_b(x,t)^2 \rd t \leq K_b(x,7/8) \int_{7/8}^{\infty} K_b(x,t) \rd t = K_b(x,7/8)\PP(\xi_x > 7/8).
\]
Both factors are $\exp(-c/b)$ uniformly for $x\in [1/4,1/2]$ by a Chernoff bound analogous to \eqref{eq:tail.bound.xi.x} (with threshold $7/8$), hence $\int_{7/8}^{\infty} K_b(x,t)^2 \rd t \leq \exp(-c/b)$. Therefore, for $b$ small enough,
\[
\int_0^{7/8} K_b(x,t)^2 \rd t
= B_b(x) - \int_{7/8}^{\infty} K_b(x,t)^2 \rd t
\geq \frac{1}{2} B_b(x) \geq c \, b^{-1/2}.
\]
Since $\widetilde{f}_0(t)=f_0(t)=1$ for $t\in[0,7/8]$, it follows that, for $b$ small enough,
\[
\EE_{\widetilde{f}_0}[\zeta_1(x)^2] = \int_0^1 K_b(x,t)^2\widetilde{f}_0(t) \, \rd t
\geq \int_0^{7/8} K_b(x,t)^2 \rd t \geq c \, b^{-1/2}.
\]
Moreover, since $\widetilde{f}_0\in\Sigma(\beta,\widetilde{L})$, we have $\|\widetilde{f}_0\|_{\infty}\leq \widetilde{L}$ and thus
\[
\big\{\EE_{\widetilde{f}_0}[\zeta_1(x)]\big\}^2 \leq \|\widetilde{f}_0\|_{\infty}^2 \leq \widetilde{L}^2,
\]
which is negligible compared to $b^{-1/2}$ as $b\to 0$. Hence, for $b$ small enough,
\begin{equation}\label{eq:sigma-lower}
\sigma_x^2 \equiv \Var_{\widetilde{f}_0}\{\zeta_1(x)\} \geq c \, b^{-1/2}, \quad x\in [1/4,1/2].
\end{equation}

\medskip
\noindent\emph{A Paley--Zygmund lower bound for $\EE_{\widetilde{f}_0}(|Z_n(x)|)$.}
Let
\[
M_x \leqdef \|Y_1(x)\|_{\infty} \leq 2 \sup_{t\in [0,1]} K_b(x,t) = 2K_b(x,x).
\]
Then $K_b(x,x) \leq C \, b^{-1/2}$ uniformly over $x\in [1/4,1/2]$ by \eqref{eq:sup-local}, so $M_x^2 \leq C \, b^{-1}$.

For the fourth moment, using independence and centering,
\[
\EE_{\widetilde{f}_0}[Z_n(x)^4]
= \frac{1}{n^4}\EE_{\widetilde{f}_0}\left[\left(\sum_{i=1}^n Y_i(x)\right)^4\right]
= \frac{1}{n^4}\left\{n \EE_{\widetilde{f}_0}\big[Y_1(x)^4\big] + 3n(n-1)\sigma_x^4\right\}.
\]
Moreover, $Y_1(x)^4 \leq M_x^2 Y_1(x)^2$, hence $\EE_{\widetilde{f}_0}[Y_1(x)^4] \leq M_x^2\sigma_x^2$. Therefore,
\[
\EE_{\widetilde{f}_0}[Z_n(x)^4] \leq \frac{M_x^2\sigma_x^2}{n^3} + \frac{3\sigma_x^4}{n^2}.
\]
Let $W = Z_n(x)^2$. Then $\EE_{\widetilde{f}_0}[W] = \EE_{\widetilde{f}_0}[Z_n(x)^2] = \sigma_x^2/n$ and $\EE_{\widetilde{f}_0}[W^2] = \EE_{\widetilde{f}_0}[Z_n(x)^4]$. By Paley--Zygmund with $\theta = 1/2$,
\begin{equation}\label{eq:to.bound.below}
\PP_{\widetilde{f}_0}\left(W \geq \frac{1}{2}\EE_{\widetilde{f}_0}[W]\right)
\geq \frac{(1 - \theta)^2(\EE_{\widetilde{f}_0}[W])^2}{\EE_{\widetilde{f}_0}[W^2]}
\geq \frac{c}{3 + M_x^2/(n\sigma_x^2)}.
\end{equation}
Using $M_x^2/(n\sigma_x^2) \leq C/(n b^{1/2})$ (from $M_x^2 \leq Cb^{-1}$ and~\eqref{eq:sigma-lower}), we find that if $n b^{1/2} \geq c_0$ then the above probability in \eqref{eq:to.bound.below} is bounded below by a positive constant. Hence, for $n b^{1/2} \geq c_0$,
\[
\EE_{\widetilde{f}_0}(|Z_n(x)|)
\geq \sqrt{\frac{1}{2}\EE_{\widetilde{f}_0}[Z_n(x)^2]} \, \PP_{\widetilde{f}_0}\left(|Z_n(x)| \geq \sqrt{\tfrac{1}{2} \EE_{\widetilde{f}_0}[Z_n(x)^2]}\right)
\geq c \, \frac{\sigma_x}{\sqrt{n}}
\geq c \, n^{-1/2}b^{-1/4},
\]
uniformly over $x\in [1/4,1/2]$, by~\eqref{eq:sigma-lower}. Integrating in $x$ over an interval of length $1/4$ proves the first inequality in $(b)$.

Finally, for $x\in [1/4,1/2]$, one has $\widetilde{f}_0(x)=f_0(x)=1$ and
\[
\EE_{\widetilde{f}_0}[\hat{f}_{n,b}(x)] = \int_0^1 K_b(x,t)\widetilde{f}_0(t) \, \rd t
= \int_0^1 K_b(x,t) \, \rd t + \int_{7/8}^1 K_b(x,t)\{\widetilde{f}_0(t)-1\} \, \rd t.
\]
Therefore,
\[
\big|\EE_{\widetilde{f}_0}[\hat{f}_{n,b}(x)] - \widetilde{f}_0(x)\big|
\leq \PP(\xi_x>1) + \|\widetilde{f}_0-1\|_{\infty} \, \PP(\xi_x>7/8)
\leq \exp(-c_0/b),
\]
for some $c_0>0$ and all $b$ small enough, using Chernoff bounds as above. This yields the last inequality in $(b)$.
\end{proof}

We now give the proof of Proposition~\ref{prop:2}. Fix $p\in [1,\infty)$, $\beta\in (2,\infty)$ and $\widetilde{L} > 1$. Let $(b_n)_{n\in \N}\subseteq (0,1)$ be arbitrary and set $b = b_n$.

By~\eqref{eq:Lp-to-L1} and the fact that $\{\widetilde{f}_0,\widetilde{f}_3\}\subseteq \Sigma(\beta,\widetilde{L})$, we have
\[
\begin{aligned}
R_n\{\hat{f}_{n,b},\Sigma(\beta,\widetilde{L})\}
&\geq \frac{1}{2}\left(R_n(\hat{f}_{n,b},\widetilde{f}_0) + R_n(\hat{f}_{n,b},\widetilde{f}_3)\right) \\
&\geq \frac{1}{2}\left(\EE_{\widetilde{f}_0}(\|\hat{f}_{n,b} - \widetilde{f}_0\|_{L^1([0,1])}) + \EE_{\widetilde{f}_3}(\|\hat{f}_{n,b} - \widetilde{f}_3\|_{L^1([0,1])})\right).
\end{aligned}
\]
Moreover, since $\|\cdot\|_{L^1([0,1])}$ is convex, Jensen's inequality yields
\[
\EE_{\widetilde{f}_3}(\|\hat{f}_{n,b} - \widetilde{f}_3\|_{L^1([0,1])}) \geq \|\EE_{\widetilde{f}_3}[\hat{f}_{n,b}] - \widetilde{f}_3\|_{L^1([0,1])},
\]
so
\begin{equation}\label{eq:sup-lower}
R_n\{\hat{f}_{n,b},\Sigma(\beta,\widetilde{L})\} \geq \frac{1}{2}\left(\EE_{\widetilde{f}_0}(\|\hat{f}_{n,b} - \widetilde{f}_0\|_{L^1([0,1])}) + \|\EE_{\widetilde{f}_3}[\hat{f}_{n,b}] - \widetilde{f}_3\|_{L^1([0,1])}\right).
\end{equation}

To handle the $\liminf$ in Proposition~\ref{prop:2}, set
\[
a_n \leqdef \frac{R_n\{\hat{f}_{n,b_n},\Sigma(\beta,\widetilde{L})\}}{r_n\{\Sigma(\beta,\widetilde{L})\}}.
\]
Let $(n_k)_{k\in\N}$ be a subsequence such that $a_{n_k}\to \liminf_{n\to\infty} a_n$. It suffices to prove that $a_{n_k}\to +\infty$. For notational simplicity, we relabel the subsequence and write $n$ for $n_k$ and $b_n$ for $b_{n_k}$ below.

We now distinguish three cases.

\medskip
\noindent\emph{Case 1: $b_n$ does not tend to $0$}
Passing to a further subsequence, we may assume $b_n\to b_\star\in (0,1]$. Then
\[
\begin{aligned}
\|\EE_{\widetilde{f}_3}[\hat{f}_{n,b_n}] - \widetilde{f}_3\|_{L^1([0,1])}
&= \left\|\int_0^1 K_{b_n}(\cdot,t)\widetilde{f}_3(t) \, \rd t - \widetilde{f}_3\right\|_{L^1([0,1])} \\
&\longrightarrow \left\|\int_0^1 K_{b_\star}(\cdot,t)\widetilde{f}_3(t) \, \rd t - \widetilde{f}_3\right\|_{L^1([0,1])},
\end{aligned}
\]
by dominated convergence. Since $b_\star>0$, the limit above is strictly positive. Hence the right-hand side of~\eqref{eq:sup-lower} is bounded away from $0$ along this subsequence, and since $r_n\{\Sigma(\beta,\widetilde{L})\}\to 0$, we obtain $a_n\to +\infty$.

\medskip
\noindent\emph{Case 2: $b_n\to 0$ but $n b_n^{1/2}$ does not tend to $+\infty$.}
Passing to a further subsequence, we may assume $s_n \leqdef n b_n^{1/2}\leq M$ for all $n$ and some $M<\infty$. Fix $x\in [1/4,1/2]$ and set
\[
\delta_n \leqdef \sqrt{C_0 \ln(1/b_n)},
\]
where $C_0>0$ is a large numerical constant to be chosen below. Consider the event
\[
A_n(x) \leqdef \bigcap_{i=1}^n \{|X_i-x|>\delta_n b_n^{1/2}\}.
\]
For $n$ large enough, the interval $\{u:|u-x|\leq \delta_n b_n^{1/2}\}$ is contained in $[0,7/8]$, where $\widetilde{f}_0\equiv 1$, so
\[
\PP_{\widetilde{f}_0}\{A_n(x)\} = \big(1-2\delta_n b_n^{1/2}\big)^n \geq \exp\{-4 \delta_n n b_n^{1/2}\} = \exp\{-4\delta_n s_n\}.
\]
On $A_n(x)$, since $u\mapsto K_{b_n}(x,u)$ is unimodal with mode at $u=x$ (increasing on $[0,x]$ and decreasing on $[x,\infty)$), we have
\[
\hat{f}_{n,b_n}(x)\leq \sup_{\{|u-x|\geq \delta_n b_n^{1/2}\}}K_{b_n}(x,u)
= \max\Big\{K_{b_n}(x,x-\delta_n b_n^{1/2}),\,K_{b_n}(x,x+\delta_n b_n^{1/2})\Big\}.
\]
Write, for any $\delta\in\R$ such that $x+\delta b^{1/2}>0$,
\[
K_b(x,x + \delta b^{1/2}) = K_b(x,x) Q_{b,\delta}(x), \quad
Q_{b,\delta}(x) \leqdef \exp\left\{\frac{x}{b}\ln\left(1 + \frac{\delta b^{1/2}}{x}\right)-\frac{\delta}{b^{1/2}}\right\},
\]
as in the proof of Lemma~\ref{lem:K-local-lower} in Appendix~\ref{app}. Since $\delta_n b_n^{1/2}\to 0$ and $\delta_n^3 \hspace{0.2mm} b_n^{1/2}\to 0$ under $b_n\to 0$, the same Taylor expansion yields $\ln\{ Q_{b_n,\pm \delta_n}(x)\} \leq -c \, \delta_n^2$ for all $n$ large enough and all $x\in[1/4,1/2]$. Using $K_{b_n}(x,x)\leq C b_n^{-1/2}$ (see \eqref{eq:sup-local} with $x\in[1/4,1/2]$), we obtain for all $n$ large enough,
\begin{equation}\label{eq:last.display}
\max\Big\{K_{b_n}(x,x-\delta_n b_n^{1/2}),\,K_{b_n}(x,x+\delta_n b_n^{1/2})\Big\}
\leq C b_n^{-1/2}\exp(-c \, \delta_n^2)
\leq C \, b_n^{c \, C_0 - 1/2}.
\end{equation}
Choosing $C_0$ large enough so that $c \, C_0 - 1/2>0$, the last display \eqref{eq:last.display} is at most $1/2$ for all $n$ large enough. Since $\widetilde{f}_0(x)=1$ on $[1/4,1/2]$, it follows that, for all $n$ large enough,
\[
\EE_{\widetilde{f}_0}\big[|\hat{f}_{n,b_n}(x)-\widetilde{f}_0(x)|\big]
\geq \frac{1}{2} \, \PP_{\widetilde{f}_0}\{A_n(x)\}
\geq \frac{1}{2} \, \exp\{-4 \delta_n s_n\}.
\]
Integrating over $x\in[1/4,1/2]$ and using~\eqref{eq:sup-lower} yields
\[
R_n\{\hat{f}_{n,b_n},\Sigma(\beta,\widetilde{L})\} \geq c \exp\{-C \, \delta_n s_n\}.
\]
Since $b_n=s_n^2/n^2$, we have $\delta_n=\sqrt{C_0\ln(1/b_n)}=\sqrt{2C_0\ln(n/s_n)}$, so
\[
R_n\{\hat{f}_{n,b_n},\Sigma(\beta,\widetilde{L})\} \geq c \, \exp\{-\widetilde{C} s_n \sqrt{\ln(n/s_n)}\}.
\]
Since $s\mapsto s\sqrt{\ln(n/s)}$ is increasing on $(0,n/\sqrt{e})$, and $s_n \leq M < n/\sqrt{e}$ for $n$ large enough, using the minimax rate in~\eqref{eq:minimax.rate} yields
\[
\frac{R_n\{\hat{f}_{n,b_n},\Sigma(\beta,\widetilde{L})\}}{r_n\{\Sigma(\beta,\widetilde{L})\}}
\geq c \, n^{\beta/(2\beta + 1)} \exp\{-\widetilde{C} M \sqrt{\ln(n/M)}\} \longrightarrow +\infty.
\]
Hence $a_n\to +\infty$ also in this subcase.

\medskip
\noindent\emph{Case 3: $b_n\to 0$ and $n b_n^{1/2}\to + \infty$.}
Then for all $n$ large enough, $b_n\in (0,b_0]$ and $n b_n^{1/2} \geq c_0$, so Lemma~\ref{lem:prop2-key} applies. From~\eqref{eq:sup-lower} and Lemma~\ref{lem:prop2-key}(a)--(b),
\[
R_n\{\hat{f}_{n,b_n},\Sigma(\beta,\widetilde{L})\} \geq c \, (b_n + n^{-1/2} b_n^{-1/4}) - \exp(-c/b_n).
\]
Since $\exp(-c/b_n) = o(b_n)$ as $b_n\to 0$, for $n$ large enough, we get
\[
R_n\{\hat{f}_{n,b_n},\Sigma(\beta,\widetilde{L})\} \geq c \, (b_n + n^{-1/2}b_n^{-1/4}).
\]
For every $n$ and every $b > 0$, writing $u = b^{1/4}$ gives
\[
b + n^{-1/2}b^{-1/4} = u^4 + \frac{1}{\sqrt{n} u} = n^{-2/5}\left((u n^{1/10})^4 + (u n^{1/10})^{-1}\right) \geq c \, n^{-2/5},
\]
because $\inf_{t > 0}(t^4 + t^{-1}) > 0$. Hence,
\begin{equation}\label{eq:rate-lower-25}
R_n\{\hat{f}_{n,b_n},\Sigma(\beta,\widetilde{L})\} \geq c \, n^{-2/5}.
\end{equation}

Since $\beta > 2$ implies $\beta/(2\beta + 1) > 2/5$, we have $n^{-2/5}/n^{-\beta/(2\beta + 1)}\to \infty$. Combining with~\eqref{eq:rate-lower-25} yields
\[
\liminf_{n\to \infty}\frac{R_n\{\hat{f}_{n,b_n},\Sigma(\beta,\widetilde{L})\}}{r_n\{\Sigma(\beta,\widetilde{L})\}}
\geq \liminf_{n\to \infty} c \, n^{\beta/(2\beta + 1)-2/5}
= +\infty.
\]
This completes the proof of Proposition~\ref{prop:2}.

\subsection{Proof of Proposition~\ref{prop:3}}\label{sec:proof.prop.3}

First, we state two auxiliary technical lemmas, whose proofs are deferred to Appendix~\ref{app}.

\begin{lemma}\label{lem:nonminimax-var}
Fix $p\in [2,\infty)$. Let $\widetilde{f}_0$ be as in the proof of Proposition~\ref{prop:2}. There exist $b_0\in (0,1)$ and $c > 0$ such that for all $b\in (0,b_0]$ and all $n\in \N$,
\[
\EE_{\widetilde{f}_0}\big(\|\hat{f}_{n,b} - \widetilde{f}_0\|_p^p\big) \geq c \, \frac{\mathcal{I}(b,p)}{(n b^{1/2})^{p/2}}, \quad \mathcal{I}(b,p) \leqdef \int_b^{1/2} x^{-p/4} \rd x.
\]
Consequently,
\[
R_n(\hat{f}_{n,b},\widetilde{f}_0) \geq c \, \frac{\{\mathcal{I}(b,p)\}^{1/p}}{(n b^{1/2})^{1/2}}.
\]
\end{lemma}

\begin{lemma}\label{lem:nonminimax-bias}
Fix $\beta\in (0,2]$. There exist $\widetilde{L} > 1$, $b_1\in (0,1)$, and $c > 0$ such that, for every $b\in (0,b_1]$, there exists a density $\widetilde{f}_{\beta,b}\in \Sigma(\beta,\widetilde{L})$ satisfying
\[
\|\EE_{\widetilde{f}_{\beta,b}}[\hat{f}_{n,b}] - \widetilde{f}_{\beta,b}\|_{L^1([0,1])} \geq c \, b^{\beta/2},
\]
and therefore, for every $p\in [1,\infty)$,
\[
R_n(\hat{f}_{n,b},\widetilde{f}_{\beta,b}) \geq c \, b^{\beta/2}.
\]
\end{lemma}

Now, we find a general lower bound on the maximal risk. Fix $n\in \N$ and $b\in (0,1)$. Since $\widetilde{f}_0\in \Sigma(\beta,\widetilde{L})$ and $\widetilde{f}_{\beta,b}\in \Sigma(\beta,\widetilde{L})$ (for the explicit definition, see \eqref{eq:fbeta-def} together with the smoothing described in the proof of Lemma~\ref{lem:nonminimax-bias}), one has
\begin{equation}\label{eq:sup-lower-prop3}
R_n\{\hat{f}_{n,b},\Sigma(\beta,\widetilde{L})\} \geq \frac{1}{2}\left\{R_n(\hat{f}_{n,b},\widetilde{f}_0) + R_n(\hat{f}_{n,b},\widetilde{f}_{\beta,b})\right\}.
\end{equation}
Applying Lemmas~\ref{lem:nonminimax-var} and~\ref{lem:nonminimax-bias} to~\eqref{eq:sup-lower-prop3} yields that for all $b \leq b_0\wedge b_1 \equiv \min(b_0,b_1)$,
\begin{equation}\label{eq:key-lower}
R_n\{\hat{f}_{n,b},\Sigma(\beta,\widetilde{L})\} \geq c \left\{\frac{\{\mathcal{I}(b,p)\}^{1/p}}{(n b^{1/2})^{1/2}} + b^{\beta/2}\right\}.
\end{equation}

Next, we optimize the lower bound we just found. Fix $p\in [4,\infty)$ and consider the function
\[
\Phi_n(b) \leqdef \frac{\{\mathcal{I}(b,p)\}^{1/p}}{(n b^{1/2})^{1/2}} + b^{\beta/2}, \quad b\in (0,b_0\wedge b_1].
\]
Then~\eqref{eq:key-lower} implies $R_n\{\hat{f}_{n,b},\Sigma(\beta,\widetilde{L})\} \geq c \, \Phi_n(b)$. Hence, for any bandwidth sequence $(b_n)$ with $b_n \leq b_0\wedge b_1$, eventually,
\begin{equation}\label{eq:inf-bound}
R_n\{\hat{f}_{n,b_n},\Sigma(\beta,\widetilde{L})\} \geq c \inf_{b\in (0,b_0\wedge b_1]}\Phi_n(b).
\end{equation}
(If $b_n$ does not eventually lie in $(0,b_0\wedge b_1]$, then there exists a subsequence $(n_k)$ such that $b_{n_k}> b_0\wedge b_1$ for all $k$. Along this subsequence, the maximal risk $R_{n_k}\{\hat{f}_{n_k,b_{n_k}},\Sigma(\beta,\widetilde{L})\}$ is bounded away from $0$ (since the bandwidth is bounded below), so the ratio in the statement of Proposition~\ref{prop:3} diverges because $r_{n_k}\{\Sigma(\beta,\widetilde{L})\}\to 0$. Therefore, to control the $\liminf$ it suffices to consider sequences that eventually satisfy $b_n\leq b_0\wedge b_1$, in which case \eqref{eq:inf-bound} applies.)

\medskip
\noindent\emph{Case $p = 4$.}
Here $\mathcal{I}(b,4) = \int_b^{1/2} x^{-1} \rd x = \ln(1/(2 b))$, so for $b$ small enough,
\[
\Phi_n(b) \geq c\left\{\frac{|\ln b|^{1/4}}{(n b^{1/2})^{1/2}} + b^{\beta/2}\right\} = c\left\{n^{-1/2}b^{-1/4}|\ln b|^{1/4} + b^{\beta/2}\right\}.
\]
Set
\[
b_n^{\star} \leqdef n^{-2/(2\beta + 1)}(\ln n)^{1/(2\beta + 1)}.
\]
For $n$ large enough, one has $b_n^{\star}\in (0,b_0\wedge b_1]$. If $b \geq b_n^{\star}$, then
\[
\Phi_n(b) \geq c \, b^{\beta/2} \geq c \, (b_n^{\star})^{\beta/2} = c \, n^{-\beta/(2\beta + 1)}(\ln n)^{\beta/(2(2\beta + 1))}.
\]
If $b < b_n^{\star}$, then $b^{-1/4} \geq (b_n^{\star})^{-1/4} = n^{1/(2(2\beta + 1))}(\ln n)^{-1/(4(2\beta + 1))}$ and $|\ln b| \geq |\ln b_n^{\star}| \geq c\ln n$ for large $n$, hence
\[
n^{-1/2}b^{-1/4}|\ln b|^{1/4}
\geq c \, n^{-1/2}n^{1/(2(2\beta + 1))}(\ln n)^{-1/(4(2\beta + 1))}(\ln n)^{1/4}
= c \, n^{-\beta/(2\beta + 1)}(\ln n)^{\beta/(2(2\beta + 1))}.
\]
Therefore,
\[
\inf_{b\in (0,b_0\wedge b_1]}\Phi_n(b) \geq c \, n^{-\beta/(2\beta + 1)}(\ln n)^{\beta/(2(2\beta + 1))}.
\]
Combining this with~\eqref{eq:inf-bound} and the minimax rate stated in \eqref{eq:minimax.rate} yields
\[
\frac{R_n\{\hat{f}_{n,b_n},\Sigma(\beta,\widetilde{L})\}}{r_n\{\Sigma(\beta,\widetilde{L})\}}
\geq c \, (\ln n)^{\beta/(2(2\beta + 1))} \longrightarrow + \infty,
\]
which proves the claim when $p = 4$.

\medskip
\noindent\emph{Case $p > 4$.}
For $b\in (0,1/4]$,
\[
\mathcal{I}(b,p) = \int_b^{1/2} x^{-p/4} \rd x \geq \int_b^{2 b} x^{-p/4} \rd x = c \, b^{1 - p/4},
\]
so $\{\mathcal{I}(b,p)\}^{1/p} \geq c \, b^{1/p-1/4}$. Hence, for $b \leq 1/4$,
\[
\Phi_n(b) \geq c \, (n^{-1/2}b^{-1/4}b^{1/p-1/4} + b^{\beta/2}) = c \, (n^{-1/2}b^{-1/2 + 1/p} + b^{\beta/2}).
\]
Let $\alpha = \beta/2$ and $\gamma = 1/2-1/p > 0$. Then the right-hand side has the form $c \, (n^{-1/2} b^{-\gamma} + b^{\alpha})$. The minimum is attained at a bandwidth $b$ of order $n^{-1/(2(\alpha + \gamma))}$, so that
\[
\inf_{b\in (0,1/4]} c \, (n^{-1/2} b^{-\gamma} + b^{\alpha}) \geq c \, n^{-\alpha/(2(\alpha + \gamma))} = c \, n^{-\beta/(2\beta + 2-4/p)}.
\]
Therefore, for $n$ large enough,
\[
\inf_{b\in (0,b_0\wedge b_1]}\Phi_n(b) \geq c \, n^{-\beta/(2\beta + 2-4/p)}.
\]
Since $p > 4$ implies $2\beta + 2-4/p > 2\beta + 1$, one has $n^{\beta/(2\beta + 2-4/p)} = \oo(n^{\beta/(2\beta + 1)})$ and thus, by~\eqref{eq:minimax.rate},
\[
\frac{R_n\{\hat{f}_{n,b_n},\Sigma(\beta,\widetilde{L})\}}{r_n\{\Sigma(\beta,\widetilde{L})\}} \geq c \, \frac{n^{\beta/(2\beta + 1)}}{n^{\beta/(2\beta + 2-4/p)}} \longrightarrow +\infty.
\]
This completes the proof of Proposition~\ref{prop:3}.

\appendix

\begin{appendices}

\section{Proofs of technical lemmas}\label{app}

\begin{proof}[Proof of Lemma~\ref{lem:nonminimax-var}]
Fix $p\in [2,\infty)$ and let $\widetilde{f}_0$ be as in the proof of Proposition~\ref{prop:2}. Recall in particular that $\widetilde{f}_0(x)=1$ for all $x\in[0,7/8]$, and hence for all $x\in[b,1/2]$ when $b\leq 1/2$.
We use the inequality $\EE[|U|^p] \geq 2^{-p} \, \EE[|U - \EE[U]|^p]$, which holds for any random variable $U$ and $p \geq 1$. Applying this pointwise to $\smash{U = \hat{f}_{n,b}(x) - \widetilde{f}_0(x)}$ yields
\[
\EE_{\widetilde{f}_0}\big[|\hat{f}_{n,b}(x) - \widetilde{f}_0(x)|^p\big] \geq 2^{-p} \, \EE_{\widetilde{f}_0}\big[|\hat{f}_{n,b}(x) - \EE_{\widetilde{f}_0}[\hat{f}_{n,b}(x)]|^p\big].
\]
Since $p \geq 2$, integrating with respect to $x$ and applying Jensen's inequality yield
\begin{equation}\label{eq:var-lower-int}
\EE_{\widetilde{f}_0}\big(\|\hat{f}_{n,b} - \widetilde{f}_0\|_p^p\big) \geq 2^{-p} \int_0^{\infty} \big\{\Var_{\widetilde{f}_0}(\hat{f}_{n,b}(x))\big\}^{p/2} \rd x.
\end{equation}

For each fixed $x\geq 0$,
\[
\Var_{\widetilde{f}_0}(\hat{f}_{n,b}(x)) = \frac{1}{n} \Var_{\widetilde{f}_0}\big(K_b(x,X_1)\big) = \frac{1}{n} \left(\EE_{\widetilde{f}_0}[K_b(x,X_1)^2] - \{\EE_{\widetilde{f}_0}[K_b(x,X_1)]\}^2\right).
\]
Recall the expansion of the squared kernel integral \eqref{eq:Bb-R} derived in the proof of Lemma~\ref{lem:prop2-key}:
\[
\int_0^{\infty} K_b(x,t)^2 \rd t = \frac{1}{2\sqrt{\pi}} \, b^{-1/2} x^{-1/2} \frac{R(x/b)^2}{R(2x/b)},
\]
where $R$ is the Stirling ratio function in \eqref{eq:Stirling.ratio}. For $x\in [b, 1/2]$, we have $x/b \geq 1$. Since $R$ is increasing and positive, the ratio $R(x/b)^2/R(2x/b)$ is bounded from below by a positive constant. Furthermore, as argued in Lemma~\ref{lem:prop2-key}, the integral on the tail $[7/8,\infty)$ is exponentially small.
Thus, for $b$ small enough and all $x\in [b,1/2]$,
\[
\EE_{\widetilde{f}_0}[K_b(x,X_1)^2]
= \int_0^1 K_b(x,t)^2 \widetilde{f}_0(t) \, \rd t
\geq \int_0^{7/8} K_b(x,t)^2 \rd t
\geq c \, b^{-1/2} x^{-1/2}.
\]
Moreover, $\{\EE_{\widetilde{f}_0}[K_b(x,X_1)]\}^2 \leq \|\widetilde{f}_0\|_{\infty}^2 < \infty$ is negligible compared to the second moment above as $b\to 0$. It follows (possibly shrinking $b_0$) that
\[
\Var_{\widetilde{f}_0}(\hat{f}_{n,b}(x)) \geq \frac{c}{n} b^{-1/2} x^{-1/2}, \quad x\in [b,1/2], ~b\in (0,b_0].
\]

Plugging this into~\eqref{eq:var-lower-int} and restricting the integral to $[b,1/2]$ yields
\[
\EE_{\widetilde{f}_0}(\|\hat{f}_{n,b} - \widetilde{f}_0\|_p^p)
\geq c \int_b^{1/2}\left(\frac{b^{-1/2} x^{-1/2}}{n}\right)^{p/2} \rd x
= c \, \frac{1}{(n b^{1/2})^{p/2}} \int_b^{1/2} x^{-p/4} \rd x,
\]
which is the first claim. Taking the $p$th root gives the second.
\end{proof}

Before proving Lemma~\ref{lem:nonminimax-bias}, we need the following local lower bound on the gamma kernel.

\begin{lemma}\label{lem:K-local-lower}
Fix $0 < a_0 < a_1 < 1$ and $\delta\in (0,3)$. Then there exist $b_2\in (0,1)$ and $c > 0$ such that for all $b\in (0,b_2]$ and all $x\in [a_0,a_1]$,
\[
K_b(x,x + \delta b^{1/2}) \geq c \, b^{-1/2}.
\]
\end{lemma}

\begin{proof}[Proof of Lemma~\ref{lem:K-local-lower}]
Write
\[
K_b(x,x + \delta b^{1/2}) = K_b(x,x) Q_{b,\delta}(x), \quad
Q_{b,\delta}(x) \leqdef \exp\left\{\frac{x}{b}\ln\left(1 + \frac{\delta b^{1/2}}{x}\right)-\frac{\delta}{b^{1/2}}\right\}.
\]
Using $\ln(1 + u) = u-u^2/2 + \OO(u^3)$ with $u = \delta b^{1/2}/x$ gives, uniformly for $x\in [a_0,a_1]$,
\[
\ln Q_{b,\delta}(x)
= \left(\frac{x}{b}\right)\left(\frac{\delta b^{1/2}}{x}-\frac{\delta^2 b}{2x^2} + \OO(b^{3/2})\right) -\frac{\delta}{b^{1/2}}
= -\frac{\delta^2}{2x} + \OO(b^{1/2}).
\]
Therefore, $Q_{b,\delta}(x)\to \exp\{-\delta^2/(2x)\}$ uniformly in $x\in [a_0,a_1]$ as $b\to 0$, so $Q_{b,\delta}(x) \geq c$ for $b$ small enough. Next, using the Stirling ratio function in \eqref{eq:Stirling.ratio}, Stirling's formula gives
\[
K_b(x,x) = \frac{x^{x/b} e^{-x/b}}{b^{x/b + 1} \Gamma(x/b + 1)} = \frac{R(x/b)}{\sqrt{2\pi x b}} \geq c \, b^{-1/2},
\]
uniformly on $x\in [a_0,a_1]$ and $b$ small enough, since $x/b\to \infty$ and $R(\cdot)$ is increasing with $\lim_{u\to \infty}R(u) = 1$.
Combining the bounds on $K_b(x,x)$ and $Q_{b,\delta}(x)$ yields the claim.
\end{proof}

\begin{proof}[Proof of Lemma~\ref{lem:nonminimax-bias}]
Fix $\beta\in (0,2]$ and $L > 1$. Let $\psi:\R\to \R$ be the compactly supported $C^2$ function
\[
\psi(u) \leqdef (1 - u^2)^3 \, \ind_{[-1,1]}(u),
\]
so that $\psi(0) = 1$, $0 \leq \psi \leq 1$, and $\psi$ is nonincreasing on $[0,1]$.
Set
\[
L_{\beta} \leqdef \frac{L}{16}, \quad N \leqdef \left\lceil \frac{1}{24 b^{1/2}}\right\rceil, \quad t^{(k)} \leqdef \frac14 + 3 b^{1/2}(2k-1), \quad k = 1,\ldots,2N,
\]
and define the preliminary (non-smooth at $x=1$) test density
\begin{equation}\label{eq:fbeta-def}
f_{\beta,b}(x) \leqdef \ind_{[0,1]}(x) + L_{\beta}(3 b^{1/2})^{\beta} \sum_{k=1}^{2N}(-1)^k \, \psi\left(\frac{x-t^{(k)}}{3 b^{1/2}}\right) \ind_{[0,1]}(x),
\end{equation}
which is illustrated in Figure~\ref{fig:fbeta}.

\begin{figure}[!ht]
\centering
\includegraphics[width=0.92\textwidth]{graph/fbeta.pdf}
\caption{Visualization of the preliminary (non-smooth at $x=1$) test density $f_{\beta,b}$ defined in \eqref{eq:fbeta-def} for $L=2$, $b=0.0005$, and various values of $\beta \in (0,2]$. The oscillations represent the localized perturbations used to lower bound the maximal risk.}
\label{fig:fbeta}
\end{figure}

For $b$ small enough, the supports of the translated bumps are disjoint and contained in $[0,7/8]$, and $L_{\beta}(3 b^{1/2})^{\beta} \leq 1/2$, so $f_{\beta,b} \geq 1/2$ on $[0,1]$. Moreover, since there are as many $+1$ as $-1$ signs in the sum, the integral of the bump sum is $0$, hence $\smash{\int_0^1 f_{\beta,b}(x) \rd x = 1}$. Thus $f_{\beta,b}$ is a density supported on $[0,1]$. However, due to the hard cutoff at $x=1$, it does not satisfy the smoothness requirements imposed by the definition of $\Sigma(\beta,L)$.

\newpage
Let $\widetilde{f}_{\beta,b}$ denote a $C^{\infty}$ modification of $f_{\beta,b}$ on $[7/8,1]$ such that:
\begin{itemize}\setlength\itemsep{0em}
\item $\widetilde{f}_{\beta,b}=f_{\beta,b}$ on $[0,7/8]$;
\item $\widetilde{f}_{\beta,b}$ is supported on $[0,1]$ and $\int_0^1 \widetilde{f}_{\beta,b}(x) \, \rd x=1$;
\item $\widetilde{f}_{\beta,b}$ is $C^{\infty}$ on $(0,\infty)$ with $\widetilde{f}_{\beta,b}^{(k)}(1)=0$ for all $k\in\N_0$;
\item $\widetilde{f}_{\beta,b}\in \Sigma(\beta,\widetilde{L})$ for an appropriate $\widetilde{L} > 1$.
\end{itemize}
Again, we omit the explicit smoothing, since it does not affect the lower bounds below: all points $x$ and $u$ that appear in the argument lie in $[0,3/4 + 12 b^{1/2}]$, which itself is contained in say $[0,5/6]$ for $b$ small enough, and for such $x$ the kernel mass on $[7/8,1]$ is exponentially small in $1/b$ given that $7/8 - 5/6 > 0$. In particular, the bias lower bound obtained below for $f_{\beta,b}$ carries over to $\widetilde{f}_{\beta,b}$ (possibly after shrinking~$b_1$ and adjusting constants).

\medskip
\noindent\emph{Lower bound on the $L^1([0,1])$ bias.}
Let $\Delta_N \leqdef \{k\in \{1,\ldots,2N\}: k \text{ is even}\}$, so that the corresponding bumps are \emph{positive}. Fix $ \e\in (0,1/2]$ (to be chosen later) and define the intervals
\[
T_k(\e,b) \leqdef \left\{x: |x-t^{(k)}| \leq \e b^{1/2}\right\}, \quad
I_k(b) \leqdef \left\{u: b^{1/2} \leq |u-t^{(k)}| \leq 2 b^{1/2}\right\}.
\]
For $k\in \Delta_N$ and $x\in T_k(\e,b)$, only the $k$th bump is active in~\eqref{eq:fbeta-def}, so
\[
f_{\beta,b}(x) = 1 + L_{\beta}(3 b^{1/2})^{\beta} \, \psi\left(\frac{x-t^{(k)}}{3 b^{1/2}}\right),
\]
and similarly for $u\in I_k(b)$. Since $\psi$ is nonincreasing on $[0,1]$, for $x\in T_k(\e,b)$ (so that $|(x-t^{(k)})/(3 b^{1/2})| \leq \e/3$)
and $u\in I_k(b)$ (so that $|(u-t^{(k)})/(3 b^{1/2})|\in [1/3,2/3]$), one has
\begin{equation}\label{eq:f.beta.diff.lower.bound}
f_{\beta,b}(x) - f_{\beta,b}(u)
= L_{\beta}(3 b^{1/2})^{\beta}\left\{\psi\left(\frac{x-t^{(k)}}{3 b^{1/2}}\right) - \psi\left(\frac{u-t^{(k)}}{3 b^{1/2}}\right)\right\}
\geq c \, b^{\beta/2},
\end{equation}
where $c \leqdef L_{\beta} 3^{\beta}\{\psi(\e/3) - \psi(1/3)\} > 0$ for fixed $ \e$.

Now fix $k\in \Delta_N$ and $x\in T_k(\e,b)$. Since $f_{\beta,b}$ is supported on $[0,1]$ and $\int_0^{\infty} K_b(x,u)\rd u = 1$, we have
\[
\EE_{f_{\beta,b}}[\hat{f}_{n,b}(x)] - f_{\beta,b}(x) = \int_0^{\infty} K_b(x,u)\{f_{\beta,b}(u) - f_{\beta,b}(x)\} \rd u,
\]
so we can write
\[
\Big|\EE_{f_{\beta,b}}[\hat{f}_{n,b}(x)] - f_{\beta,b}(x)\Big| \geq U_k(x) - V(x),
\]
where
\[
\begin{aligned}
U_k(x) &\leqdef \int_{I_k(b)} K_b(x,u)\{f_{\beta,b}(x) - f_{\beta,b}(u)\} \rd u, \\
V(x) &\leqdef \int_{\{f_{\beta,b}(u) \geq f_{\beta,b}(x)\}} K_b(x,u)\{f_{\beta,b}(u) - f_{\beta,b}(x)\} \rd u.
\end{aligned}
\]
From the previous pointwise difference lower bound \eqref{eq:f.beta.diff.lower.bound},
\begin{equation}\label{eq:Ak-lower}
U_k(x) \geq c \, b^{\beta/2} \int_{I_k(b)} K_b(x,u) \rd u.
\end{equation}
To bound the integral in \eqref{eq:Ak-lower} from below, note that for $x\in T_k(\e,b)$ and $u\in [t^{(k)} + \frac{3}{2} b^{1/2}, t^{(k)} + 2 b^{1/2}]$, one has $u\in I_k(b)$ and
\[
u - x\in \big[(\tfrac{3}{2}-\e)b^{1/2}, \, (2+\e)b^{1/2}\big] \subseteq [b^{1/2}, (5/2)b^{1/2}],
\]
provided $ \e \leq 1/2$. Moreover, for $b$ small enough, $x\in [1/4,3/4 + 12 b^{1/2}] \subseteq [0,5/6]$. Since for each fixed $x>0$ the function $u\mapsto K_b(x,u)$ is decreasing on $[x,\infty)$ (its mode is at $u=x$), it follows that, for all such $u$,
\[
K_b(x,u) \geq K_b\!\left(x,x+\frac{5}{2}b^{1/2}\right).
\]
Hence Lemma~\ref{lem:K-local-lower} (with $a_0 = 1/4$, $a_1 = 5/6$ and $\delta = 5/2$) implies
\[
\inf_{u\in [t^{(k)} + \frac{3}{2} b^{1/2}, \, t^{(k)} + 2 b^{1/2}]} K_b(x,u) \geq c \, b^{-1/2},
\]
so
\begin{equation}\label{eq:intK-lower}
\int_{I_k(b)} K_b(x,u) \rd u \geq \int_{t^{(k)} + \frac{3}{2} b^{1/2}}^{t^{(k)} + 2 b^{1/2}} K_b(x,u) \rd u
\geq c \, b^{-1/2}\cdot \frac{b^{1/2}}{2}
\geq c.
\end{equation}
Combining~\eqref{eq:Ak-lower} and~\eqref{eq:intK-lower} yields
\[
U_k(x) \geq c \, b^{\beta/2}.
\]

Next, set $H_b \leqdef L_{\beta}(3 b^{1/2})^{\beta}$. Since $0\leq \psi \leq 1$ and the translated bumps have disjoint supports, one has
\[
\sup_{u\in [0,1]} f_{\beta,b}(u) = 1 + H_b.
\]
Therefore, whenever $f_{\beta,b}(u) \geq f_{\beta,b}(x)$,
\[
0 \leq f_{\beta,b}(u) - f_{\beta,b}(x) \leq (1 + H_b) - f_{\beta,b}(x)
= H_b \, \left\{1 - \psi\left(\frac{x-t^{(k)}}{3 b^{1/2}}\right)\right\}.
\]
Since $\psi(v)=(1-v^2)^3$ for $|v|\leq 1$, one has $1-\psi(v)=3v^2-3v^4+v^6\leq 3v^2$, and thus, for $x\in T_k(\e,b)$,
\[
1 - \psi\left(\frac{x-t^{(k)}}{3 b^{1/2}}\right)
\leq 3\left(\frac{\e}{3}\right)^2 = \frac{\e^2}{3}.
\]
Hence,
\[
0 \leq f_{\beta,b}(u) - f_{\beta,b}(x) \leq \frac{H_b \, \e^2}{3},
\quad \text{whenever } f_{\beta,b}(u) \geq f_{\beta,b}(x),
\]
and therefore
\[
V(x) \leq \frac{H_b \, \e^2}{3} \int_0^1 K_b(x,u) \rd u \leq \frac{H_b \, \e^2}{3} \leq C \e^2 b^{\beta/2}.
\]
Choosing $\e > 0$ small enough so that $C \e^2 \leq c/2$, we obtain
\[
\Big|\EE_{f_{\beta,b}}[\hat{f}_{n,b}(x)] - f_{\beta,b}(x)\Big| \geq \frac{c}{2} b^{\beta/2}, \quad x\in T_k(\e,b), ~k\in \Delta_N.
\]
Integrating over $x$ and summing over $k\in \Delta_N$ gives
\[
\|\EE_{f_{\beta,b}}[\hat{f}_{n,b}] - f_{\beta,b}\|_{L^1([0,1])}
\geq \sum_{k\in \Delta_N} \int_{T_k(\e,b)} \Big|\EE_{f_{\beta,b}}[\hat{f}_{n,b}(x)] - f_{\beta,b}(x)\Big| \rd x
\geq c \, b^{\beta/2}\sum_{k\in \Delta_N} |T_k(\e,b)|.
\]
Since $|T_k(\e,b)| = 2 \e b^{1/2}$ and $|\Delta_N| = N \asymp b^{-1/2}$, the sum is bounded below by a positive constant independent of $b$. This proves $\|\EE_{f_{\beta,b}}[\hat{f}_{n,b}] - f_{\beta,b}\|_{L^1([0,1])} \geq c \, b^{\beta/2}$ for $b$ small enough.

Finally, since $\widetilde{f}_{\beta,b}$ differs from $f_{\beta,b}$ only on $[7/8,1]$ and, for $x\in[1/4,5/6]$, the kernel mass on $[7/8,1]$ is $\OO(\exp(-c/b))$, the same lower bound holds for $\widetilde{f}_{\beta,b}$ after possibly shrinking $b_1$ and adjusting constants. In particular,
\[
\|\EE_{\widetilde{f}_{\beta,b}}[\hat{f}_{n,b}] - \widetilde{f}_{\beta,b}\|_{L^1([0,1])} \geq c \, b^{\beta/2},
\]
and on a set of Lebesgue measure $1$, $\|g\|_p \geq \|g\|_{L^1([0,1])}$ for every $p\in[1,\infty)$ (see \eqref{eq:Lp.L1}), so
\[
R_n(\hat{f}_{n,b},\widetilde{f}_{\beta,b}) \geq \|\EE_{\widetilde{f}_{\beta,b}}[\hat{f}_{n,b}] - \widetilde{f}_{\beta,b}\|_p \geq \|\EE_{\widetilde{f}_{\beta,b}}[\hat{f}_{n,b}] - \widetilde{f}_{\beta,b}\|_{L^1([0,1])} \geq c \, b^{\beta/2}.
\]
This concludes the proof.
\end{proof}

\end{appendices}

\section*{Funding}
\addcontentsline{toc}{section}{Funding}

Ouimet is supported by the start-up fund (1729971) from the Universit\'e du Qu\'ebec à Trois-Rivi\`eres.

\section*{References}
\addcontentsline{toc}{chapter}{References}

% This line reduces the gap between references
%\setlength{\bibsep}{0pt plus 0.3ex}

\bibliographystyle{plainnat}
\bibliography{bib}

\end{document}

