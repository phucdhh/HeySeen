\documentclass[12pt]{amsart}

\usepackage{amscd,amssymb}
\usepackage[all]{xy}
%\usepackage[colorlinks=true,urlcolor=blue]{hyperref}
\usepackage[plainpages,backref,urlcolor=blue]{hyperref}
\usepackage[utf8]{inputenc}
%\usepackage[notref,notcite]{showkeys}
\usepackage{mathtools}
\DeclarePairedDelimiter{\floor}{\lfloor}{\rfloor}

\topmargin=0.1in
\textwidth5.95in
\textheight8.60in
\oddsidemargin=0.3in
\evensidemargin=0.3in

\theoremstyle{plain}
\newtheorem{thm}[subsection]{Theorem}
\newtheorem{lem}[subsection]{Lemma}
\newtheorem{prop}[subsection]{Proposition}
\newtheorem{cor}[subsection]{Corollary}

\theoremstyle{definition}
\newtheorem{rk}[subsection]{Remark}
\newtheorem{definition}[subsection]{Definition}
\newtheorem{ex}[subsection]{Example}
\newtheorem{conj}[subsection]{Conjecture}
\newtheorem{question}[subsection]{Question}

\numberwithin{equation}{section}
\setcounter{tocdepth}{1}

\newcommand{\OO}{\mathcal O}
\newcommand{\X}{\mathcal X}
\newcommand{\Y}{\mathcal Y}
\newcommand{\ZZ}{\mathcal Z}
\newcommand{\RR}{\mathcal R}
\newcommand{\I}{\mathcal I}
\newcommand{\J}{\mathcal J}
\newcommand{\F}{\mathcal F}
\newcommand{\A}{\mathcal A}
\newcommand{\B}{\mathcal B}
\newcommand{\QQ}{\mathcal Q}
\newcommand{\M}{\mathcal M}
\newcommand{\wC}{\widehat{C}}
\newcommand{\wB}{\widehat{\mathcal B}}
\newcommand{\wI}{\widehat{I}}
\newcommand{\wJ}{\widehat{J}}
\newcommand{\wy}{\widehat{y}}
\newcommand{\wz}{\widehat{z}}
\newcommand{\wP}{\widehat{\mathbb{P}^2}}

\newcommand{\D}{\mathcal D}
\newcommand{\CC}{\mathcal C}
\newcommand{\LL}{\mathcal L}
\newcommand{\E}{\mathcal E}
\newcommand{\V}{\mathcal V}
\newcommand{\w}{\mathbf w}
\newcommand{\al}{\alpha}
\newcommand{\be}{\beta}
\newcommand{\NN}{\mathcal N}
\newcommand{\SSS}{\mathcal S}

\newcommand{\Z}{\mathbb Z}
\newcommand{\Q}{\mathbb Q}
\newcommand{\R}{\mathbb R}
\newcommand{\C}{\mathbb C}
\newcommand{\PP}{\mathbb P}
%\newcommand{\S}{\mathbb S}
\newcommand{\N}{\mathbb N}

\newcommand{\bd}{\bf d}
\newcommand{\bc}{\bf c}
\newcommand{\bb}{\bf b}

\newcommand{\tX}{\widetilde{X}}
\newcommand{\tY}{\widetilde{Y}}
\newcommand{\tZ}{\widetilde{Z}}
\newcommand{\tM}{\widetilde{M}}

\DeclareMathOperator{\Hom}{Hom}
\DeclareMathOperator{\rank}{rank}
\DeclareMathOperator{\im}{im}
\DeclareMathOperator{\coker}{coker}
\DeclareMathOperator{\ab}{ab}
\DeclareMathOperator{\id}{id}
\DeclareMathOperator{\pd}{pd}
\DeclareMathOperator{\mult}{mult}
\DeclareMathOperator{\defect}{def}
\DeclareMathOperator{\codim}{codim}
\DeclareMathOperator{\supp}{supp}
\DeclareMathOperator{\genus}{genus}
\DeclareMathOperator{\depth}{depth}
\DeclareMathOperator{\reg}{reg}
\DeclareMathOperator{\indeg}{indeg}

\newcommand{\surj}{\twoheadrightarrow}
\newcommand{\eqv}{\Longleftrightarrow}

\newenvironment{romenum}
{\renewcommand{\theenumi}{\roman{enumi}}
\renewcommand{\labelenumi}{(\theenumi)}
\begin{enumerate}}{\end{enumerate}}



\begin{document}

\title[Graded Betti numbers of surfaces]{Graded Betti numbers of the Jacobian algebra of surfaces in $\PP^3$}

\author[Alexandru Dimca]{Alexandru Dimca$^1$}
\address{Universit\'e C\^ote d'Azur, CNRS, LJAD, France and Simion Stoilow Institute of Mathematics,
P.O. Box 1-764, RO-014700 Bucharest, Romania}
\email{Alexandru.Dimca@univ-cotedazur.fr}

\author[Gabriel Sticlaru]{Gabriel Sticlaru}
\address{Faculty of Mathematics and Informatics,
Ovidius University
Bd. Mamaia 124, 900527 Constanta,
Romania}
\email{gabriel.sticlaru@gmail.com }

\thanks{$^1$ partial support from the project ``Singularities and Applications'' - CF 132/31.07.2023 funded by the European Union - NextGenerationEU - through Romania's National Recovery and Resilience Plan.}

\subjclass[2010]{Primary 14H50; Secondary  14B05, 13D02, 32S22}

\keywords{Jacobian ideal, Jacobian algebra, exponents,  Tjurina numbers, graded Betti numbers}

\begin{abstract}

We compute an explicit
closed formula for the Hilbert polynomial of the Jacobian algebra $M(f)$ of a reduced  surface $X:f=0$ in $\PP^3$ in terms of the graded Betti numbers of the algebra $M(f)$. When $X$ has only isolated singularities, a result by A. du Plessis and C. T. C. Wall  yields new necessary condition for a set of positive integers to be the graded Betti numbers of the Jacobian algebra of such a surface.
The comparison with the plane curve case is discussed in detail and additional information is given in the case of nodal surfaces.

\end{abstract}

\maketitle

%%%%%%%%%%%%%%%%%%%%%%%%%%%%%%%%%%%%%%%%%%%%%%%%%%%%
\section{Introduction}
%%%%%%%%%%%%%%%%%%%%%%%%%%%%%%%%%%%%%%%%%%%%%%%%%%%%

Let $S=\C[x,y,z,t]$ be the polynomial ring in four variables $x,y,z,t$ with complex coefficients, and let $X:f=0$ be a reduced surface of degree $d\geq 3$ in the complex projective space $\PP^3$. 
We denote by $J_f$ the Jacobian ideal of $f$, i.e. the homogeneous ideal in $S$ spanned by the partial derivatives $f_x,f_y,f_z,f_t$ of $f$, and  by $M(f)=S/J_f$ the corresponding graded quotient ring, called the Jacobian (or Milnor) algebra of $f$.
Consider the general form of the minimal resolution of the Milnor algebra $M(f)$
of a reduced surface $X:f=0$ 
\begin{equation}
\label{res2A}
0 \longrightarrow \bigoplus_{k=1}^{r} S(1-d-b_k)
 \longrightarrow \bigoplus_{j=1}^{q} S(1-d-c_j)
   \longrightarrow \bigoplus_{i=1}^{p} S(1-d-d_i)
   \longrightarrow S^4(1-d)
   \longrightarrow S.
\end{equation}
where $p \geq 3$, $q \geq 0$ and $r \geq 0$.


We call the ordered sequence of degrees 
$${\bf d}=(d_1, \ldots, d_p), \  
{\bf c}=(c_1, \ldots, c_q) 
 \text{ and } {\bf b}=(b_1, \ldots, b_r)$$ 
  the {\it graded Betti numbers of the Jacobian algebra} $M(f)$, since they determine and are determined by the usual
graded Betti numbers of the Jacobian algebra $M(f)$ as defined for instance in \cite{Eis}. 

It is known that there is a unique polynomial $P(M(f))(u) \in \Q[u]$, called the {\it Hilbert polynomial} of $M(f)$, and an integer $k_0\in \N$ such that
\begin{equation}
\label{Hpoly}
 H(M(f))(k)= P(M(f))(k)
\end{equation}
for all $k \geq k_0$. We denote by $\Sigma$ the singular subscheme of $X$, which is defined by the Jacobian ideal $J(f)$. The general theory of Hilbert polynomials says that the degree of  $P(M(f))$ is given by the dimension of the support of  $\OO_{\Sigma}$, the coherent sheal associated to the graded $S$-module $ M(f)$. Hence the assumption $\dim \Sigma=0$ implies that the polynomial
$P(M(f))$ is a constant, namely the total Tjurina number of $X$, given by
\begin{equation}
\label{ab0}
P(M(f))=\tau(X)=\sum_{s \in \Sigma} \tau(X,s),
\end{equation}
where $\tau(X,s)$ denotes the Tjurina number of the isolated singularity $(X,s)$, and
$\dim \Sigma=1$ implies that 
\begin{equation}
\label{ab1}
P(M(f))(u)=au+b,
\end{equation}
where $a=\deg(\Sigma)$, the degree of the subscheme $\Sigma$. 

The first main result of this note is the following computation of the Hilbert polynomial $P(M(f))$ in terms of the graded Betti numbers of the Jacobian algebra $M(f)$
introduced in \eqref{res2A}.

 \begin{thm}
\label{thm1}
For the minimal resolution \eqref{res2A} of the Jacobian algebra $M(f)$ of a reduced surface $X:f=0$ of degree $d$ in $\PP^3$, one has the following.
\begin{enumerate}



\item For any such surface, one has 
$$p+r=q+3 \text{ and }  \sum_{i=1}^pd_i- \sum_{j=1}^qc_j+ \sum_{k=1}^rb_k=d-1.$$

\item The surface $X$ has at most isolated singularities if and only if
$$(d-1)^2+\sum_{i=1}^pd_i^2-\sum_{j=1}^qc_j^2+\sum_{k=1}^rb_k^2=0.$$
If this is the case, then the total Tjurina number of $X$ is given by the formula
$$6\tau(X)=(d-1)^3-\sum_{i=1}^pd_i^3+\sum_{j=1}^qc_j^3-\sum_{k=1}^rb_k^3.$$

\item The surface $X$ is smooth if and only if $p=6$, $q=4$, $r=1$ and
$${\bf d}=(d-1)_6, \ {\bf c}=(2d-2)_4 \text{ and }  {\bf b}=3(d-1).$$
\item If the surface $X:f=0$ has a 1-dimensional singularity subscheme $\Sigma$, then the Hilbert polynomial $P(M(f))$ of the  Jacobian algebra $M(f)$ is given by
$$P(M(f))(u)=\frac{A}{2}u -B,$$
where
$$A=  (d-1)^2+\sum_{i=1}^pd_i^2-\sum_{j=1}^qc_j^2+\sum_{k=1}^rb_k^2     $$
and
$$B= \frac{d(d-3)^2-4}{3}+\frac{(d-3)}{2}\left(\sum_{i=1}^pd_i^2-\sum_{j=1}^qc_j^2+\sum_{k=1}^rb_k^2 \right)+\frac{1}{6}\left(\sum_{i=1}^pd_i^3-\sum_{j=1}^qc_j^3+\sum_{k=1}^rb_k^3\right).  $$


\end{enumerate}

\end{thm}
When the surface $X$ has only isolated singularities, that is when $\dim \Sigma =0$, then using a result by A. du Plessis and C.T.C. Wall quoted below in Theorem \ref{thmC}, we obtain the following new restrictions on the
graded Betti numbers of the Jacobian algebra $M(f)$.

\begin{cor}
\label{corC} 
If the surface $X$ has  isolated singularities,
then
$$6d_1(d-d_1-1)(d-1) \leq  \sum_{i=1}^pd_i^3-\sum_{j=1}^qc_j^3+\sum_{k=1}^rb_k^3 \leq 6d_1(d-1)^2.$$
\end{cor}
Theorem \ref{thm1} is proved in Section 2 and Corollary \ref{corC} is proved in Section 3. 

In Section 4 a comparison with the plane curve case is discussed in detail. Surprizingly, many facts holding for the graded Betti numbers of curves fail in the case of surfaces, see for instance Proposition \ref{propT} and Remark \ref{rk1}.
The similar behavior of the graded Betti numbers of  maximal Tjurina curves and the graded Betti numbers of surfaces with large number of nodes, such as the Cayley surface from Example \ref{ex10} and the Kummer surface from Example \ref{ex20} is highlighted in Remark \ref{rk2} and it may be an interesting direction of further research.
However, Example \ref{ex30} and Proposition \ref{propN} show that this analogy is rather subtle.
 
In Section 5 we collect a number of additional examples. All the minimal resolutions  corresponding to \eqref{res2A} are computated in this note using the
Computer Algebra softwares CoCoA \cite{CoCoA} and SINGULAR \cite{Singular}.



\section{Proof of Theorem \ref{thm1}}


We start with the following two Lemmas, whose proofs are elementary and straightforward, so we leave them to the reader.
\begin{lem}
\label{lem1}
For any integer $a$ and $k > |a|$, one has
$$\dim S_{k+a}=\binom{k+a+3}{3}=\frac{k^3+3(a+2)k^2+(3a^2+12a+11)k+a^3+6a^2+11a+6}{6}.$$
\end{lem}
Using the resolution \eqref{res2A}, we get the following
\begin{equation}
\label{res2C1}
 \dim M(f)_{s+d-1}= \dim S_{s+d-1}-4 \dim S_s+\sum_{i=1}^p\dim S_{s-d_i}-\sum_{j=1}^q\dim S_{s-c_j}+\sum_{k=1}^r\dim S_{s-b_k},
\end{equation}
for any $s$ large enough.
Using now Lemma \ref{lem1}, we get the following.

\begin{lem}
\label{lem2}
With the above notation, for any large integer $s$, one has the equality
 $$6 \dim M(f)_{s+d-1}=$$
 $$=(p-q+r-3)s^3+3(d-1-d_1-d_2-d_3-\sum_{i=4}^p(d_i-2)+\sum_{j=1}^q(c_j-2)-\sum_{k=1}^r(b_k-2))s^2+
 $$
 $$+  (3d^2+6d+2-44+\sum_{i=1}^p(3d_i^2-12d_i+11)-  \sum_{j=1}^q(3c_j^2-12c_j+11) +\sum_{k=1}^r(3b_k^2-12b_k+11) )s+ $$
 $$+d^3+3d^2+2d-24-\sum_{i=1}^p(d_i^3-6d_i^2+11d_i-6)+\sum_{j=1}^q(c_j^3-6c_j^2+11c_j-6)-\sum_{k=1}^r(b_k^3-6b_k^2+11b_k-6).$$
\end{lem}
Using these computations, we can prove Theorem \ref{thm1} as follows.
By definition of the Hilbert polynomial $P(M(f))$ we have
$$6P(M(f))(s+d-1)=6 \dim M(f)_{s+d-1}.$$
Since the surface $X$ is reduced, we have
$$\deg P(M(f))=\dim \Sigma \leq 1,$$
and hence the coefficients of $s^3$ and of $s^2$ in Lemma \ref{lem2} must vanish. This proves the claim (1).

To prove the claim (2), we note that $X$ has at most isolated singularities if and only if $\dim \Sigma <1$.  This last condition is equivalent to the vanishing of the coefficients of $s$ in Lemma \ref{lem2}, in addition to the vanishings from the claim (1).
When all these vanishing holds, then we conclude the proof of claim (2) by using \eqref{ab0} and the expression of the constant term in 
 Lemma \ref{lem2}, simplified by using the equalities in (1) and the first equality in (2).
 
 To prove the claim (3), assume first that $X:f=0$ is smooth. Then the partial derivatives $f_x,f_y,f_z$ and $f_t$ form a regular sequence in $S$ and the resolution of $M(f)$ is well known in this case, and has the form
 $$0 \to S(4-4d) \to S(3-3d)^4 \to S(2-2d)^6 \to S(1-d)^4 \to S.$$
It follows that  ${\bf d}$, $ {\bf c}$ and $ {\bf b}$ are given by the equalities in claim (3) when $X$ is smooth. Conversely, if ${\bf d}$, $ {\bf c}$ and $ {\bf b}$ are given by the equalities in claim (3), then using the claim (2) we see that $X$ has at most isolated singularities and that $\tau(X)=0$. Therefore the surface $X$ is smooth.

The proof of claim (4) follows directly from \eqref{ab1} and  Lemma \ref{lem2}, where $s$ has to be replaced by $u-(d-1)$.

%%%%%%%%%%%%%%%%%%%%%%%%%%%%%%%%%%%%%%%%%%%%%%%%%%%%

%%%%%%%%%%%%%%%%%%%%%%%%%%%%%%%%%%%%%%%%%%%%%%%%%%%%
\section{Proof of Corollary \ref{corC}}
%%%%%%%%%%%%%%%%%%%%%%%%%%%%%%%%%%%%%%%%%%%%%%%%%%%%


One has the following result, see \cite[Theorem 5.3]{duPCTC01}.
\begin{thm}
\label{thmC}
If the surface $X$ has at most isolated singularities,
then
$$(d-1)^3-d_1(d-1)^2 \leq \tau(X) \leq (d-1)^3-d_1(d-d_1-1)(d-1).$$
\end{thm}
In the case of plane curves, the corresponding result was obtained in \cite{duPCTC}, 
and played a key role in the understanding of free curves. Indeed, the reduced curve $C$ is free if and only if 
$$\tau(C)=(d-1)^2-d_1(d-d_1-1),$$
i.e. the upper bound is attained, see \cite{Dmax, E} for related results.
A free surface $X$ has necessarily non-isolated singularities,
and so freeness must be related to other invariants, see for instance
\cite{DmaxS}.


\begin{rk}
\label{rkC} 
The lower bound in Theorem \ref{thmC} is attained for any pair $(d,d_1)$.
Indeed, it is enough to find a degree $d$, reduced curve $C: f'(x,y,z)=0$ such that $d_1$ is the minimal exponent of $C$ and
$$\tau(C)=(d-d_1-1)(d-1),$$
and then take $X:f=0$, with 
$$f(x,y,z,t)=f'(x,y,z)+t^d.$$
The existence of curves $C$ as above is shown in \cite[Example 4.5]{3syz} and a complete characterization of them is given in \cite[Theorem 3.5 (1)]{3syz}.
\end{rk}

\begin{rk}
\label{rkC2} 
The upper bound in Theorem \ref{thmC} is attained for any pair $(d,d_1)$ with $2d_1<d$, since for such pairs $(d,d_1)$ the existence of free plane curves $C:f'=0$ of degree $d$ and with exponents $(d_1,d_2)$ is shown in
\cite{DStExpo} and then one constructs the surface $X$ as in Remark \ref{rkC} above.
It is an interesting {\it open question} to improve the upper bound in Theorem \ref{thmC} when $2d_1 \geq d$. The best upper bound for such pairs is (at least conjecturally) known in the case of plane curves, see  \cite{duPCTC, maxTjurina}, and is given by the stronger inequality
$$\tau(C) \leq (d-1)^2-d_1(d-d_1-1)-{2d_1+2-d \choose 2}.$$
If  we start with a degree $d$ and a reduced curve $C: f'(x,y,z)=0$ such that $d_1 \geq d/2$ and
 take $X:f=0$, with $$f(x,y,z,t)=f'(x,y,z)+t^d,$$
 then $f$ and $f'$ have the same minimal exponent $d_1$ and
 \begin{equation} 
\label{betterb} 
\tau(X) \leq (d-1)^3-d_1(d-d_1-1)(d-1)-{2d_1+2-d \choose 2}(d-1).
\end{equation}  
However, this stronger inequality fails for surfaces not constructed as suspensions of plane curves, as the following examples show.
\end{rk}
\begin{ex}
\label{ex10}
Consider the Cayley surface
$$X:f=xyz+xyt+xzt+yzt =0$$
in $\PP^3$ having four $A_1$-singularities. Then $d=3$, $d_1=2 >d/2$ and 
$\tau(X)=4$. Indeed, the minimal resolution of the Jacobian algebra is given by
$$0 \to S[-6]^2 \to S[-5]^8 \to S[-4]^9 \to S[-2]^4 \to S$$
and hence 
$${\bf d}=(2_9), \ {\bf c} =(3_8) \text{ and } {\bf b}=(4_2).$$
The inequality in Theorem \ref{thmC} is in this case
$0 \leq \tau(X) \leq 8,$
while the bound given by \eqref{betterb} is $2$, which is clearly not good.
\end{ex}

\begin{ex}
\label{ex20}
Consider the Kummer surface
$$X:f=x^4+y^4+z^4+t^4-y^2 z^2-z^2 x^2-x^2 y^2-x^2 t^2-y^2 t^2-z^2 t^2 =0$$
in $\PP^3$ having sixteen $A_1$-singularities. Then $d=4$, $d_1=3 >d/2$ and 
$\tau(X)=16$. Indeed, the minimal resolution of the Jacobian algebra is given by
$$0 \to S(-8)^3\to S(-7)^{12} \to S(-6)^{12} \to S(-3)^4\to S\to 0$$
and hence 
$${\bf d}=(3_{12}), \ {\bf c} =(4_{12}) \text{ and } {\bf b}=(5_3).$$
The inequality in Theorem \ref{thmC} is in this case
$0 \leq \tau(X) \leq 27,$
while the bound given by \eqref{betterb} is $15$, which is clearly not good.
\end{ex}

\begin{ex}
\label{ex30}
Consider finally the octic  surface with 144 nodes 
$$X:f=16(x^8+y^8+z^8+t^8)+224(x^4y^4+x^4z^4+x^4t^4+y^4z^4+y^4t^4+z^4t^4)+$$
$$+
2688x^2y^2z^2t^2-9(x^2+y^2+z^2+t^2)^4=0.$$
This is a special case of Chebyshev hypersurfaces, which are classical examples of
nodal hypersurfaces with many singularities. They were introduced by Chmutov to construct complex projective
hypersurfaces with a large number of nodes, see \cite{AGV} Vol. 2, p. 419, \cite{Chm} as well as \cite[Corollary 3.2, (iii)]{DStChm}.
The minimal resolution of the Jacobian algebra is given by
$$0\to S(-20)^4 \oplus S(-22) \to S(-17)^4 \oplus S(-18)^{13} \to S(-14)^6\oplus S(-16)^9\to S(-7)^4\to S$$
and hence 
$${\bf d}=(7_6,9_9), \ {\bf c} =(10_4,11_{13}) \text{ and } {\bf b}=(13_4,15).$$
The inequality in Theorem \ref{thmC} is in this case
$0 \leq \tau(X)=144 \leq 343,$
while the bound given by \eqref{betterb} is $147$, which is also good.
\end{ex}

\section{Comparison to the curve case}
Let $R=\C[x,y,z]$ and 
consider  the general form of the minimal resolution of the Milnor algebra $M(f')$
of a reduced plane curve $C:f'=0$, which is assumed not to be free
\begin{equation}
\label{res2A2}
0 \longrightarrow \bigoplus_{j=1}^{q'} R(1-d-c'_j)
   \longrightarrow \bigoplus_{i=1}^{p'} R(1-d-d'_i)
   \longrightarrow R^3(1-d)
   \longrightarrow R,
\end{equation}
with $c'_1\leq ...\leq c'_{q'}$ and $d'_1\leq ...\leq d'_{p'}$, where $d=\deg f'$, see for instance \cite{HS}. One has
$$p'=q'+2,$$
which corresponds to the first equality in Theorem \ref{thm1} (1) above.
It follows from \cite[Lemma 1.1]{HS} that
\begin{equation}
\label{res2B}
 \epsilon_j=  d'_{j+2} -c'_j \ge 1
\qquad j = 1,\dots,q'.
\end{equation}

Using \cite[Formula (13)]{HS}, one obtains the relation
\begin{equation}
\label{res2C}
d'_1 + d'_2 = d - 1 + \sum_{j=1}^{q'} \epsilon_j
\end{equation}
or, equivalently,
\begin{equation}
\label{res2C2}
\sum_{i=1}^{p'}d_i-\sum_{j=1}^{q'}c'_j=d-1,
\end{equation}
which corresponds to the second equality in Theorem \ref{thm1} (1) above. 

A new invariant was recently introduced for a reduced plane curve $C$, namely the type of $C$ defined by
 \begin{equation} 
\label{t1} 
t(C)=d'_1+d'_2-d+1, 
 \end{equation} 
 see \cite{ADP}. When the curve $C$ is not free, then one clearly has
 \begin{equation} 
\label{t11} 
t(C)= \sum_{j=1}^{q'} \epsilon_j.
 \end{equation}  
 This invariant has nice properties, for instance a 
 reduced plane curve $C$ is free (resp. plus-one generated) if and only if
 $t(C)=0$ (resp. $t(C)=1$). More, one has $t(C) \geq 0$ for any reduced curve, and the curves with $t(C)=2$ and $t(C)=3$ have been studied in detail in \cite{ADP, Type3}.
 One may try to extend this invariant to surfaces $X$ in $\PP^3$ by setting
 \begin{equation} 
\label{t2}
t(X)=d_1+d_2+d_3+1-d.
 \end{equation} 
 The next result shows that this case is much more complicated.
  \begin{prop} 
\label{propT}
Let $X:f=0$ be a reduced surface in $\PP^3$ with the minimal resolution \eqref{res2A}. Then
$$t(X)=d_1+d_2+d_3+1-d \geq 0$$
if there are first order Jacobian syzygies $\rho_1, \rho_2$ and $\rho_3$
which are linearly independent over the field of fractions $K$ of the polynomial ring $S$ and
such that $\deg \rho_j=d_j$ for $j=1,2,3$. 
 \end{prop} 
Note that if the surface $X$ is tame with respect to the pair
$(\rho_1, \rho_2)$ as in \cite[Definition 1.2]{HD}, then  the first order Jacobian syzygies $\rho_1, \rho_2$ and $\rho_3$
 are linearly independent over the field $K$ for any new additional syzygy $\rho_3$. However, unlike the curve case, Example \ref{ex4.1} shows that in general one may have $t(X)<0$.
 
 \proof
 Let $M(E,\rho_1,\rho_2,\rho_3)$ be the $4 \times 4$ matrix with the first row $(x,y,z,t)$ and the $j$-th row, for $j=2,3,4$ being given by the components of the  syzygy $\rho_{j-1}$. Then 
 $$g=\det M(E,\rho_1,\rho_2,\rho_3) \ne 0$$
 if and only if  the syzygies $\rho_1, \rho_2$ and $\rho_3$
 are linearly independent over the field $K$. On the other hand, at any
 smooth point $s \in X$, the rows of $M(E,\rho_1,\rho_2,\rho_3)$ evaluated at $s$ are tangent vectors to $X$ at $s$. Since their number is larger than the dimension of this vector space, it follows that $g(s)=0$
 at any smooth point of $X$. Therefore $g$ vanishes on $X$, and hence $g$ is divisible by $f$, the surface $X$ being reduced.
 It follows that
 $$1+d_1+d_2+d_3=\deg g \geq \deg f=d,$$
 which proves our claim.
 \endproof
To see how the equality \eqref{t11} extends to the surface case, one may proceed as follows.
If we set $\al_j=c_j-d_{j+3}$ for $j=1,..., p-3$ and $\be_k=b_k-c_{p-3+k}$ for $k=1,\ldots ,r$, it follows from Theorem \ref{thm1} (1) that one has
$$t(X)= \sum_{j=1}^{p-3}\al_j-\sum_{k=1}^r\be_k.$$
\begin{rk}
\label{rk1}
It is not true that for any reduced surface $X$ one has $\al_j \geq 1$ for $j=1,\ldots, p-3$ and $\be_k \geq 1$ for $k=1,\ldots ,r$, see Examples
\ref{ex4}, \ref{ex4.1} and \ref{ex5}. This is surprising if compared with \eqref{res2B}.
 \end{rk}
 
 \begin{rk}
\label{rk2}
A reduced plane curve is said to be {\it Tjurina maximal} if $2d_1' \geq d$ and 
$$\tau(C)= (d-1)^2-d'_1(d-d'_1-1)-{2d'_1+2-d \choose 2}.$$
It is known that for such a curve one has $\epsilon_j=1$ for all $j=1,\ldots, q'$ and $d_1'=\ldots =d_{p'}$, see \cite[Theorem 3.1]{maxTjurina}.
It is interesting to note that the Cayley and the Kummer surfaces considered in Examples \ref{ex10} and \ref{ex20} enjoy similar properties for their graded Betti numbers, that is the $d_i$'s are all equal, and
$\al_j=\be_k=1$ for $j=1, \ldots, d-3$ and $k=1,\ldots ,r$.
It would be nice to have a theoretical explanation for this fact.
\end{rk}
The nodal surface of degree $d=8$ in Example \ref{ex30} has two values for the degrees
$d_i$'s, namely one has in this case ${\bf d}=(7_6,9_9)$.
A partial explanation of these two values is given by the following result.
Before stating it, we need some notation. 
Consider the graded $S$-module of {\it all Jacobian relations} of $f$ or, equivalently, the module of derivations killing $f$, namely
\begin{equation}
\label{eqD0}
AR(f)= \{\theta \in Der(S) \ : \ \theta(f)=0\}= \{\rho=(a_0,a_1, \ldots, a_n) \in S^{n+1} \ : \  \sum_0^na_jf_j=0\}.
\end{equation}
Inside this $S$-module, there is the graded $S$-submodule of {\it Koszul type relations} $KR(f)$ generated by the $N$ Koszul type derivations $\theta_{i,j}$ of degree $d-1$, where
$$N=\binom{4}{2}=6$$
and $\{i,j\}$ is a subset of $\{x,y,z,t\}$ having 2 elements and
$$\theta_{i,j}(f)=f_i f_j-f_jf_i=0.$$
Let $ER(f)=AR(f)/KR(f)$ be the quotient module and let
$$mdr(f)=\min \{k \ : \ ER(f)_k \ne 0\}.$$
  \begin{prop} 
\label{propN}
Let $X:f=0$ be a nodal surface in $\PP^3$ of degree $d \geq 5$.
Then
$$mdr(f) \geq 2d-  \floor[\Big]{ \frac{d}{2} } -3 > d-1.$$
In particular, the graded Betti numbers $\bf d$ of the surface $X$ satisfy
$$d_i=d-1 \text{ for } i=1, \ldots, N$$
and
$$d_i \geq 2d-   \floor[\Big]{ \frac{d}{2} } -3$$
for $i>N$.
\end{prop}
\proof
The inequality
$$mdr(f) \geq 2d-  \floor[\Big]{ \frac{d}{2} } -3 $$
follows from \cite[Equation (4.5) and Corollary 4.3]{Ferrara}.
The inequality 
$$2d-  \floor[\Big]{ \frac{d}{2} } -3 \geq d-1$$
clearly holds for any $d \geq 5$.
Since the $N$ Koszul type derivations $\theta_{ij}$ are linearly independent in $AR(f)_{d-1}$, we need to have exactly $N$ elements of degree $d-1$ in any minimal set of generators for $AR(f)$. The other generators we add yield nonzero elements in $ER(f)$, and hence their degree is at least $mdr(f)$.
\endproof

\begin{rk}
\label{rk3}
(i) Note that in Example \ref{ex30} we have a nodal surface of degree $d=8$ and the corresponding module $AR(f)$ has $N=6$ generators of degree $d-1=7$ and the remaining generators have degree 
$d_i=9$ for $i>6$. Since in this case
$$2d-  \floor[\Big]{ \frac{d}{2} } -3=9,$$
we see that our Proposition \ref{propN} is sharp.

(ii) A similar result to Proposition \ref{propN} may be stated in the more general situation when the surface $X$ has only isolated weighted homogeneous singularities, by using \cite[Theorem 9]{DSa}, which extends
the results in \cite{Ferrara} to this setting.
\end{rk}
   
 \section{ Examples}
 

\begin{ex}
\label{ex1}
Consider the cubic surface
$$X: f=xyz-t^3=0.$$
A direct computation using CoCoA or SINGULAR shows that
the corresponding minimal resolution for $M(f)$ is given by
$$0 \to  S(-5)^2 \to S(-3)^2 \oplus  S(-4)^3 \to  S(-2)^4 \to S.$$
It follows that $p=5$, $q=2$ and $r=0$, and the graded Betti numbers of this surface are
$$ {\bf d}=(1_2,2_3) \text{ and } {\bf c}=(3_2),$$
with the convention that $u_v$ means $u$ is repeated $v$ times.
The surface has 3 singularities $A_3$, hence $\tau(X)=6$.
All the claims in Theorem \ref{thm1} hold. With the notation from Remark \ref{rk1} one has
$$\al_1=c_1-d_4=3-2=1 \text{ and } \al_2=c_2-d_5=3-2=1.$$
\end{ex}

\begin{ex}
\label{ex2}
Consider the cubic surface
$$X: f=txz + y^2z+x^3-z^3=0.$$
A direct computation  shows that
the corresponding minimal resolution for $M(f)$ is given by
$$0 \to S(-6) \to  S(-5)^5 \to S(-3) \oplus  S(-4)^6 \to  S(-2)^4 \to S.$$
It follows that $p=7$, $q=5$ and $r=1$, and the graded Betti numbers of this surface are
$$ {\bf d}=(1,2_6),  \ {\bf c}=(3_5) \text{ and } {\bf b}=(4).$$
The surface has  $\tau(X)=5$.
All the claims in Theorem \ref{thm1} hold and  one has
$$\al_1=c_1-d_4=3-2=1, \  \al_2=c_2-d_5=3-2=1, \  \al_3=c_3-d_6=3-2=1, $$
$$ \al_4=c_4-d_7=3-2=1, \be_1=b_1-c_5=4-3=1.$$
\end{ex}

\begin{ex}
\label{ex3}
Consider the sextic surface
$$X: f=x^5z+y^6+x^4yt+xy^5=0.$$
A direct computation  shows that
the corresponding minimal resolution for $M(f)$ is given by
$$0  \to S(-9) \to S(-6) \oplus S(-7) \oplus  S(-8)^2 \to  S(-5)^4 \to S.$$
It follows that $p=4$, $q=1$ and $r=0$, and the graded Betti numbers of this surface are
$$ {\bf d}=(1,2,3_2) \text{ and } {\bf c}=(4).$$
The surface $X$ is nearly-free and has a 1-dimensional singular locus, with the Hilbert polynomial
$$P(M(f))(u)=16u-27.$$
All the claims in Theorem \ref{thm1} hold and one has
$\al_1=c_1-d_4=4-3=1. $
\end{ex}

\begin{ex}
\label{ex4}
Consider the surface of degree $d=9$ given by
$$X: f=(x^3-yzt)^3+(t^3-xyz)^3=0.$$
Then $X$  is a union of 3 cubic surfaces in $\PP^3$ and
a direct computation  shows that
the corresponding minimal resolution for $M(f)$ is given by
$$0 \to S(-18)\oplus S(-19)  \to S(-13)\oplus S(-16)\oplus S(-17)^3 \oplus S(-18)^2 \to $$
$$\to S(-9)  \oplus  S(-12)^2 \oplus S(-15)^2 \oplus S(-16)^3 \to  S(-8)^4 \to S.$$
It follows that $p=8$, $q=7$ and $r=2$, and the graded Betti numbers of this surface are
$$ {\bf d}=(1,4_2,7_2,8_3), \  {\bf c}=(5,8,9_3,10_2)  \text{ and } {\bf b}=(10,11).$$
All the claims in Theorem \ref{thm1} hold, in particular one has
$$P(M(f))(u)=38u-119.$$
With the notation from Remark \ref{rk1} one has
$$\be_1=b_1-c_6=10-10=0. $$
\end{ex}

\begin{ex}
\label{ex4.1}
Consider the surface of degree $d=12$ given by
$$X: f=(x^3-yzt)^4+(t^3-xyz)^4=0.$$
Then $X$  is a union of 4 cubic surfaces in $\PP^3$ and
a direct computation  shows that
the corresponding minimal resolution for $M(f)$ is given by
$$0 \to S(-25)^2 \to S(-16) \oplus S(-23)^2\oplus S(-24)^4 \to  $$
$$\to S(-12) \oplus S(-15)^2 \oplus  S(-22)^5 \to  S(-11)^4 \to  S.$$
It follows that $p=8$, $q=7$ and $r=2$, and the graded Betti numbers of this surface are
$$ {\bf d}=(1,4_2,11_5), \  {\bf c}=(5,12_2,13_4)  \text{ and } {\bf b}=(14_2).$$
All the claims in Theorem \ref{thm1} hold, in particular one has
$$P(M(f))(u)=81u-491.$$
With the notation from Remark \ref{rk1} one has
$$\al_1=c_1-d_4=5-11=-6, \ \al_2=\al_3=1, \ \al_4=\al_5=2$$
and
$$\be_1=b_1-c_6=14-13=1=b_2-c_7=\be_2. $$
It follows that $t(S)=-2 <0$ in this case.

\end{ex}

\begin{ex}
\label{ex5}
Consider the surface of degree $d=16$ given by
$$X: f=(x^4-yzt^2)^4+(t^4-xyz^2)^4=0.$$
Then $X$  is a union of 4 quartic surfaces in $\PP^3$ and
a direct computation  shows that
the corresponding minimal resolution for $M(f)$ is given by
$$0 \to S(-29)\oplus S(-30)\oplus S(-32)  \to S(-22)^2\oplus S(-23)\oplus S(-28)^4 \oplus S(-29)^3 \oplus S(-31)^2 \to $$
$$\to S(-20)^2  \oplus  S(-21)^3 \oplus S(-27)^4 \oplus S(-28)^2 \oplus S(-30) \to  S(-15)^4 \to S.$$
It follows that $p=12$, $q=12$ and $r=3$, and the graded Betti numbers of this surface are
$$ {\bf d}=(5_2,6_3,12_4,13_2,15), \  {\bf c}=(7_2,8,13_4,14_3,16_2)  \text{ and } {\bf b}=(14,15,17).$$
All the claims in Theorem \ref{thm1} hold, in particular one has
$$P(M(f))(u)=147u-1382.$$
With the notation from Remark \ref{rk1} one has
$$\al_9=c_9-d_{12}=14-15=-1,$$
$$\be_1=b_1-c_{10}=14-14=0,$$
and
$$\be_2=b_2-c_{11}=15-16=-1. $$
\end{ex}

%\section*{Conflict of Interests}
%We declare that there is no conflict of interest regarding the publication of this paper.

%\section*{Data Availability Statement}
%We do not analyze or generate any datasets, because this work proceeds within a theoretical and mathematical approach.

\begin{thebibliography}{00}

%\bibitem{Abe}  T. Abe, Plus-one generated and next to free arrangements of hyperplanes. \textit{Int. Math. Res. Not.} \textbf{2021(12)}: 9233 -- 9261 (2021).



\bibitem{CoCoA}
    {J. Abbott, A. M.  Bigatti  and L. Robbiano},
    {CoCoA  4.7.4 }: a system for doing {C}omputations in {C}ommutative {Algebra}. Available at {http://cocoa.dima.unige.it}
    
 \bibitem{ADP}  T. Abe, A. Dimca, P. Pokora, A new hierarchy for complex plane curves,
Canadian Mathematical Bulletin. Published online 2025:1-24. doi:10.4153/S0008439525101422   
    
    
\bibitem{AGV}
V. I. Arnold, S. M.  Gusein-Zade, A. N. Varchenko,  Singularities
  of Differentiable Maps. vols 1/2, Monographs in Math., \textbf{ 82/83},
Birkh\"auser, Basel (1985/1988)    
    
\bibitem{Chm}  S. V. Chmutov, Examples of projective surfaces with many singularities, J. Algebr. Geom. 1, 191--196 (1992).

\bibitem{Singular} { W. Decker, G.-M. Greuel, G. Pfister \and H. Sch{\"o}nemann.} \newblock {\sc Singular} {4-0-2} --- {A} computer algebra system for  polynomial computations. Available at {http://www.singular.uni-kl.de}.

\bibitem{DmaxS} A. Dimca, Freeness versus maximal degree of the singular subscheme for surfaces in $P^3$, 
Geom. Dedicata 183(2016), 101--112.

\bibitem{Dmax}  A. Dimca, Freeness versus maximal global Tjurina number for plane curves. \textit{Math. Proc. Cambridge Phil. Soc.}  \textit{163}:  161 -- 172 (2017).

\bibitem{Ferrara}  A. Dimca, 
On the syzygies and Hodge theory of nodal hypersurfaces, Ann. Univ. Ferrara Sez. VII Sci. Mat. 63 (2017), 87--101.

%\bibitem{CMreg} A. Dimca, On the Castelnuovo-Mumford regularity of curve arrangements.  \textit{Rev. Roumaine Math. Pures Appl}. 70 (2025),  73--84.

\bibitem{DSa} A. Dimca, M. Saito,
Generalization of theorems of Griffiths
and Steenbrink to hypersurfaces
with ordinary double points, Bull. Math. Soc. Sci. Math. Roumanie, 60(108) (2017), 351--371.

\bibitem{DStChm} A. Dimca, G. Sticlaru, On the syzygies and Alexander polynomials of nodal hypersurfaces,  Math. Nachrichten 285 (2012), 2120--2128.

\bibitem{DStExpo} A. Dimca, G. Sticlaru, On the exponents of free and nearly free projective plane curves, Rev. Mat. Complut. 30(2017), 259--268.

%\bibitem{DStRIMS} A. Dimca, G. Sticlaru, Free and nearly free curves vs. rational cuspidal plane curves. \textit{Publ. RIMS Kyoto Univ.} \textbf{54}: 163 -- 179 (2018).

\bibitem{3syz} A. Dimca and G. Sticlaru, Plane curves with three syzygies, minimal Tjurina curves curves, and nearly cuspidal curves. \textit{Geom. Dedicata} \textbf{207}: 29 -- 49 (2020).

\bibitem{maxTjurina} A. Dimca, G. Sticlaru, Jacobian syzygies, Fitting ideals, and plane curves with maximal global Tjurina numbers, Collect. Math. 
73 (2022), 391--409.

%\bibitem{Brian} A. Dimca, G. Sticlaru, Plus-one generated curves, Briançon-type polynomials and eigenscheme ideals,   Results Math.(2025) https://doi.org/10.1007/s00025-025-02371-z
 
%\bibitem{SameDeg} A. Dimca, G. Sticlaru, Curves with Jacobian syzygies of the same degree, Rendiconti del Circolo Matematico di Palermo Series 2, (2025) https://doi.org/10.1007/s12215-025-01203-x

\bibitem{HD}
A.~Dimca and G.~Sticlaru,
Bourbaki modules and the module of Jacobian derivations of projective hypersurfaces, arXiv:2506.23950.

\bibitem{Type3}
A.~Dimca and G.~Sticlaru,
On type three complex plane curves,
arXiv:2601.01824 [math.AG], 2026.

 

\bibitem{duPCTC} A.A. du Plessis,  C.T.C. Wall, Application of the theory of the discriminant
to highly singular plane curves, Math. Proc. Camb. Phil. Soc. 126
(1999) 256--266.

%\bibitem{duPCTC00} A.A. du Plessis,  C.T.C. Wall,  Singular hypersurfaces, versality and Gorenstein algebras, Jour. Alg. Geom. 9 (2000) 309--322.

\bibitem{duPCTC01} A.A. du Plessis,  C.T.C. Wall, Discriminants, vector fields and singular hypersurfaces. New developments in singularity theory (Cambridge, 2000), 351--377, NATO Sci. Ser. II Math. Phys. Chem., 21, Kluwer Acad. Publ., Dordrecht, 2001.

\bibitem{Eis} { D. Eisenbud}, \emph{The Geometry of Syzygies: A Second Course in Algebraic Geometry and Commutative Algebra}, Graduate Texts in Mathematics, Vol. 229, Springer 2005. 

\bibitem{E} Ph. Ellia,  Quasi complete intersections and global Tjurina number of plane curves, J. Pure Appl. Algebra 224 (2020),  423--431.

\bibitem{HS} S. H. Hassanzadeh, A. Simis, Plane Cremona maps: Saturation and regularity of the base ideal. \textit{J. Algebra} \textbf{371}: 620 -- 652 (2012).

\end{thebibliography}

\end{document}

\section{Preliminary facts}

The {\it Hilbert function} $H(M(f)): \N \to \N$ of the graded $S$-module $M(f)$ is defined by
\begin{equation}
\label{Hfunc}
 H(M(f))(k)= \dim M(f)_k,
\end{equation}
and it is often encoded in the {\it Hilbert-Poincar\'e series} of $M(f)$
\begin{equation}
\label{Hseries}
 HP(M(f);t)= \sum_k\dim M(f)_kt^k.
\end{equation}



Let $\Sigma= \Sigma_1 \cup \Sigma_0$, where $\Sigma_1$ denotes the components of the singular locus $\Sigma$ of dimension 1 as well as  their embedded 0-dimensional components, and $\Sigma_0$ denotes the union of the isolated points in $\Sigma$, with their multiple structure.
Then the points in $\Sigma_0$ corresponds to the isolated singularities of $S$ and their contribution to the Hilbert polynomial  $P(M(f))$ is again a constant $\tau_0(S)$, the total Tjurina number of $S$, which is just the sum of the individual Tjurina numbers of the isolated singularities of $S$.

The general theory of Hilbert polynomials says that the degree of  $P(M(f))$ is given by the dimension of the support of  $\OO_{\Sigma}=\tilde M(f)$. Hence the assumption $\dim \Sigma=1$ implies that
$$P(M(f))(k)=ak+b$$
where 
\begin{equation}
\label{abformula1}
a=\deg(\Sigma_1) \text { and } b= \chi(\Sigma_1, \OO_{\Sigma_1})+\tau_0(V).
\end{equation}
However, the calculation of $a$ and $b$ in general can be very difficult, due to the multiple structure and/or the singularities of the subscheme $\Sigma_1$.
